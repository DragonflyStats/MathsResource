\documentclass[11pt]{article} % use larger type; default would be 10pt

\usepackage[utf8]{inputenc} % set input encoding (not needed with XeLaTeX)

\usepackage{geometry} % to change the page dimensions
\geometry{a4paper} % or letterpaper (US) or a5paper or....

\usepackage{framed}
\usepackage{graphicx} % support the \includegraphics command and options
\usepackage{subfiles}
\usepackage{booktabs} % for much better looking tables
\usepackage{array} % for better arrays (eg matrices) in maths
\usepackage{paralist} % very flexible & customisable lists (eg. enumerate/itemize, etc.)
\usepackage{verbatim} % adds environment for commenting out blocks of text & for better verbatim
\usepackage{subfig}
\usepackage{fancyhdr} % This should be set AFTER setting up the page geometry
\pagestyle{fancy} % options: empty , plain , fancy
\renewcommand{\headrulewidth}{0pt} % customise the layout...
\lhead{MA4702 2016}\chead{MID TERM EXAM 2}\rhead{16th March 2016}
\lfoot{}\cfoot{\thepage}\rfoot{}
\voffset=-1.5cm
\oddsidemargin=0.0cm
\textwidth = 470pt
\usepackage{sectsty}
\allsectionsfont{\sffamily\mdseries\upshape} % (See the fntguide.pdf for font help)
\usepackage[nottoc,notlof,notlot]{tocbibind} % Put the bibliography in the ToC
\usepackage[titles,subfigure]{tocloft} % Alter the style of the Table of Contents
\renewcommand{\cftsecfont}{\rmfamily\mdseries\upshape}
\renewcommand{\cftsecpagefont}{\rmfamily\mdseries\upshape} % No bold!
\begin{document}
	
	
	\Large
\newpage

\begin{framed}
\begin{description}
\item[NAME]  ..........................................................................................
\bigskip
\item[STUDENT ID] ............................................................................
\bigskip
\item[DEGREE] ......................................................................................
\end{description}
\end{framed}
\noindent Are You Approved for Extra Time in Exams (LENS students only)  .......
\section*{Examination Procedures}

\begin{itemize} 
	\item This exam will start at 17:05, and will last 45 minutes.
	
	\item Each question will be worth either 2 or 3 Marks. There are 15 Marks worth of questions.
	\item All questions must be attempted (LENS students please see below)
	
	\item Write all of your answers in the exam script. Write the script number on any other documents you submit.
	
	\item It is your responsibility to return the script to collection box. An audit of scripts will take place immediately after the exam. If your script is account for in that audit,  you are deemed to be absent, and will receive no marks.
	
	\item \textbf{LENS Student}
	Specifically approved LENS students have to answer any selection of questions that have an aggregate mark of 11 Marks.  
	\begin{itemize}
		\item They may skip any two of the 2-Mark Questions
		%			\item OR - They may skip a 1-Mark Question and a 2-Mark Question
		\item The mark will be rescaled accordingly to give a result out of 15 marks.
	\end{itemize}
(Type : C)	
	
\end{itemize}
%============================================================================ %

\newpage


\LARGE
\section*{Formula Sheet}


\begin{itemize}
	\item Horizontal Asymptote
	\[ \lim_{x \to \infty } f(x) \]
	\item Maclaurin Series
	\[f(0) + f^{\prime}(0) + \frac{f^{\prime \prime}(0)}{2!} + \frac{f^{\prime \prime}(0)}{2!} + \frac{f^{\prime\prime \prime}(0)}{3!} + \ldots \]
\end{itemize}

%\subsection*{Difference of Two Cubes}
%\[ a^3 - b^3 = (a-b)(a^2 + ab + b^2)\]
%\subsection*{Sequences and Series}
%\[ \sum_{i=1}^{n} i = \frac{n(n+1)}{2}\]
%
%\[ S_n = \frac{n}{2} \left(2a + (n-1) d \right)\]
%
%\[ S_n = a\left(\frac{1-r^n}{1-r}\right)\]
%
%\[ S_\infty = \frac{a}{1-r}\]
\newpage




%=====================================================%
%% Inverse of Functions




\begin{enumerate}[(i)]
	\item (2 Marks) Suppose the general term $u_n$ is given as \[u_n = \displaystyle{\frac{3^n}{n!} } .\]State $u_{n+1}$ and hence calculate a simplified expression for $r$, where 
	{
		\Large
		\[ r = \frac{u_{n+1}}{u_n}\]
	}
\vspace{8.8cm}

	\item (2 Mark) Given the function $g(x) = \sqrt{x-3}$ find the inverse of the function $g^{-1}(x)$.

%% Answer - $x^2+3 %%
\newpage


	
	\newpage
	%=========================================================%
		
\newpage

\item (2 Marks) You are given the following functions $g(x)$ and $f(x)$:
\[g(x)= x^2 - 2x  +1\]    \[f(x) = \sqrt{x-1}\] Determine the values of $f \circ g(5)$ and $g \circ f(5)$. Show your workings.\\ (\textit{You can express any relevant answers as square root terms .i.e. in form ``$\sqrt{a}$" where $a$ is some integer}.)

\newpage
	
	
%=====================================================%
	%% Domain and Range READY
	\item (2 Mark) Find the domain and the range of the function:
	
	\[ f(x) = 4 sin(x)  + 5 \]
%%  - X and y intercept
	\item (3 Marks) Find the x-intercepts and the y-intercept of the following function

\[ f(x) = x^2 + 6x + 8  \]

%=====================================================%
%% - Horizontal and Vertical Asymptotes
%% - http://media.wix.com/ugd/70b98b_39a909a40ea3098062c0bebfb4304ed6.pdf
%% - READY	
	\newpage
	
	\item (2 Marks) Determine the vertical asymptote(s) of the following function


\[ y = f(x) = \frac{2x^2}{x^2-16} \]	

\vspace{5.8cm}
	
	\item (2 Marks) Determine the horizontal asymptote(s) of the following function
	
	\[ y = f(x) = \frac{2x^2}{x^2-16} \]
	
\end{enumerate}
%======================================================== %
\end{document}
