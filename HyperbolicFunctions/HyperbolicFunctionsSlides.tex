\documentclass{beamer}

\usepackage{amsmath}
\usepackage{amssymb}

\begin{document}
\begin{frame}
	\frametitle{Hyperbolic Functions}	
	hyperbolic functions are analogs of the ordinary trigonometric, or circular, functions. 
	\begin{itemize}
	\item The basic hyperbolic functions are the hyperbolic sine "sinh" and the hyperbolic cosine "cosh"  from which are derived the hyperbolic tangent "tanh" (/ˈtæntʃ/ or /ˈθæn/),[3] hyperbolic cosecant "csch" or "cosech" , hyperbolic secant "sech" , and hyperbolic cotangent "coth" corresponding to the derived trigonometric functions. 
	\item The inverse hyperbolic functions are the area hyperbolic sine "arsinh" (also called "asinh" or sometimes "arcsinh")and so on.
	\end{itemize}
	
\end{frame}
%================================================================ %
\begin{frame}
	\frametitle{Hyperbolic Functions}
\begin{itemize}
\item Just as the points (cos t, sin t) form a circle with a unit radius, the points (cosh t, sinh t) form the right half of the equilateral hyperbola.\item  The hyperbolic functions take a real argument called a hyperbolic angle. 
\item The size of a hyperbolic angle is the area of its hyperbolic sector. 
\item The hyperbolic functions may be defined in terms of the legs of a right triangle covering this sector.
\end{itemize}

\end{frame}
%================================================================ %
\begin{frame}
\frametitle{Hyperbolic Functions}
\begin{itemize}
\item Hyperbolic functions occur in the solutions of some important linear differential equations, for example the equation defining a catenary, of some cubic equations, and of Laplace's equation in Cartesian coordinates.
\item The latter is important in many areas of physics, including electromagnetic theory, heat transfer, fluid dynamics, and special relativity.
\end{itemize}
\end{frame}
%================================================================ %
\begin{frame}
	\frametitle{Hyperbolic Functions}

{
	\Large
\[sinh(x) =  \frac{e^x - e^{-x}}{2}\]

\[cosh(x) = \frac{e^x + e^{-x}}{2}\]
}

\end{frame}
%================================================================ %
\begin{frame}
	\frametitle{Hyperbolic Functions}

\[csch(x) = 1/sinh(x) = \frac{2}{e^x - e^{-x}}\]


\[sech(x) = 1/cosh(x) = \frac{2}{e^x + e^{-x}}\]

\[tanh(x) = sinh(x)/cosh(x) = \frac{e^x - e^{-x}}{e^x + e^{-x}}\]

\[coth(x) = 1/tanh(x) = ( ex + e-x)/( ex - e-x )\]
\end{frame}
%================================================================ %
\begin{frame}
	\frametitle{Hyperbolic Functions}

\[cosh^2(x) - sinh^2(x) = 1\]

\[tanh^22(x) + sech^2(x) = 1\]

\[coth^22(x) - csch^2(x) = 1\]
\end{frame}
%===================================================================================================== %
\begin{frame}
Odd and even functions:
\begin{align}
\sinh (-x) &= -\sinh x \\
\cosh (-x) &=  \cosh x
\end{align}
\end{frame}

\begin{frame}
Hence:
\begin{align}
\tanh (-x) &= -\tanh x \\
\coth (-x) &= -\coth x \\
\operatorname{sech} (-x) &=  \operatorname{sech} x \\
\operatorname{csch} (-x) &= -\operatorname{csch} x
\end{align}
\end{frame}

\end{document}