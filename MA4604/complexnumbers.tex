%%- http://tutorial.math.lamar.edu/Classes/Alg/ComplexNumbers.aspx

Example 1  Perform the indicated operation and write the answers in standard form.
\item[(a)] 
\item[(b)] 
\item[(c)] 


Example 2  Multiply each of the following and write the answers in standard form.
\item[(a)]    
\item[(b)]    
\item[(c)]    
\item[(d)]    


Example 3  Write each of the following in standard form.
\item[(a)]    
\item[(b)]    
\item[(c)]    
\item[(d)]    

%====================================================================%
MA4604: Science Maths 4, Homework Week 1
\begin{enumerate}
\item Write each of the complex numbers below in the form a + ib, that is simplify each
expression to find the real numbers a and b.
\item[(a)] (3 + 8i) + (2i − 6) − (−5 + i) \item[(b)] (3 − i)(5 + 6i) \item[(c)] (1 + i)(2 − 5i)(7 + 3i)
\item[(d)] 3 − 4i
2 − i
\item[(e)] 5 − i
2 − 3i
\item[(f)] 2 + 4i
i(1 − i)
.
\item Find the real number(s) t that makes each expression below real:
\item[(a)] (4 + 6i)(3i)(t + 6i) \item[(b)] i(t + 4i)
2
\item[(c)] 5 − 10i
4i + t
.
\item Solve the complex equation 2z + i$\bar{z}$ = 5 + 4i (hint: write z = x + iy).
\item Find all the roots of, and hence factorise fully, each given polynomial:
\item[(a)] x
2 − 8x + 25 \item[(b)] z
2 + (4 + 3i)z + 14 + 6i \item[(c)] x
4 − 16 \item[(d)] x
3 + 8.
\item Given that x = 1 + 4i is a root of the quartic polynomial x
4 + 13x
2 + 34x, find the other
three roots and write it as a product of linear factors.
\item Plot each of the given complex numbers in an Argand diagram:
\item[(a)] 2 + i \item[(b)] −1 + 3i \item[(c)] −2i \item[(d)] −2 − 2i
\item[(e)] 26
\pi
4
\item[(f)] 46
\pi
2
(g) 36
\pi
3
(h) 56 \pi
(i) 4 e
− i\pi
4 (j) 2e
− i\pi
2 (k) 3 e
4i\pi
3 (l) √
3 e
5i\pi
3 .
\item Write each of the complex numbers in Q.6 \item[(e)] through (l) in the form x + iy, that is
convert them from polar or exponential form to standard form.
\end{enumerate}

\newpage
%============================================================================%
MA4604: Science Maths 4, Homework Week 2
\bein{enumerate}
\item Write each of the complex numbers below in polar form r6 θ and in exponential form reiθ
(hint: first find r = |z| and θ = arg(z)):
\item[(a)] z = −2 \item[(b)] z = −5i \item[(c)] z = 3 − 3i \item[(d)] z = −3 − 3i.
\item For each complex number given below in exponential form reiθ, find its absolute value |z|
and its argument Arg(z); express z in the form x + iy:
\item[(a)] 3 e
i\pi
2 \item[(b)] 2 e
i\pi
3 \item[(c)] 4 e
− i\pi
4 \item[(d)] 6 e
2i\pi \item[(e)] 5 e
−i\pi
.
\item Use de Moivre’s theorem to simplify the given powers: \item[(a)] (1 + i)
20 \item[(b)] (1 + i
√
3)12
.
\item Use de Moivre’s theorem with n = 3 to write cos(3θ) in terms of cos θ and to write sin(3θ)
in terms of sin θ (hint: you’ll also need to use the fact that cos2
θ + sin2
θ = 1).
\item Find the twelve twelfth roots of unity in exponential and in Cartesian form.
\item By first writing the given number in complex exponential form, evaluate each of the
following in Cartesian form: \item[(a)] all cube roots of 27; \item[(b)] all cube roots of −64;
\item[(c)] all cube roots of −64i; \item[(d)] all fourth roots of 81; \item[(e)] all fourth roots of −4;
\item[(f)] all square roots of −9i.

\end{enumerate}
