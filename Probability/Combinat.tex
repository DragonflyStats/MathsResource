\documentclass[]{report}

\voffset=-1.5cm
\oddsidemargin=0.0cm
\textwidth = 480pt

\usepackage{framed}
\usepackage{subfiles}
\usepackage{enumerate}
\usepackage{graphics}
\usepackage{newlfont}
\usepackage{eurosym}
\usepackage{amsmath,amsthm,amsfonts}
\usepackage{amsmath}
\usepackage{color}
\usepackage{amssymb}
\usepackage{multicol}
\usepackage[dvipsnames]{xcolor}
\usepackage{graphicx}
\begin{document}

	
	%--------------------------------------------------------%
	{
		\subsection{Permutations and Combinations}
		
		
		Often we are concerned with computing the number of ways of selecting and arranging groups of items. \begin{itemize} \item  A \textbf{\emph{combination}} describes the selection of items from a larger group of items.  \item A \textbf{\emph{permutation}} is a combination that is arranged in a particular way.
		\end{itemize}
		
		\bigskip
		\begin{itemize}
			\item Suppose we have items A,B,C and D to choose two items from.
			\item AB is one possible selection, BD is another. AB and BD are both combinations.
			\item More importantly, AB is one combination, for which there are two distinct permutations: AB and BA.
		\end{itemize}
	}
	
	%--------------------------------------------------------%
	{
		\subsection{Combinations}
		
		\textbf{Combinations: }
		The number of ways of selecting $k$ objects from $n$ unique objects is:
		
		\[ ^n C_k = {n!  \over k! \times (n-k)!} \]
		
		In some texts, the notation for finding the number of possible combination is written
		
		\[ ^n C_k =  {n \choose k} \]
		
	}
	
	%--------------------------------------------------------%
	{
		\subsection{Example of Combinations}
		How many ways are there of selecting two items from possible 5?
		
		\[ ^5 C_2   \left( \mbox{ also }  {5 \choose 2}  \right) =  {5!  \over 2! \times 3!} =  {5 \times 4 \times 3!  \over 2 \times 1 \times 3!} = 10  \]
		
		\bigskip
		Discuss how combinations can be used to compute the number of rugby matches for each group in the Rugby World Cup.
		
	}
	%--------------------------------------------------------%
	{
		\subsection{The Permutation Formula}
		The number of different permutations of r items from n unique items is written as $^n P_k$
		
		
		\[ ^n P_k = \frac{n!}{(n-k)!}\]
	}
	
	%--------------------------------------------------------%
	{
		\subsection{Permutations}
		\textbf{Example:}
		How many ways are there of arranging 3 different jobs, between 5 workers, where each worker can only do one job?
		
		
		\[ ^5 P_3 = \frac{5!}{(5-3)!}  = {5! \over 2!} = 60\]
		
	}
	
	
	

	
	%--------------------------------------------------------%
	[fragile]
	\subsection{Using \texttt{R}}
	When implementing combination calculations in \texttt{R}, we use the \texttt{choose()} function.
	
	\begin{verbatim}
	> choose(5,0)
	[1] 1
	> choose(5,1)
	[1] 5
	> choose(5,2)
	[1] 10
	> choose(5,3)
	[1] 10
	> choose(5,4)
	[1] 5
	> choose(5,5)
	[1] 1
	\end{verbatim}
	
	
	%--------------------------------------------------------%
	{
		\subsection{Example of Combinations}
		
		\textbf{Part 2}
		
		Thus a committee of 2 men and 2 women can be selected in $ 6 \times 3  = 18 $ ways.\\
		\bigskip
		\textbf{Part 3}
		
		The probability of two men and two women on a committee is
		\[ {\mbox{Number of ways of selecting 2 men and 2 women} \over \mbox{Number of ways of selecting 4 from 7}} = {18 \over 35 }\]
		
	}

	%--------------------------------------------------------%
	{
		\subsection{Example of Combinations}
		
		\textbf{Part 4}
		\begin{itemize}
			\item We have to compute the number of ways of selecting 1 male from 4 (4 ways) and the number of ways of selecting three females from 2 ( only 1 way)
			\item The probability of selecting three females is therefore ${4 \times 1 \over 35} = 4/35$
			\item So using the addition rule
			\[ Pr(\mbox{ at least 2 females }) = Pr(\mbox{ 2 females }) + Pr(\mbox{ 3 females }) \]
			\[ Pr(\mbox{ at least 2 females })  = 18/35 + 4/35 = 22/35 \]
		\end{itemize}
		
	}
	
	%--------------------------------------------------------%
	\section{ Combinations and Permutations }
	
		
{The Permutation Formula}
		The number of different permutations of r items from n unique items is written as $^n P_k$
		
		
		\[ ^n P_k = \frac{n!}{(n-k)!}\]

	%--------------------------------------------------------%
	{
		{Permutations}
		\t{Example:}
		How many ways are there of arranging 3 different jobs, between 5 workers, where each worker can only do one job?
		
		
		\[ ^5 P_3 = \frac{5!}{(5-3)!}  = {5! \over 2!} = 60\]
		

		{Example of Combinations}
		
		\t{Part 2}
		
		Thus a committee of 2 men and 2 women can be selected in $ 6 \times 3  = 18 $ ways.\\
		\bigskip
		\t{Part 3}
		
		The probability of two men and two women on a committee is
		\[ {\mbox{Number of ways of selecting 2 men and 2 women} \over \mbox{Number of ways of selecting 4 from 7}} = {18 \over 35 }\]
		
	
		{Example of Combinations}
		
		\t{Part 4}
		\begin{itemize}
			\item The probability of at least two females is the probability of 2 females or 3 females being selected.
			\item We can use the addition rule, noting that these are two mutually exclusive events.
			\item From before we know that probability of 2 females being selected is 18/35.
		\end{itemize}
		
	}
	%--------------------------------------------------------%
	{
		{Example of Combinations}
		
		\t{Part 4}
		\begin{itemize}
			\item We have to compute the number of ways of selecting 1 male from 4 (4 ways) and the number of ways of selecting three females from 2 ( only 1 way)
			\item The probability of selecting three females is therefore ${4 \times 1 \over 35} = 4/35$
			\item So using the addition rule
			\[ Pr(\mbox{ at least 2 females }) = Pr(\mbox{ 2 females }) + Pr(\mbox{ 3 females }) \]
			\[ Pr(\mbox{ at least 2 females })  = 18/35 + 4/35 = 22/35 \]
		\end{itemize}
		
	}
	
	% http://www.mathsisfun.com/combinatorics/combinations-permutations.html
	
	
	
	%==================================================================%
	
	\section{Techniques for Counting}
	
	%==================================================================%
	
	\begin{itemize}
		\item Combinations
		\item Permutations
		\item Permutations with constraints
	\end{itemize}
	
	\section{ Combinations and Permutations }
	

	

	\bigskip
	Discuss how combinations can be used to compute the number of rugby matches for each group in the Rugby World Cup.
	
	\subsection{The Permutation Formula}
	The number of different permutations of r items from n unique items is written as $^n P_k$
	
	
	\[ ^n P_k = \frac{n!}{(n-k)!}\]
	
	
	%--------------------------------------------------------%
	{
		
		\noindent \textbf{Example:}
		How many ways are there of arranging 3 different jobs, between 5 workers, where each worker can only do one job?
		
		
		\[ ^5 P_3 = \frac{5!}{(5-3)!}  = {5! \over 2!} = 60\]
		
	}
	
	
	
	
	
	
	
	%==================================================================%
	
	\noindent \textbf{Permutations of subsets}
	
	The number of permutations of subsets of $k$ elements selected from a set of $n$ different elements is
	
	\[P(n,r) = \frac{n!}{(n-k)!}  \]
	
	
	%==================================================================%
	
	\noindent \textbf{Combinations of subsets}
	
	The number of combinations that can be selected from $n$ items is
	
	\[ {n \choose k} = \frac{n!}{k! \times (n-k)!}  \]
	

	
	Discuss how combinations can be used to compute the number of rugby matches for each group in the Rugby World Cup.
	
	%%- \subsection{The Permutation Formula}
	The number of different permutations of r items from n unique items is written as $^n P_k$
	
	
	\[ ^n P_k = \frac{n!}{(n-k)!}\]
	
	%%- \subsection{Permutations}
	\textbf{Example:}
	How many ways are there of arranging 3 different jobs, between 5 workers, where each worker can only do one job?
	
	
	\[ ^5 P_3 = \frac{5!}{(5-3)!}  = {5! \over 2!} = 60\]
	
	
	\subsection{Combinations}
	In mathematical terms, a combination is an subset of items from a larger set such that the order of the items does not matter.
	
	\subsection{Permutations}
	
	There are two types of permutation:
	\begin{enumerate}
		\item Repetition is Allowed: such as the lock above. It could be "333".
		\item No Repetition: for example the first three people in a running race. You can't be first and second.
	\end{enumerate}
	
	
	
	\begin{framed}
		\begin{itemize}
			\item Permutations where repetition is allowed: 
			\[ n! \]
			\item Permutations where repetition is not allowed
			\[ \frac{n!}{(n-k)!} \]
		\end{itemize}
	\end{framed}
	
	
	
	
	\subsection{Permutations}
	
	
	\textbf{Part 2 : PERMUTE}\\
	\begin{itemize}
		\item We re-express the answer from part 2 as follows:
		
		\[\frac{7!}{2!} =  \frac{5040}{2} = \boldsymbol{2520} \]
	\end{itemize}
	
	\textbf{Part 4 : LITTLE}\\
	\begin{itemize}
		\item The word LITTLE has 6 letters, but there are two Ls and two Ts.
		\item From before, there are 6! ways to arrange 6 letters.
		\item Again, interchanging the two Ls and Ts does not result in a new permutation. 
		
		\[\frac{6!}{2!\times 2!} =  \frac{720}{4} = \boldsymbol{180} \]
	\end{itemize}
	%====================================================================================%
	
	\begin{itemize}
		\item In how many permutations are there of counting a subset of k elements, when there are $n$ elements in total.
		
		\item The number of permutations of a set of n elements is denoted n! (pronounced n factorial.)
	\end{itemize}
	
	%------------------------------------------------------ %
	\subsection{Permutations}
	
	\begin{itemize}
		\item The notion of permutation relates to the act of permuting (rearranging) objects or values. 
		\item Informally, a permutation of a set of objects is an arrangement of those objects into a particular order. 
		
		\item For example, there are six permutations of the set $\{1,2,3\}$, namely (1,2,3), (1,3,2), (2,1,3), (2,3,1), (3,1,2), and (3,2,1). 
		\item As another example, an anagram of a word is a permutation of its letters. 
		
	\end{itemize}
	
	If the probability of C is $70 \%$ then the probability of $C^{\prime}$ is $30\%$		
	
	%================================================================ %
	
	\subsection*{Permutations : Worked Example}
	
	How many anagrams (permutations of the letters) are there of the following words
	
	\begin{framed}
		\begin{multicols}{2}
			\begin{enumerate}
				\item ANSWER
				\item PERMUTE
				\item ANAGRAM
				\item LITTLE
			\end{enumerate}
		\end{multicols}
	\end{framed}
	
	
	\textbf{Part 1 : ANSWER}\\
	Some possible outcomes:
	\begin{center}
		ASNWRE,\;
		SANERW,\;
		REWSAN,\;...
	\end{center}
	
	Since ANSWER has 6 distinct letters, the number of permutations (anagrams) is
	
	\[6! = 6\times 5 \times 4 \times 3 \times 2\times 1 = \boldsymbol{720} \]
	
	%==================================================================================%
	\textbf{Part 2 : PERMUTE}\\
	\begin{itemize}
		\item[$\bullet$] The word PERMUTE has 7 letters, but only 6 different letters. 
		\item[$\bullet$] There are 7! ways to arrange 7 letters.
		\item[$\bullet$] However, interchanging the two Es does not result in a new permutation. There would be two identical anagrams.
	\end{itemize}
	
	\begin{center}
		P\textcolor{red}{E}RMUT\textcolor{blue}{E}, \; MUT\textcolor{red}{E}P\textcolor{blue}{E}R, \; P\textcolor{red}{E}T\textcolor{blue}{E}MUR,\; ..\\
		P\textcolor{blue}{E}RMUT\textcolor{red}{E}, \; MUT\textcolor{blue}{E}P\textcolor{red}{E}R, \; P\textcolor{blue}{E}T\textcolor{red}{E}MUR,\; ..
	\end{center}
	
	
	\begin{itemize}
		\item[$\bullet$]  The number of permutations (anagrams) is half of 7! .
		
		\[\frac{7!}{2} =  \frac{5040}{2} = \boldsymbol{2520} \]
	\end{itemize}
		%==================================================================================%
	\textbf{Part 2 : VENEER}\\
	\begin{itemize}
		\item[$\bullet$] The word VENEER has 6 letters, but only 4 different letters. 
		\item[$\bullet$] There are 6! ways to arrange 6 letters.
		\item[$\bullet$] However, interchanging the three Es does not result in a new permutation. There would be six identical anagrams.
	\end{itemize}
	
	\begin{center}
V\textcolor{red}{E}R\textcolor{green}{E}N\textcolor{blue}{E}, \; N\textcolor{green}{E}R\textcolor{red}{E}V\textcolor{blue}{E}R, \; \textcolor{green}{E}V\textcolor{red}{E}\textcolor{blue}{E}NR,\; ..\\
V\textcolor{red}{E}R\textcolor{blue}{E}N\textcolor{green}{E}, \; N\textcolor{blue}{E}R\textcolor{red}{E}V\textcolor{green}{E}R, \; \textcolor{blue}{E}V\textcolor{red}{E}\textcolor{green}{E}NR,\; ..\\
V\textcolor{green}{E}R\textcolor{red}{E}N\textcolor{blue}{E}, \; N\textcolor{red}{E}R\textcolor{green}{E}V\textcolor{blue}{E}R, \; \textcolor{red}{E}V\textcolor{green}{E}\textcolor{blue}{E}NR,\; ..\\
V\textcolor{green}{E}R\textcolor{blue}{E}N\textcolor{red}{E}, \; N\textcolor{blue}{E}R\textcolor{green}{E}V\textcolor{red}{E}R, \; \textcolor{blue}{E}V\textcolor{green}{E}\textcolor{red}{E}NR,\; ..\\

V\textcolor{blue}{E}R\textcolor{red}{E}N\textcolor{green}{E}, \; N\textcolor{red}{E}R\textcolor{blue}{E}V\textcolor{green}{E}R, \; \textcolor{red}{E}V\textcolor{blue}{E}\textcolor{green}{E}NR,\; ..\\
V\textcolor{blue}{E}R\textcolor{green}{E}N\textcolor{red}{E}, \; N\textcolor{green}{E}R\textcolor{blue}{E}V\textcolor{red}{E}R, \; \textcolor{green}{E}V\textcolor{blue}{E}\textcolor{red}{E}NR,\; ..\\
	\end{center}
	
	
	\begin{itemize}
		\item[$\bullet$]  The number of permutations (anagrams) is half of 7! .
		
		\[\frac{7!}{2} =  \frac{5040}{2} = \boldsymbol{2520} \]
	\end{itemize}
	%==================================================================================%	
	
	\textbf{Part 3 : ANAGRAM}\\
	\begin{itemize}
		\item The word ANAGRAM has 7 letters, but there are three As.
		\item From before, there are 7! ways to arrange 7 letters.
		\item How many new permutations are found by re-arranging the As?
	\end{itemize}
	
	\begin{multicols}{2}
		\begin{itemize}
			\item[(i)]	\textcolor{red}{A}N\textcolor{blue}{A}GR\textcolor{green}{A}M 
			\item[(ii)]		\textcolor{red}{A}N\textcolor{green}{A}GR\textcolor{blue}{A}M 
			\item[(iii)] 		\textcolor{blue}{A}N\textcolor{red}{A}GR\textcolor{green}{A}M  
			\item[(iv)]		\textcolor{green}{A}N\textcolor{red}{A}GR\textcolor{blue}{A}M 
			\item[(v)]	\textcolor{blue}{A}N\textcolor{green}{A}GR\textcolor{red}{A}M 
			\item[(vi)]		\textcolor{green}{A}N\textcolor{blue}{A}GR\textcolor{red}{A}M 
		\end{itemize}
	\end{multicols}
	
	\begin{itemize}
		\item We divide 7! by 3! to account for the identical anagrams.
		
		\[\frac{7!}{3!} =  \frac{5040}{6} = \boldsymbol{840} \]
	\end{itemize}
	
	%-------------------------------------%
	
	
	
	\subsection{Permutation Formula}
	
	A formula for the number of possible permutations of k objects from a set of n. This is usually written $^nP_k$ .
	
	
	\noindent\textbf{Formula:}	
	\[ ^nP_k = \frac{n!}{(n-k)!} =  n.(n-1).(n-2).\ldots(n-k+1) \]
	
	
	
	\noindent \textbf{Example:}\\	
	How many ways can 4 students from a group of 15 be lined up for a photograph?\\
	
	%--------------- %
	\noindent \textbf{Answer:	}\\
	There are $^{15}P_4$ possible permutations of 4 students from a group of 15.
	\[ ^{15}P_4 = \frac{15!}{11!} = 15\times 14\times 13\times 12 = 32760 \]
	There are 32760 different lineups.
	
	
	
	
	
	
	
	
	\subsection*{Summary}
	\begin{itemize}
		\item If the order doesn't matter, it is a Combination.
		\item If the order does matter it is a Permutation.
	\end{itemize}
	
	
	
	
	%============================================================== %		
	
	
	
	Probability of $x$ \emph{given} that $y$ has occured.
	
	%\subsection{Computing binomial Coefficients with R}
	%When implementing combination calculations in \texttt{R}, we use the \texttt{choose()} function.
	%\begin{framed}	
	%	\begin{verbatim}
	%	> choose(5,0)
	%	[1] 1
	%	> choose(5,1)
	%	[1] 5
	%	> choose(5,2)
	%	[1] 10
	%	> choose(5,3)
	%	[1] 10
	%	> choose(5,4)
	%	[1] 5
	%	> choose(5,5)
	%	[1] 1
	%	\end{verbatim}
	%\end{framed}	
	
	
	
	
	\textbf{Permuations of subsets}
	
	The number of permutations of subsets of $k$ elements selected from a set of $n$ different elements is
	
	\[P(n,r) = \frac{n!}{(n-k)!}  \]
	
	
	\textbf{Combinations of subsets}
	
	The number of combinations that can be selected from $n$ items is
	
	\[ {n \choose k} = \frac{n!}{k! \times (n-k)!}  \]
	
	
	
	\section{ Combinations and Permutations }
	
	
	

	
	\bigskip
	Discuss how combinations can be used to compute the number of rugby matches for each group in the Rugby World Cup.
	
	\textbf{The Permutation Formula}
	The number of different permutations of r items from n unique items is written as $^n P_k$
	
	
	\[ ^n P_k = \frac{n!}{(n-k)!}\]
	
	
	
	\textbf{Permutations}
	\textbf{Example:}
	How many ways are there of arranging 3 different jobs, between 5 workers, where each worker can only do one job?
	
	
	\[ ^5 P_3 = \frac{5!}{(5-3)!}  = {5! \over 2!} = 60\]
	
	
	\textbf{Permutations with Constraints}
	
	How many different four digit numbers greater than 5000 can be formed from the digits \[2,4,5,8,9\] if each digit can only be used once in any given number.
	
	\textbf{Permutations with Constraints}
	
	How many of these four digit numbers are odd, given they are greater than 5000?
	\[2,4,5,8,9\]\\
	
	
	\textbf{Permutations with Constraints}
	
	
	%http://www.mathsireland.com/LCHGeneralNotes/PermCombProb/5_5_Prob_MultAnd/Q_5_5_Prob_MultAnd.html
	
	

	
	
	
	
	\section{Combinations}
	
	\textbf{Combinations}
	\begin{itemize}
		\item In the case of permutations, the order in which the objects are arranged is important.
		\item In the case of
		combinations, we are concerned with the number of different groupings of objects that can occur without regard
		to their order. 
		\item Therefore, an interest in combinations always concerns the number of different subgroups that
		can be taken from n objects. The number of combinations of n objects taken r at a time is
	\end{itemize}	
	%=========================================================================== %	
	
	
	\subsection{Permutation Example}
	
	
	Suppose a four letter code is made from the letters \textbf{\textit{\{a,b,c,d,e\}}}, where repetitions are allowed and the order of the letters in the code is significant\\ \bigskip For example
	\textbf{\textit{a,a,e,c}} is a different code to \textbf{\textit{a,c,e,a}}.
	
	
	
	
	
	\textbf{Permutations}
	
	\begin{itemize}
		\item Let $\mathcal{U}$ be the set of all such codes.
		\item Let $\mathcal{V}$ be the set of all such codes beginning with a vowel.
		\item Let $\mathcal{P}$ be the set of all such codes which are palindromic.
	\end{itemize} 
	\bigskip
	(A palindromic code is a string of letters which read the same backwards as forwards, for example \textbf{\textit{a,e,c,e,a}} is a 5 letter palindromic code.)\\ \bigskip
	
	
	
	
	
	\textbf{Permutations}
	
	How many elements are there in the set $\mathcal{U}$?
	\begin{center}
		\begin{tabular}{|c|c|c|c|}
			\hline (i) &  (ii) &  (iii) &  (iv) \\ 
			\hline {\color{white}More space} &{\color{white}More space}  & {\color{white}More space} &{\color{white}More space}  \\ 
			{\color{white}More space} &{\color{white}More space}  & {\color{white}More space} &{\color{white}More space}  \\ 
			\hline 
		\end{tabular} 
	\end{center}
	
	
	
	
	How many elements are there in the set $\mathcal{V}$?
	\begin{center}
		\begin{tabular}{|c|c|c|c|}
			\hline (i) &  (ii) &  (iii) &  (iv) \\ 
			\hline {\color{white}More space} &{\color{white}More space}  & {\color{white}More space} &{\color{white}More space}  \\ 
			{\color{white}More space} &{\color{white}More space}  & {\color{white}More space} &{\color{white}More space}  \\ 
			\hline 
		\end{tabular} 
	\end{center}
	
	
	
	
	How many elements are there in the set $\mathcal{P}$?
	\begin{center}
		\begin{tabular}{|c|c|c|c|}
			\hline (i) &  (ii) &  (iii) &  (iv) \\ 
			\hline {\color{white}More space} &{\color{white}More space}  & {\color{white}More space} &{\color{white}More space}  \\ 
			{\color{white}More space} &{\color{white}More space}  & {\color{white}More space} &{\color{white}More space}  \\ 
			\hline 
		\end{tabular} 
	\end{center}
	
	
	
	
	How many elements are there in the sets $\mathcal{V}$ and $\mathcal{P}$?
	\begin{center}
		\begin{tabular}{|c|c|c|c|}
			\hline (i) &  (ii) &  (iii) &  (iv) \\ 
			\hline {\color{white}More space} &{\color{white}More space}  & {\color{white}More space} &{\color{white}More space}  \\ 
			{\color{white}More space} &{\color{white}More space}  & {\color{white}More space} &{\color{white}More space}  \\ 
			\hline 
		\end{tabular} 
	\end{center}
	
	
	
	%--------------------------------------------------------%
	%--------------------------------------------------------%
	\section{ Combinations and Permutations }
	
	
	

	%--------------------------------------------------------%
	{

		
		\bigskip
		Discuss how combinations can be used to compute the number of rugby matches for each group in the Rugby World Cup.
		

		\textbf{The Permutation Formula}
		The number of different permutations of r items from n unique items is written as $^n P_k$
		
		
		\[ ^n P_k = \frac{n!}{(n-k)!}\]
	}
	
	%--------------------------------------------------------%
	{
		\textbf{Permutations}
		\textbf{Example:}
		How many ways are there of arranging 3 different jobs, between 5 workers, where each worker can only do one job?
		
		
		\[ ^5 P_3 = \frac{5!}{(5-3)!}  = {5! \over 2!} = 60\]
		
	}
	
	

	
	
	%--------------------------------------------------------%
	[fragile]
	\textbf{Using \texttt{R}}
	When implementing combination calculations in \texttt{R}, we use the \texttt{choose()} function.
	
	\begin{verbatim}
	> choose(5,0)
	[1] 1
	> choose(5,1)
	[1] 5
	> choose(5,2)
	[1] 10
	> choose(5,3)
	[1] 10
	> choose(5,4)
	[1] 5
	> choose(5,5)
	[1] 1
	\end{verbatim}
	
	
	%--------------------------------------------------------%
	{
		\textbf{Example of Combinations}
		
		\textbf{Part 2}
		
		Thus a committee of 2 men and 2 women can be selected in $ 6 \times 3  = 18 $ ways.\\
		\bigskip
		\textbf{Part 3}
		
		The probability of two men and two women on a committee is
		\[ {\mbox{Number of ways of selecting 2 men and 2 women} \over \mbox{Number of ways of selecting 4 from 7}} = {18 \over 35 }\]
		
	}
	%--------------------------------------------------------%

	%--------------------------------------------------------%
	{
		\textbf{Example of Combinations}
		
		\textbf{Part 4}
		\begin{itemize}
			\item We have to compute the number of ways of selecting 1 male from 4 (4 ways) and the number of ways of selecting three females from 2 ( only 1 way)
			\item The probability of selecting three females is therefore ${4 \times 1 \over 35} = 4/35$
			\item So using the addition rule
			\[ Pr(\mbox{ at least 2 females }) = Pr(\mbox{ 2 females }) + Pr(\mbox{ 3 females }) \]
			\[ Pr(\mbox{ at least 2 females })  = 18/35 + 4/35 = 22/35 \]
		\end{itemize}
		
	}
	
	
	
	
	\section{ Combinations and Permutations }
	
	

	
	Discuss how combinations can be used to compute the number of rugby matches for each group in the Rugby World Cup.
	
	

	
	%--------------------------------------------------------%
	
	\textbf{Using \texttt{R}}
	When implementing combination calculations in \texttt{R}, we use the \texttt{choose()} function.
	
	\begin{verbatim}
	> choose(5,0)
	[1] 1
	> choose(5,1)
	[1] 5
	> choose(5,2)
	[1] 10
	> choose(5,3)
	[1] 10
	> choose(5,4)
	[1] 5
	> choose(5,5)
	[1] 1
	\end{verbatim}
	
	
	%--------------------------------------------------------%
	{
		\textbf{Example of Combinations}
		
		\textbf{Part 2}
		
		Thus a committee of 2 men and 2 women can be selected in $ 6 \times 3  = 18 $ ways.\\
		
		\textbf{Part 3}
		
		The probability of two men and two women on a committee is
		\[ {\mbox{Number of ways of selecting 2 men and 2 women} \over \mbox{Number of ways of selecting 4 from 7}} = {18 \over 35 }\]
		
		

		
		
		\textbf{Example of Combinations}
		
		\textbf{Part 4}
		\begin{itemize}
			\item We have to compute the number of ways of selecting 1 male from 4 (4 ways) and the number of ways of selecting three females from 2 ( only 1 way)
			\item The probability of selecting three females is therefore ${4 \times 1 \over 35} = 4/35$
			\item So using the addition rule
			\[ Pr(\mbox{ at least 2 females }) = Pr(\mbox{ 2 females }) + Pr(\mbox{ 3 females }) \]
			\[ Pr(\mbox{ at least 2 females })  = 18/35 + 4/35 = 22/35 \]
		\end{itemize}
		
	}
	
	
	
	%----------------------------------------------------------------------------------------------------%
	
	
	A college teaches a range of courses including maths, physics and IT.
	Students choose a range of courses from these three subject areas. Currently 600
	students are enrolled of whom 300 study maths courses, 120 study IT
	and 380 study physics courses. 
	
	\begin{itemize}
		\item 40 students study courses from all three subject
		areas. 
		\item 200 maths students study physics as well. 60 physics students
		also study IT and 70 IT students also study maths. 20 students study physics and IT, but not maths.
	\end{itemize}
	
	
	
	\begin{itemize}
		\item How many students study none of these courses at all? (90)
		
		\item How many students study maths but not physics or IT? (70)
		
		\item How many students study both maths and physics but not IT? (160)
		
		\item How many students study courses from precisely two of these subject
		areas? (210)
	\end{itemize}
	
	
	
	

	

	

	

	
	

	
	
	
	
	An ordered sequence of four digits is formed by choosing digits without
	repetition from the set {1, 2, 3, 4, 5, 6, 7} .
	
	(i) the total number of such sequences; (780)
	(ii) the number of sequences which begin with an odd number; (480) N(A)
	(iii) the number of sequences which end with an odd number; (480) (NB)
	(iv) the number of sequences which begin and end with an odd number;(240)
	(v) the number of sequneces which begin with an odd number or end with an
	odd number or both; (720)
	(vi) the number of sequences which begin with an odd number or end with an
	odd number but not both. (480)
	%----------------------------------------------------------------------------------------------------%
	A college teaches a range of courses including maths, physics and IT.
	Students choose a range of courses from these three subject areas. Currently 600
	students are enrolled of whom 300 study maths courses, 120 study IT
	and 380 study physics courses. 40 students study courses from all three subject
	areas. 200 maths students study physics as well. 60 physics students
	also study IT and 70 IT students also study maths. 20 students study physics and IT, but not maths.
	
	How many students study none of these courses at all? (90)
	
	How many students study maths but not physics or IT? (70)
	
	How many students study both maths and physics but not IT? (160)
	
	How many students study courses from precisely two of these subject
	areas? (210)
	\newpage
	%--------------------------------------------------------%
	\section{ Combinations and Permutations }
	
	%--------------------------------------------------------%
	{
		\textbf{Factorials Numbers}
		
		A factorial is a positive whoe numnber, based on a number $n$ , and which is written as $``n!"$. The factorial $n!$ is defined as follows:
		
		\[n!  =n \times (n-1) \times (n-2) \times \ldots \times 2 \times 1 \]
		
		Remark $n!  =n \times (n-1)!$\\ \bigskip
		
		\textbf{ Example: }
		
		\begin{itemize}
			\item $3!  = 3 \times 2  \times 1 = 6 $
			
			\item $4!  = 4 \times 3! = 4 \times 3 \times 2 \times 1 = 24$
		\end{itemize}
		Remark $0! = 1$ not $0$.
		
		

	

%===================================================%
		\section{Example of Combinations}
		
		A committee of 4 must be chosen from 3 engineers and 4 computer scientists. 
		\begin{itemize}
			\item In how many ways can the comittee be chosen
			\item in how many cans 2 males and 2 females be chosen
			\item compute the probability of a committee of 2males and 2 females ris
			\item compute the probability of at least two females.
		\end{itemize}
	
	
\textbf{Example of Combinations}
		
		\textbf{Part 1}
		
		We need to choose 4 people from 7:
		
		This can be done in 
		
		\[
		{7 \choose 4} = {7!  \over 4! \times 3!} =  {7 \times 6 \times 5 \times 4!  \over 4! \times 3!} = 35 \mbox{ ways.}
		\]
		
		
		\textbf{Part 2}
		
		With 4 men to choose from, 2 men can be selected in \[
		{4 \choose 2} = {4!  \over 2! \times 2!} =  {4 \times 3 \times 2!  \over 2! \times 2!} = 6\mbox{ ways.}
		\]
		
		Similarly  2 women can be selected from 3 in 
		\[
		{3 \choose 2} = {3!  \over 2! \times1!} =  {3 \times 2!  \over 2! \times 1!} = 3\mbox{ ways.}
		\]
		
	}
	%--------------------------------------------------------%
	{
		\textbf{Example of Combinations}
		
		\textbf{Part 2}
		
		Thus a comittee of 2 men and 2 women can be selected in $ 6 \times 3  = 18 $ ways.\\
		\bigskip
		\textbf{Part 3}
		
		The probability of two men and two women on a comittee is 
		\[ {\mbox{Number of ways of selecting 2 men and 2 women} \over \mbox{Number of ways of selecting 4 from 7}} = {6 \over 35 }\]
		
	}
	
	
	%--------------------------------------------------------%
	
	
\end{document}
