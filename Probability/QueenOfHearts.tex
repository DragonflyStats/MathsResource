For mutually exclusive events, that is events which cannot occur together:
$P(A\cap B) = 0$. The addition rule therefore reduces to
\[ P(A\cup B) = P(A) + P(B)\]

\textbf{Addition Rule: Worked Example}
Suppose we wish to find the probability of drawing either a Queen or a Heart
in a single draw from a pack of 52 playing cards. We define the events $Q$ =
`draw a queen' and $H$ = `draw a heart'.
\begin{itemize}
\item $P(Q)$ probability that a random selected card is a Queen
\item  $P(H)$ probability that a randomly selected card is a Heart.
\item  $P(Q\cap H)$ probability that a randomly selected card is the Queen of
Hearts.
\item  $P(Q\cup H)$ probability that a randomly selected card is a Queen or a Heart.
\end{itemize}

%=============================================================================%



\noindent \textbf{Solution}
\begin{itemize} 
\item Since there are 4 Queens in the pack and 13 Hearts, so the $P(Q)$ = $4/52$
and $P(H) = 13/52$ respectively.
\item The probability of selecting the Queen of Hearts is $P(Q\cap H) = 1/52$.
\item We use the addition rule to find $P(Q\cup H)$:
\[ P(Q\cup H) = (4/52) + (13/52) - (1/52) = 16/52 \]
\item So, the probability of drawing either a queen or a heart is
$16/52 (= 4/13)$.
\end{itemize}
A card is drawn at random from a standard pack of playing cards. It is an ace. What is the probability that it is the ace of diamonds?


\begin{framed}
Solution

\begin{itemize}
\item What is the probability the ace picked is the ace of diamonds (given that we know that it is an ace). 

\item 
Wording of this question is very important.

\item 
There are four card suits ( hearts, diamonds, clubs, spades)

\item 
The card has a 'one in four' chance of being an ace of diamonds.
\end{itemize}
\end{framed}
