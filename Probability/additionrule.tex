
\section{General rule of addition}
\begin{itemize}
\item For events that are not mutually exclusive, the probability of the joint occurrence of the two events is
subtracted from the sum of the simple probabilities of the two events. We can represent the probability of joint
occurrence by P(A and B).\item  In the language of set theory this is called the intersection of A and B and the
probability is designated by P(A and B).  Thus, the rule of addition for events that are not mutually exclusive is
\[ P(A \mbox{ or }B) = P(A)+ P(B) - P(A \mbox{ and }B)\]
\end{itemize}

\noindent \textbf{Example}

When drawing a card from a deck of playing cards, the events ``ace" and ``spade" are not mutually
exclusive. The probability of drawing an ace (A) or spade (S) (or both) in a single draw is
\begin{eqnarray} P(A \mbox{ or }B) &=& P(A) + P(S) - P(A \mbox{ and }B)\\ &=& 4/52 + 13/52 -1/52 \\&=& 16/52 \\
&=& \textbf{4/13} 
\end{eqnarray}

\subsection{Addition Rule}
The addition rule is a result used to determine the probability that event $A$ or
event $B$ occurs or both occur. The result is often written as follows, using set
notation:
\[ P(A\cup B) = P(A) + P(B)- P(A \cap B)\]
\begin{itemize}
\item $P(A)$ = probability that event $A$ occurs.
\item $P(B)$ = probability that event $B$ occurs.
\item $P(A\cup B)$ = probability that either event $A$ or event $B$ occurs, or both
occur.
\item $P(A\cap B)$ = probability that event $A$ and event $B$ both occur.
\end{itemize}\bigskip

\noindent \textbf{Remark:} $P(A\cap B)$ is subtracted to prevent the relevant outcomes being
counted twice.



\subsection{Addition Rule (Continued)}
For mutually exclusive events, that is events which cannot occur together:
$P(A\cap B) = 0$. The addition rule therefore reduces to
\[ P(A\cup B) = P(A) + P(B)\]
}




For mutually exclusive events, that is events which cannot occur together:
$P(A\cap B) = 0$. The addition rule therefore reduces to
\[ P(A\cup B) = P(A) + P(B)\]



\section{Probability: Addition Rule for Any Two Events}

\begin{itemize}
\item For any two events A and B, the probability of A or B is the sum of the probability of A and the probability of B minus the probability of both A and B:
\[P(A \cup B) = P(A) + P(B) - P(A \cap B)\]
\vspace{-0.2cm}
\item We subtract the probability of $A \cap B$ to prevent it getting counted twice.

\item \textit{($A \cup B$ and $A \cap B$ denotes  ``A or B" and ``A and B" respectively) }

\item If events A and B are \textbf{mutually exclusive}, then the probability of A or B is the sum of the probability of A and the probability of B:

\[P(A \cup B) = P(A) + P(B)\]

\item If A and B are mutually exclusive, then the probability of both A and B is zero.
\end{itemize}

