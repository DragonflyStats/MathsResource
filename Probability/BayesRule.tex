


%------------------------------------------------------------%
{

\section{Bayes' Theorem}
Bayes' Theorem is a result that allows new information to be used to update the conditional probability of an event.
\bigskip

Recall the definition of conditional probability:
\[ P(A|B) = \frac{P(A \cap B)}{P(B)} \]


Using the multiplication rule, gives Bayes' Theorem in its simplest form:

\[ P(A|B) = \frac{P(B|A)\times P(A)}{P(B)} \]



In its simplest algebraic form, Bayes’ theorem is concerned with determining the conditional probability of
event A given that event B has occurred. 

The general form of Bayes’ theorem is
\[ P(A|B) =
\frac{P(A \mbox{ and }B)}{P(B)} \]






The general form of Bayes’ theorem is
\[ P(A|B) =
\frac{P(A \mbox{ and }B)}{P(B)} \]

Equivalently Bayes' Theorem can be expressed as
\begin{equation*}
P(B|A)=\frac{P\left(A|B\right) \times P(B) }{P\left( A\right) }.
\end{equation*}



%------------------------------------------------------------%
{
	\textbf{Bayes' Theorem}
	Bayes' Theorem is a result that allows new information to be used to update the conditional probability of an event.
	\bigskip
	
	Recall the definition of conditional probability:
	\[ P(A|B) = \frac{P(A \cap B)}{P(B)} \]
	
	
	Using the multiplication rule, gives Bayes' Theorem in its simplest form:
	
	\[ P(A|B) = \frac{P(B|A)\times P(A)}{P(B)} \]
	
}




%------------------------------------------------------------%
{
\subsection{Bayes' Theorem}
Bayes' Theorem is a result that allows new information to be used to update the conditional probability of an event.
\bigskip

\[ P(A|B) = \frac{P(B|A)\times P(A)}{P(B)} \]

Use this theorem to compue $P(S|F)$, the probability that a student studies statistics, given that she is female.

\[ P(S|F) = \frac{P(F|S)\times P(S)}{P(F)} = {0.55 \times 0.40 \over 0.38} = 0.578\]
}

