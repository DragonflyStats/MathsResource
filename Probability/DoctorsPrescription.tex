	
\section{Doctor's Presciption Question}


A doctor treating a patient issues a prescription for antibiotics and provides for two repeat prescriptions. The probability that the infection will be cleared by the first prescription is p1 =0.6. 


The probability that successive treatments are successful, given that previous prescriptions were not successful are $p_2$ = 0.5, $p_3$ = 0.4. Calculate the probability that  

\begin{itemize}   
	\item[(a.)] the patient is still infected after the third prescription,
	\item[(b.)] the patient is cured by the second prescription,
	\item[(c.)] the patient does not require a third prescription,
	\item[(d.)] the patient is cured by the second prescription, given that the patient is eventually cured.
\end{itemize}





Solution


P(need 2nd) =P(Fc) = 1-P(F) =0.4


P(need 3rd) =P(Sc|Fc) =P(Sc)P(Fc) = 0.50.4 = 0.2


P(not cured) =P(Tc| need 2nd ) =P(Tc)P(Sc)P(Fc)= 0.60.50.4 = 0.12


P(2nd cured) = P(S)P(Fc) = 0.50.4 = 0.2




Alternative solution

[ lets use cohort of 1000 patients ]


\begin{itemize}
	
	\item probability that a person is cured after first prescription         P[F]                = 0.6  [600 patients]
	
	\item probability that a person is still infected after first prescription   P[Fc]             = 1-0.6 = 0.4  [400 patients]
	
	
	[400 patients will need second prescription]
	
	
	\item probability that a person is cured after second prescription     P[S]                    = 0.5  [200 patients]
	
	\item probability that a person is still infected after second prescription  P[Sc]              = 1-0.5 = 0.5  [200 patients]
	
	
	\item [200 patients will need third prescription. 800 patients now cured]
	
	
	\item probability that a person is cured after third prescription          P[T]               = 0.4  [80 patients]
	
	\item probability that a person is still infected after third prescription      P[Tc]          = 1-0.4 = 0.6  [120 patients]
	
	\item 
	[120 patients will need treatment. 880 patients now cured]
	
\end{itemize}

