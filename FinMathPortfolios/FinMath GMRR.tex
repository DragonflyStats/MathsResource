\documentclass{beamer}

\usepackage{amsmath}
\usepackage{graphicss}
\usepackage{amssymb}
\usepackage{framed}

\begin{document}
\begin{frame}
\Huge
\[ \mbox{Financial Mathematics}  \]
\huge
\[ \mbox{Geometric Mean Rate of Return}\]
\Large
\[ \mbox{www.Stats-Lab.com}   \]
\[ \mbox{Twitter: @StatsLabDublin} \]
\end{frame}
%------------------------------------------%
\begin{frame}
\frametitle{Geometric Mean Rate of Return}
\Large
\[ = \sqrt{n}{(1+r_1)\cdot(1+r_2)\ldots(1+r_n)} - 1
\]
\begin{itemize}
\item $r_i$ return for period $i$
\item $n$ is the number of periods
\end{itemize}
\end{frame}
%------------------------------------------%
%------------------------------------------%
\begin{frame}
\frametitle{Geometric Mean Rate of Return}
\Large
Suppose a firm earns the following return on an investment over a 5 year period.
\begin{tabular}{|c|c|c|}
\hline Year & Return (\%) & Return (2.d.p)\\ 
\hline 1 & 6\% & 0.06 \\ 
\hline 2 & 5\% & 0.05\\ 
\hline 3 & 2\% & 0.02\\ 
\hline 4 & 0\% & 0.00\\
\hline 5 & 4\% & 0.04\\
\hline 
\end{tabular} 
\end{frame}
\begin{frame}
\frametitle{Geometric Mean Rate of Return}
\Large
\[ = \sqrt{n}{(1+r_1)\cdot(1+r_2)\ldots(1+r_n)} - 1
\]

\[ = \sqrt{5}{(1.06)\cdot(1.05)\cdot(1.02)\cdot(1.00)\cdot(1.04)} - 1
\]
\end{frame}
%------------------------------------------%
\end{document}
