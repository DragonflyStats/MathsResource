%% Water Nitrate

Three investigators performed six determination of nitrate in water using the same procedure. The results in µM were:

Investigator 1 Investigator 2 Investigator 3

6.7 6.3 6.8

6.9 6.2 6.9

6.6 6.1 7.1

6.7 6.5 7.0

7.0 6.6 7.2

7.1 6.4 7.1

The analysis of variance is used to determine if there is a significant difference between the mean of the determinations made by the three investigators.

The following output is obtained in R and presented below.

investigator <- c(6.7,6.9,6.6,6.7,7,7.1, 6.3,6.2,6.1,6.5,6.6,6.4,6.8,6.9,7.1,7,7.2,7.1)

group <- factor(rep(1:3,each=6))

anova(lm(investigator~group))

Analysis of Variance Table

Response: investigator

Df SumSq MeanSq Fvalue Pr(>F)

group ? 1.42333 ? ? 3.133e-05

Residuals 15 ? ?

Total ? 1.9

---

(a) Fill in the missing values in the ANOVA table and explain how these statistics can be used to assess if there is any difference between the mean determinations made by the three investigators. Clearly state the null and alternative hypothesis.

(10 marks)

(b) What are the assumptions underlying ANOVA? Assess the validity of these assumptions interpreting the qqplot, the plot of the residuals and the Shapiro-Wilk and the Bartlett tests below. Clearly state the null and alternative hypothesis for the two tests.

(10 marks)

Shapiro-Wilk normality test

data: residuals

W = 0.9431, p-value = 0.3271

Bartlett test of homogeneity of variances

data: investigator by group

Bartlett's K-squared = 0.4142, df = 2, p-value = 0.813

%============================================================================%
% %Vegetables

(c) The following data give the recovery of bromide from spiked samples of vegetable matter, measured using a gas-liquid chromatographic method. The same amount of bromide was added to each specimen. Also given is a sample of MINITAB output calculated from these data.


Tomato: 777 790 759 790 770 758 764 mg g-1

Cucumber: 782 773 778 765 789 797 782 mg g-1



-------------------------------------------------------------------------------


Two-Sample T-Test and CI: Tomato, Cucumber

Two-sample T for Tomato vs Cucumber

N Mean StDev SE Mean

Tomato 7 772.6 13.6 5.1

Cucumber 7 780.9 10.4 3.9

Difference = mu Tomato - mu Cucumber

Estimate for difference: -8.29

95% CI for difference: (-22.37, 5.80)

T-Test of difference = 0 (vs not =):

T-Value = -1.28 P-Value = 0.224 DF = 12

Both use Pooled StDev = 12.1

-------------------------------------------------------------------------------

Do the recoveries from the two vegetables have variances that differ significantly? Justify your answer. Write down the null and alternative hypotheses being considered.

[4 marks]

Do the mean recovery rates differ significantly? Justify your answer. Write down the null and alternative hypotheses being considered.

[4 marks]



%===============================================================================================%



An engineer is designing a battery for use in a device that will be subjected to some extreme variations in temperature. 
The only design parameter that he can select at this point is the plate material for the battery, and he has three possible choices. 
When the device is manufactured and is shipped to the field, the engineer has no control over the temperature extremes that the device will encounter, and he knows from experience 
that temperature will probably affect the effective battery life. 

However, temperature can be controlled in the product development laboratory for the purposes of a test. 
The engineer decides to test all three place materials at three temperature levels – 15, 70, and 125ºF – because these temperature levels are consistent with the product 
end-use environment. Four batteries are tested at each combination of plate material and temperature, and all 36 tests are run in random order.

The following partial ANOVA table resulted:

Analysis of Variance for Battery Life Data
----------------------------------------------

Source of Sum of Degrees of Mean

Variation Squares Freedom Square

Material types 10,683.72 2 5,341.86

Temperature 39,118.72 2 19,559.36

Interaction 9,613.78 4 2,403.44

Error 18,230.75 27 675.21

Total 77,646.97 35
----------------------------------------------

i) Carry out appropriate tests stating clearly the null hypotheses and conclusions. 6

ii) Would the engineer be satisfied with his design of the experiment? Explain your answer. 4


%=================================================================%
Q7 These data are from a statement by Texaco, Inc. to the Air and Water Pollution Subcommittee of the Senate Public Works Committee on June 26, 1973. Mr. John McKinley, President of Texaco, cited the Octel filter, developed by Associated Octel Company as effective in reducing pollution. However, questions had been raised about the effects of pollution filters on aspects of vehicle performance, including noise levels. He referred to data presented associated with this question as evidence that the Octel filter was at least as good as a standard silencer in controlling vehicle noise levels. The data constitute a 3-way factorial experiment with 3 replications.

Small Medium Large

Filter Right 810.00 840.00 785.00

Type 820.00 840.00 790.00

“Standard” 820.00 845.00 785.00

Left 835.00 845.00 760.00

835.00 855.00 760.00

835.00 850.00 770.00

Filter Right 820.00 820.00 775.00

Type 820.00 820.00 775.00

“Octel” 820.00 825.00 775.00

Left 825.00 815.00 770.00

825.00 825.00 760.00

825.00 825.00 765.00

Main effects plots, interaction plots, and MINITAB output relevant to these data are given below.

(a) Explain what you understand by “the main effect of a factor”. Write down the model corresponding to the ANOVA table below. Explain why the main effect for the factor “Side” can not be dropped from the model despite its high p-value ?

[8 marks]

(b) In your answer sheet draw an interaction plot for the factors “Type” and “Size” when (i) considering only data collected on the “Right Side” of automobiles, (ii) considering only data collected on the “Left Side” of automobiles.

[4 marks]

(c) Using the interaction plots from part (b), interpret the statistically significant three-way interaction found in the ANOVA table.

[4 marks]

(d) Comment on the overall validity of the claim made by John McKinley.

[4 marks]

%=================================================================%


Q7 A supermarket buys a particular product from four suppliers, A, B, C, D, and regular tasting tests by expert panels 
are carried out as the product is sold in their food halls. 
Various characteristics are scored and an analysis of the totals of these scores is made. 
Four tasters a, b, c, d obtained these results at four sessions 1-4.

Taster a b c d

21 17 18 20

20 22 23 19

20 24 22 19

22 21 22 26



Analysis of Variance for Score, using Adjusted SS for Tests

Source DF Seq SS MS F P

Taster ? 0.5000 0.1667 0.20 0.893

Session ? 28.5000 9.5000 11.40 0.007

Supplier ? 44.0000 14.6667 17.60 0.002

Error ? ? ?

Total ? ?
\begin{itemize}

(a) In the context of the above example, distinguish between the treatment and the blocking variables involved. Give reasons.

[5 marks]

(b) The above data are an example of a particular experimental design. What is the general name given to this type of experimental design? Name one serious limitation of this type of experimental design.

[5 marks]

(c) Complete the ANOVA table substituting the symbols ? with their correct values.

[5 marks]

(d) Interpret the results.

[5 marks]

(e) What is the key property of the experimental design above which allows factor effects to be estimated independently of one another. Show how this property presents itself in the above design.

[5 marks]

\end{itemize}
%=================================================================%
