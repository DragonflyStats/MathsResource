\documentclass{beamer}
\usepackage{array}
\usepackage{booktabs}
\usepackage{amsmath}
\usepackage{amssymb}
\usepackage{graphics}
\setlength{\heavyrulewidth}{1.5pt}
\setlength{\abovetopsep}{4pt}


\begin{document}

\begin{frame}
\bigskip
{
\Huge
\[ \mbox{Net Present Value} \]

}
{
\LARGE
\[ \mbox{www.stats-lab.com} \]
\[ \mbox{Twitter: @StatsLabDublin} \]
}
\end{frame}
%---------------------------------------------%
\begin{frame}
\frametitle{Net Present Value}
Suppose that a manufacturing company, which is contemplating the purchase of a new piece of 
Machinery. The rate of interest in this instance is $8\%$. The purchase price is €5,000. 

If purchased, the machine will last for four years and in each year generate extra revenue 
equivalent to $\$1000$. The resale value of the machine at the end of its lifetime is zero.

\end{frame}
%---------------------------------------------%
\begin{frame}
\frametitle{Net Present Value}
 The 
NPV of this project is:
\end{frame}
%---------------------------------------------%
\begin{frame}
\frametitle{Net Present Value}
\[\mbox{NPV} = {1000\over (1.1)^4} + {1000\over (1.1)^3} + {1000 \over (1.1)^2} + {1000\over(1.1)^1} – 1000 = 865.14 \]
 
As the NPV of the project exceeds zero, it should be accepted.
\end{frame}
%---------------------------------------------%
\end{document}


