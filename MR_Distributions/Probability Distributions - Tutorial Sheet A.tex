
\documentclass[]{article}

\voffset=-1.5cm
\oddsidemargin=0.0cm
\textwidth = 480pt

\usepackage{framed}
\usepackage{subfiles}
\usepackage{graphics}
\usepackage{newlfont}
\usepackage{eurosym}
\usepackage{amsmath,amsthm,amsfonts}
\usepackage{amsmath}
\usepackage{enumerate}
\usepackage{color}
\usepackage{multicol}
\usepackage{amssymb}
\usepackage{multicol}
\usepackage[dvipsnames]{xcolor}
\usepackage{graphicx}
\begin{document}

\section*{Probability Distributions}

\section*{Question 1}
\noindent \textbf{Computing Z-Scores} 


\[\mbox{Important Formula: } z_0 =  \frac{x_o - \mu }{ \sigma}  \mbox{ where } X \sim N(\mu,\sigma^2) \]
\begin{enumerate}[(a)]
\item Suppose $X \sim N(100,225)$. Compute the z-scores for X=118,91,120 and 124.
\item Suppose $X \sim N(100,256)$. Compute the z-scores for X=128, 124, and 126.
\item Suppose $X \sim N(100,100)$. Determine the values for $X$ that would yield z-scores of 1.24 and -1.63 respectively.
\end{enumerate}


\section*{Question 2}
%- Poisson Distributions (9 Marks)
Telephone calls arrive at a switchboard at the rate of 12 per hour.  Telecentre operators typically take 3 minutes to deal with a customer query.
Calculate the probability of :

\begin{enumerate}[(a)]
\item 2 or more calls arriving in any 15 minute period.
\item No phone calls arriving in a 15 minute period,
\item Exactly one phone call arriving in any 15 minute period,
\item What is the expected value and standard deviation of the number
of phone calls arriving in a 30 minute period.
\end{enumerate}


\section*{Question 3}
Flaws occur in an LCD display at the rate of 0.5 per square mm. Calculate the probability
that:
\begin{enumerate}[(a)]
    \item exactly 2 flaws will occur in a square mm section,
\item exactly 3 flaws will occur in a 5 square mm section,
\item 1 flaw will occur in a 10 square mm section.
\end{enumerate}


\section*{Question 4}
Suppose $X$ is a binomial variable specified as $X \sim \mbox{Bin}(1000,0.001)$. Using an appropriate approximation method, compute the following probabilities
\begin{enumerate}[(a)]
    \item The probability that X is zero: $P(X=0)$
\item The probability that X is equal to one:  $P(X=1)$
\item The probability that X is equal to two:  $P(X=2)$
\end{enumerate}


\section*{Question 5}
The average lifespan of a PC monitor is 6 years. You may assume that the lifespan of monitors
follows an exponential probability distribution.
\begin{enumerate}[(a)]
    \item  What is the probability that the lifespan of the monitor will be at least 5 years?
\item What is the probability that the lifespan of the monitor will not exceed 4 years?
\item  What is the probability of the lifespan being between 5 years and 7 years?
\end{enumerate}


\section*{Question 6}
% Exponential Distribution (8 Marks)
A power supply unit for a computer component is assumed to follow an
exponential distribution with a mean life of 1,400 hours. What is the
probability that the component will:

\begin{enumerate}[(a)]
\item  fail in the first 700 hours?
\item  survive more than 1,750 hours?
\item  last between 1,050 hours and 1,750 hours?
\end{enumerate}

%=========================================%

\section*{Question 7}
%- Probability Distributions (7 Marks)
On average, six people per hour use an electronic teller machine during the
prime shopping hours in a department store. Therefore it is assumed that
the expected time until the next customer will arrive will be 10 minutes.
You may assume that the distributions of waiting times can be described
by the exponential probability distribution.
\begin{itemize}
\item[(a)]  What is the probability that at least 10 minutes will pass
between the arrival of two customers?
\item[(b)]  What is the probability that after a customer leaves, another
customer does not arrive for at least 20 minutes?
%\item[(c)]  What is the probability that a second customer arrives within 1 min after a first customer begins a banking transaction?
\end{itemize}


\end{document}
