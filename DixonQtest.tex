\documentclass[]{article}

\voffset =-1.5cm
\oddsidemargin=0.0cm
\textwidth=460pt

\usepackage{framed}
\usepackage{graphicx}
\usepackage{enumerate}
%opening
\title{}
\author{}

\begin{document}

\LARGE
\section*{Dixon Q-Test for Outliers}

Use the Dixon Q-Test to determine whether or not there is an outlier value present in this data set.
\\ \textit{Use a 5\% significance level}.


{
\Huge
\begin{framed}
\[19, 36,33,  25, 30, 28, 31,36, 29, 37\]
\end{framed}
}


%==================================================%

\newpage
\begin{enumerate}[(i)]
\item Arrange the data set into ascending order.
\begin{framed}
\[19, 25, 28, 29, 30, 31, 33, 36, 36, 37\]
\end{framed}
\item Here the potential outlier is the lowest value, i.e. 19

\newpage
\item We can formally state the null and alternative hypothesis as follows

\begin{framed}
\begin{description}
\item[H$_0$] There are no outliers present in the data.
\item[H$_1$] There is one outlier present \\ (i.e. the lowest value 19)
\end{description}
\end{framed}
\newpage


\item The test statistic for this procedure is as follows:

\[ Q_{TS} =  \frac{\mbox{Gap}}{\mbox{Range}} \]

\item The gap is the difference of the outlier from the next value. In this case , the next value is 25, so the gap is 
\[ \mbox{Gap} = 25 - 19 = 6\]
\item The range is simply the difference between the maximum and minimum value.
\[ \mbox{Range} = 37-19 =18\]

\item \[ Q_{TS} =  \frac{\mbox{Gap}}{\mbox{Range}} = \frac{6}{18} =0.333 \]

\newpage
\item Before we look at the critical value, we confirm the size of the data set is $n=10$.

\item  The critical value can be determined from the following table. \\ $\bullet$ The column to chose is the significance level (here 5\% or 0.05 ).
\\ $\bullet$ The row to use is $n$, the number of items in the data set.
\bigskip

{
\huge
\begin{tabular}{|c|c|c|c|}
\hline
$n$	&	$\alpha=0.10$	&	$\alpha=0.05$	&	$\alpha=0.01$	\\ \hline
3	&	0.941	&	0.970	&	0.994	\\ \hline
4	&	0.765	&	0.829	&	0.926	\\ \hline
5	&	0.642	&	0.710	&	0.821	\\ \hline
6	&	0.560	&	0.625	&	0.740	\\ \hline
7	&	0.507	&	0.568	&	0.680	\\ \hline
8	&	0.468	&	0.526	&	0.634	\\ \hline
9	&	0.437	&	0.493	&	0.598	\\ \hline
10	&	0.412	&	0.466	&	0.568	\\ \hline
11	&	0.392	&	0.444	&	0.542	\\ \hline
12	&	0.376	&	0.426	&	0.522	\\ \hline
13	&	0.361	&	0.410	&	0.503	\\ \hline
14	&	0.349	&	0.396	&	0.488	\\ \hline
15	&	0.338	&	0.384	&	0.475	\\ \hline

\end{tabular} 

}
%=======================================%
\newpage

\item \textbf{Rule of Thumb}

\begin{framed}
\begin{itemize}
\item If the Test Statistic is greater than the Critical value, reject the null hypothesis
\[ Q_{TS} > Q_{CV}\]

\item Otherwise we fail to reject the null hypothesis

\end{itemize}
\end{framed}
\end{enumerate}

\newpage
{
\huge
\[ \mbox{Expected Value}  = \frac{\mbox{Column Total} \times \mbox{Row Total}}{\mbox{Overall Total}} \]
}

\newpage

\begin{framed}
\begin{itemize}
\item If the Test Statistic is greater than the Critical value, reject the null hypothesis
\[ \chi^2_{TS} > \chi^2_{CV}\]

\item Otherwise we fail to reject the null hypothesis

\end{itemize}
\end{framed}

\end{document}
