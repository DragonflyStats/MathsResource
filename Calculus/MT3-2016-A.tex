\documentclass[11pt]{article} % use larger type; default would be 10pt

\usepackage[utf8]{inputenc} % set input encoding (not needed with XeLaTeX)

\usepackage{geometry} % to change the page dimensions
\geometry{a4paper} % or letterpaper (US) or a5paper or....

\usepackage{framed}
\usepackage{graphicx} % support the \includegraphics command and options
\usepackage{subfiles}
\usepackage{booktabs} % for much better looking tables
\usepackage{array} % for better arrays (eg matrices) in maths
\usepackage{paralist} % very flexible & customisable lists (eg. enumerate/itemize, etc.)
\usepackage{verbatim} % adds environment for commenting out blocks of text & for better verbatim
\usepackage{subfig}
\usepackage{fancyhdr} % This should be set AFTER setting up the page geometry
\pagestyle{fancy} % options: empty , plain , fancy
\renewcommand{\headrulewidth}{0pt} % customise the layout...
\lhead{MA4702}\chead{MID TERM EXAM 3}\rhead{13th April 2016}
\lfoot{}\cfoot{\thepage}\rfoot{}
\voffset=-1.5cm
\oddsidemargin=0.0cm
\textwidth = 470pt
\usepackage{sectsty}
\allsectionsfont{\sffamily\mdseries\upshape} % (See the fntguide.pdf for font help)
\usepackage[nottoc,notlof,notlot]{tocbibind} % Put the bibliography in the ToC
\usepackage[titles,subfigure]{tocloft} % Alter the style of the Table of Contents
\renewcommand{\cftsecfont}{\rmfamily\mdseries\upshape}
\renewcommand{\cftsecpagefont}{\rmfamily\mdseries\upshape} % No bold!
\begin{document}
	
	
	\Large
	\newpage
	
	\begin{framed}
		\begin{description}
			\item[NAME]  ..........................................................................................
			\bigskip
			\item[STUDENT ID] ............................................................................
			\bigskip
			\item[DEGREE] ......................................................................................
		\end{description}
	\end{framed}
	\noindent Are You approved for Extra Time in Exams (LENS students only)  .......
	\section*{Examination Procedures}
	
	\begin{itemize} 
		\item This exam will start at 17:05, and will last 45 minutes.
		
		\item Each question will be worth either 2 or 3 Marks. There are 15 Marks worth of questions.
		\item All questions must be attempted (LENS students please see below)
		
		\item Write all of your answers in the exam script. Write the script number on any other documents you submit.
		
		\item It is your responsibility to return the script to collection box. An audit of scripts will take place immediately after the exam. If your script is account for in that audit,  you are deemed to be absent, and will receive no marks.
		
		\item \textbf{LENS Student}
		Specifically approved LENS students have to answer any selection of questions that have an aggregate mark of 11 Marks.  
		\begin{itemize}
			\item They may skip any combination of questions that have a total mark of 4
			%			\item OR - They may skip a 1-Mark Question and a 2-Mark Question
			\item The mark will be rescaled accordingly to give a result out of 15 marks.
		\end{itemize}
		(Type: A)		
		
	\end{itemize}
\newpage
\begin{enumerate}[(i)]
	
	\item (2 Marks) State the derivative $\displaystyle \frac{dy}{dx}$ of the following function
	\[y = \cos(2x)\]
	\item (3 Marks) Compute the following indefinite integral.
	\[ \int 4x^2 + 3x + 2 \; dx \]
	
\newpage	
	\item (1 Mark) State the derivatives of 
	$y = \cosh (x)$ and $y = \sinh(x)$.
\bigskip
	\item (3 Marks) Compute the following indefinite integral.
			\[ \int \sin(3x)  + e^{2x} \; dx \]

\newpage
	\item (2 Marks) Compute the following indefinite integral.
{\Large	\[  \int  \; \ln(x+1) + \ln(x+2) \; dx\]
}
\bigskip

\newpage
	\item (4 Marks) Use the substitution technique to solve the following indefinite integral.
		\[  \int  \frac{4x}{(x^2+9)^{10} } dx\]
\end{enumerate}

	%======================================================== %
\end{document}


% X and Y axis Intercepts
% Differentiation
 - $cosh(x)$
 - $sinh(x)$
 - $e^{3x}$
 - $e^{-x}$
 - $cos(x)$



%====================================================================%



%====================================================================%
