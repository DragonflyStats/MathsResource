\documentclass{beamer}

\usepackage{amsmath}
\usepackage{graphics}
\usepackage{amssymb}

\begin{document}

\begin{frame}
\frametitle{Asymptotes}
\Large



%% - http://www.purplemath.com/modules/asymtote.htm
Vertical asymptotes are vertical lines which correspond to the zeroes of the denominator of a rational function. (They can also arise in other contexts, such as logarithms, but you'll almost certainly first encounter asymptotes in the context of rationals.)
\end{frame}
\begin{frame}
	\frametitle{Asymptotes}
	\Large
Let's consider the following equation:
\[
y = [x^2 + 2x - 3] / [x^2 - 5x - 6]
\]

This is a rational function. More to the point, this is a fraction. Can you have a zero in the denominator of a fraction? 

\end{frame}
\begin{frame}
	\frametitle{Asymptotes}
	\Large
No. So if I set the denominator of the above fraction equal to zero and solve, this will tell me the values that x cannot be:
\[x^2 – 5x – 6 = 0 \]
\[(x – 6)(x + 1) = 0 \]
\[x = 6 or –1 \]

So x cannot be 6 or –1, because then I'd be dividing by zero.
\end{frame}
\begin{frame}
	\frametitle{Asymptotes}
	\Large


Now look at the graph:



%graph of y = [x^2 + 2x - 3] / [x^2 - 5x - 6]
\end{frame}
\begin{frame}
	\frametitle{Asymptotes}
	\Large
\begin{itemize}
\item You can see how the graph avoided the vertical lines x = 6 and x = –1. 
\item This avoidance occurred because x cannot be –1 or 6. 
\item In other words, the fact that the function's domain is restricted is reflected in the function's graph. 
\item More usefully, you can use the domain to help you graph, because whichever values are not allowed in the domain will be vertical asymptotes on the graph.
\end{itemize}

\end{frame}
\begin{frame}
	\frametitle{Asymptotes}
	\Large

You can draw the vertical asymptote as a dashed line to remind you not to graph there, like this:


(It's alright that the graph appears to climb right up the sides of the asymptote on the left. This is common. As long as you don't draw the graph crossing the vertical asymptote, you'll be fine.)
\end{frame}
\begin{frame}
	\frametitle{Asymptotes}
	\Large


% % graph of y = [x^2 + 2x - 3] / [x^2 - 5x - 6] with asymptotes dashed in

Let's review this relationship between the domain and the vertical asymptotes.

Find the domain and vertical asymptotes(s), if any, of the following function:
\[y = [x + 2] / [x^2 + 2x - 8]\]


\end{frame}
\begin{frame}
	\frametitle{Asymptotes}
	\Large
The domain is the set of all x-values that I'm allowed to use. The only values that could be disallowed are those that give me a zero in the denominator. So I'll set the denominator equal to zero and solve.

\[x2 + 2x – 8 = 0 \]
\[(x + 4)(x – 2) = 0\] 
\[x = –4 or x = 2\]

\end{frame}
\begin{frame}
	\frametitle{Asymptotes}
	\Large
Since I can't have a zero in the denominator, then I can't have x = –4 or x = 2 in the domain. This tells me that the vertical asymptotes (which tell me where the graph can not go) will be at the values x = –4 or x = 2.

\begin{itemize}
\item domain:   all x not equal to -4 or 2 
\item vertical asymptotes:  x = –4, 2
\end{itemize}
Note that the domain and vertical asymptotes are "opposites". The vertical asymptotes are at –4 and 2, and the domain is everywhere but –4 and 2. This is always true.
\end{frame}
\begin{frame}
	\frametitle{Asymptotes}
	\Large
Find the domain and vertical asymptote(s), if any, of the following function:
\[y = [x^3 - 8] / [x^2 + 9]\]

To find the domain and vertical asymptotes, I'll set the denominator equal to zero and solve. The solutions will be the values that are not allowed in the domain, and will also be the vertical asymptotes.

\[x^2 + 9 = 0 \]
\[x^2 = –9\]

\end{frame}
\begin{frame}
	\frametitle{Asymptotes}
	\Large
Oops! That doesn't solve! So there are no zeroes in the denominator. Since there are no zeroes in the denominator, then there are no forbidden x-values, and the domain is "all x". Also, since there are no values forbidden to the domain, there are no vertical asymptotes.

domain:  all x 
vertical asymptotes:  none
\end{frame}
\begin{frame}
	\frametitle{Asymptotes}
	\Large
Note again how the domain and vertical asymptotes were "opposites" of each other.

Find the domain and vertical asymptote(s), if any, of the following function:
\[y = [x^3 - 8] / [x^2 + 5x + 6]\]

\end{frame}
\begin{frame}
	\frametitle{Asymptotes}
	\Large
I'll check the zeroes of the denominator:

x2 + 5x + 6 = 0 
(x + 3)(x + 2) = 0 
x = –3 or x = –2
\end{frame}
\begin{frame}
	\frametitle{Asymptotes}
	\Large
Since I can't divide by zero, then I have vertical asymptotes at x = –3  and x = –2, and the domain is all other x-values.

domain:   for x not equal to -3 or -2 
vertical asymptotes:  x = –3   and x = –2
\end{frame}
\begin{frame}
	\frametitle{Asymptotes}
	\Large
When graphing, remember that vertical asymptotes stand for x-values that are not allowed. Vertical asymptotes are sacred ground. Never, on pain of death, can you cross a vertical asymptote. Don't even try!
\end{frame}
\end{document}
