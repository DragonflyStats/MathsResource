
\documentclass{beamer}

\usepackage{amsmath}
\usepackage{amssymb}
\begin{document}
\begin{frame}
\tableofcontents
\end{frame}
\section{Example 1 - Partial Differentation}
%------------------------------------------------------%
\begin{frame}
\frametitle{Partial Differentiation}
\Large
\vspace{-1cm}
Compute the partial derivatives of the expression $w$ with respect to $x$, $y$ and $z$.
\LARGE
\[ w = x^3y -12y^2z^3 + 42 \sqrt{z} \]

\end{frame}
%------------------------------------------------------%
\begin{frame}
\frametitle{Partial Differentiation}
\Large
\vspace{-2cm}
Part 1. Differentiate with respect to $x$
\[ w = x^3y -12y^2z^3 + 42 \sqrt{z} \]

\LARGE
\[ \phantom{{\partial w \over \partial x } =3x^2.y} \]


\end{frame}
%------------------------------------------------------%
\begin{frame}
\frametitle{Partial Differentiation}
\Large
\vspace{-2cm}
Part 1. Differentiate with respect to $x$
\[ w = x^3y -12y^2z^3 + 42 \sqrt{z} \]

\LARGE
\[ {\partial w \over \partial x } =3x^2.y \]


\end{frame}

%------------------------------------------------------%
\begin{frame}
\frametitle{Partial Differentiation}
\Large
\vspace{-2cm}
Part 2. Differentiate with respect to $y$
\[ w = x^3y -12y^2z^3 + 42 \sqrt{z} \]

\LARGE
\[ \phantom{{\partial w \over \partial y } =3x^3 + 24yz^3} \]
\end{frame}
%------------------------------------------------------%
\begin{frame}
\frametitle{Partial Differentiation}
\Large
\vspace{-2cm}
Part 2. Differentiate with respect to $y$
\[ w = x^3y -12y^2z^3 + 42 \sqrt{z} \]

\LARGE
\[ {\partial w \over \partial y } =3x^3 + 24yz^3 \]
\end{frame}

%------------------------------------------------------%
\begin{frame}
\frametitle{Partial Differentiation}
\Large
\[ w = x^3y -12y^2z^3 + 42 \sqrt{z}\]

\[ {\partial w \over \partial z } =3x^3 + 24yz^3\]
\end{frame}
%------------------------------------------------------%

%----------------------------------------------------------------------------------%
\section{Example 2 - Partial Differentiation Using the Quotient Rule}
%--------------------------------------------------------------%
\begin{frame}
\frametitle{Partial Derivatives}
\Large
\vspace{-2cm}

\[f(x,y)  = \frac{x^2}{x+y} + x\;sin\left(\frac{x}{y}\right) \]

Verify that 

\[ f(x,y) = { \partial f \over \partial x } + y { \partial f \over \partial y } \]

\end{frame}
%--------------------------------------------------------------%
\begin{frame}
\frametitle{Partial Derivatives}
\Large
\vspace{-1cm}

\begin{itemize}
\item We will split $f(x,y)$ into two expressions such that $f(x,y)$ = $f_1(x,y) + f_2(x,y)$
\item The first part is 
\[f_1(x,y)  = \frac{x^2}{x+y} \]
\item The second component is 
\[f_2(x,y)  = x\;sin\left(\frac{x}{y}\right) \]
\end{itemize}
\end{frame}
%--------------------------------------------------------------%
\begin{frame}
\frametitle{Partial Derivatives}
\Large
\vspace{-2cm}

\begin{itemize}
\item Differentiate the expression $f_1(x,y)$ with respect to both $x$ and $y$.
\[f_1(x,y)  = \frac{x^2}{x+y} \]
\item To differentiate with respect to $x$, we use the \textbf{\textit{Chain Rule}}.
\end{itemize}
\end{frame}
\section{Example 3 - Partial Differentiation}
%--------------------------------------------------------------%
\begin{frame}
\frametitle{Partial Derivatives}
\Large
\vspace{-2cm}

\end{frame}
%--------------------------------------------------------------%
\end{document}