\documentclass[11pt]{article} % use larger type; default would be 10pt

\usepackage[utf8]{inputenc} % set input encoding (not needed with XeLaTeX)

\usepackage{geometry} % to change the page dimensions
\geometry{a4paper} % or letterpaper (US) or a5paper or....

\usepackage{framed}
\usepackage{graphicx} % support the \includegraphics command and options
\usepackage{subfiles}
\usepackage{booktabs} % for much better looking tables
\usepackage{array} % for better arrays (eg matrices) in maths
\usepackage{paralist} % very flexible & customisable lists (eg. enumerate/itemize, etc.)
\usepackage{verbatim} % adds environment for commenting out blocks of text & for better verbatim
\usepackage{subfig}
\usepackage{fancyhdr} % This should be set AFTER setting up the page geometry
\pagestyle{fancy} % options: empty , plain , fancy
\renewcommand{\headrulewidth}{0pt} % customise the layout...
\lhead{MA4702}\chead{MID TERM EXAM 1}\rhead{24th February 2016}
\lfoot{}\cfoot{\thepage}\rfoot{}
\voffset=-1.5cm
\oddsidemargin=0.0cm
\textwidth = 470pt
\usepackage{sectsty}
\allsectionsfont{\sffamily\mdseries\upshape} % (See the fntguide.pdf for font help)
\usepackage[nottoc,notlof,notlot]{tocbibind} % Put the bibliography in the ToC
\usepackage[titles,subfigure]{tocloft} % Alter the style of the Table of Contents
\renewcommand{\cftsecfont}{\rmfamily\mdseries\upshape}
\renewcommand{\cftsecpagefont}{\rmfamily\mdseries\upshape} % No bold!
\begin{document}
	
	% % Part A (4 Marks)
	% 1 Floor and Ceiling Functions
	% 2 Evaluate Function for two values
	% 2 Cross Multiplication
	
	% % PArt B Limits
	% Simple one
	% Divide by Zero Problem
	% Divide by Infinity Problem
	
	% Part C sequences and Series
	% 2 Marks Sum of integers
	% 3 Geometric Sequence
	
	\Large
\newpage

\begin{framed}
\begin{description}
\item[NAME]  ..........................................................................................
\bigskip
\item[STUDENT ID] ............................................................................
\bigskip
\item[DEGREE] .....................................................................................
\end{description}
\end{framed}
\noindent Are You Approved for Extra Time in Exams (LENS students only)  .......
\section*{Examination Procedures}
\begin{itemize} 
	\item This exam will start at 17:05, and will last 45 minutes.
		\item Each question will be worth 1 Mark each. There are 10 Marks worth of questions.
	\item All questions must be attempted (LENS students please see below)
		\item Write \textbf{all of your answers} in the exam script. Write the script number on any other documents you submit.
		\item It is your responsibility to return the script to the collection box. 
		An audit of scripts will take place immediately after the exam. If your script is not accounted for in that audit,  you are deemed to be absent, and will receive no marks.
	
	\item \textbf{LENS Students: }
	Specifically approved LENS students have to answer any selection of questions that have an aggregate total of 7 Marks.  

	
	
\end{itemize}

%============================================================================ %

\newpage


\LARGE
\section*{Formula Sheet}

\subsection*{Difference of Two Cubes}
\[ a^3 - b^3 = (a-b)(a^2 + ab + b^2)\]
\subsection*{Sequences and Series}
\[ \sum_{i=1}^{n} i = \frac{n(n+1)}{2}\]

\[ S_n = \frac{n}{2} \left(2a + (n-1) d \right)\]

\[ S_n = a\left(\frac{1-r^n}{1-r}\right)\]

\[ S_\infty = \frac{a}{1-r}\]
	\subsection*{Logarithms}
	If $a^b = c$ then $\mbox{log}_a c = b$.
	\subsection*{Sum and Difference of Two Cubes}
	\[ a^3 + b^3 = (a-b)(a^2 - ab + b^2)\]
	\[ a^3 - b^3 = (a-b)(a^2 + ab + b^2)\]
	
	%======================================== %
	
	\subsection*{Sequences and Series}
	
	\[ \sum_{i=1}^{n} i = \frac{n(n+1)}{2}\]
	
	Arithmetic Series Summation:
	\[ S_n = \frac{n}{2} \left(2a + (n-1) d \right)\]
	
	Geometric Series Summation:
	\[ S_n = a\left(\frac{1-r^n}{1-r}\right)\]
	
	\[ S_\infty = \frac{a}{1-r}\]

	\subsection*{Ratio Test}
	
	For a series with general term $u_n$, if
	
	\[ \lim_{n \to \infty } \left| \frac{u_{n+1}}{u_n} \right| = r\]
	then the series converges (absolutely) if $r<1$


	%==========================================================================================%
	
	



%================================================================================== %
\subsection*{Part A Fundamentals of Mathematics (2 Marks) } %1AREADY

\begin{itemize}
	%========================================================%
	
	\item[(i)] (2 Marks) Determine the values of A and B from the following expression
	\[  \frac{7}{x^2-x-12} = \frac{A}{x+3} + \frac{B}{x-4}\]
	% %	\item[(ii)] (2 Marks) Determine whether or not the function \[f(x) = x cos(x)\] is odd, even or neither.
	\vspace{1.5cm}
	
	\item[(ii)] (2 Marks) Evaluate the function for the values of  $ x = \{0.25, 0.5 , 0.75 \}$
	
	\[  f(x) = \sqrt{1+x^2} \]
	

	\vspace{1.8cm}
	\newpage
	
	\item [(iii)](1 Mark) Find the value of $x$
	
	\[log_3(x + 1) + log_3(5) = 5\]
	
		
		\vspace{1.8cm}
		
	\item[(iv)] (1 Mark) Find the value of $x$
		\[e^{x+5} = 3. \]
	
\end{itemize}

%======================================================== %
\newpage
\subsection*{Part B Limits of Functions (4 Marks)}


%------------------------------------%
%Limits

\item[(i)](1 Mark)

\item[(ii)](1 Mark)

\item[(iii)](1 Mark)

\item[(iv)](1 Mark)


%------------------------------------%

% Early Questions 

Evaluate the following limits

\[  \lim_{x\to 3} \frac{x+3}{x^2-9}       \]
\[  \lim_{x\to -1} \frac{x+1}{x^2+x}      \]
\[  \lim_{x\to 1} \frac{1}{x^2+1}          \]
\[  \lim_{x\to 1} x^ + 5x - \frac{1}{2-x}   \]
\[  \lim_{x\to 1} \frac{x^2-1}{x^2+2x-3}    \]


\begin{enumerate}
	\item[(i)]  Compute the limit of the following function
	
	\[\lim_{x \to 4 } \frac{x^2-15}{x-4}\]
	\vspace{1.5cm}
	\item[(ii)]  Compute the limit of the following function
	\vspace{1.5cm}
	\[\lim_{x \to 4 } \frac{x^2-x-12}{x-4}\]
	\vspace{1.5cm}
	\item[(iii)]  Compute the limit of the following function
	\[ \lim_{x \to \infty } \frac{3 + 2x^2 - 8x^3 }{4x^3 - 7x + 5} \]	
	
\end{enumerate}
\newpage
%===================================================%
\subsection*{Part C Sequences and Series (4 Marks)}

\begin{enumerate}
	\item[(i)](1 Mark)  Compute the following summation
	
	\[ \sum_{i=1}^{80} i \]

	\vspace{5.5cm}
	\item[(ii)] (2 Marks) Find the sum of the following geometric series: 
		\[5 + 15 + 45 +  \ldots + 3645\]
		\newpage

	
	\item[(iii)] (1 Mark) Express the following repeating decimal number as a simple fraction. S
	]how your workings.
	
	\[0.243243243243243....\]
	\vspace{5.5cm}
	\item[(iv)] (2 Marks) Find the sum of the following telescoping series
		\[  \sum^{\infty}_{n=1}   \frac{6}{(3n+1)(3n+4)}  \]
		
\end{enumerate}



\begin{document}
\begin{itemize}



% Fundamentals

\item[(v)](1 Mark)

\[ \log_e (2x) = 5\]

\item[(vi)](1 Mark)

\[ \cosh(2x) \]
%------------------------------------%
% Sequences and Series

\item[(vii)](1 Mark)

\[u_n = (2n)! \]

\item[(viii)](1 Mark)

\item[(ix)](1 Mark)
% Geometric Series

\item[(x)](1 Mark)


%------------------------------------%
\end{itemize}

Add Change of Base Formula

\[ Log_A(B) = \frac{ \log(B) }{ \log(A) }  \]


%=============================================%


% End of Semester Question






Limits of Piecewise Functions[edit]
Evaluate the following limits or state that the limit does not exist.

37. Consider the function

 f(x) = \begin{cases} (x-2)^2 & \mbox{if }x<2 \\ x-3 & \mbox{if }x\geq 2. \end{cases} 
a.  \lim_{x\to 2^-}f(x) 
0
b.  \lim_{x\to 2^+}f(x) 
-1




% Questions 30 and 31

 \lim_{x\to \infty} \frac{2x^2-32}{x^3-64} 

Answer: 0
\end{document}
\end{document}
