\documentclass{beamer}

\usepackage{graphicx}
\usepackage{amsmath}
\usepackage{amssymb}
\begin{document}
%================================================================================ %
\begin{frame} 
	\frametitle{The Shape of a Graph}
% Let’s take at another example.

Example 3  Find and classify all the critical points of the following function.  Give the intervals where the function is increasing and decreasing.
\end{frame}
%================================================================================ %
\begin{frame} 
	\frametitle{The Shape of a Graph}
Solution\\
First we’ll need the derivative so we can get our hands on the critical points.  Note as well that we’ll do some simplification on the derivative to help us find the critical points.

\end{frame}
%================================================================================ %
\begin{frame} 
	\frametitle{The Shape of a Graph}
So, it looks like we’ll have four critical points here.  They are,


Finding the intervals of increasing and decreasing will also give the classification of the critical points so let’s get those first.  Here is a number line with the critical points graphed and test points.
\end{frame}
%================================================================================ %
\begin{frame} 
	\frametitle{The Shape of a Graph}
% % ShapeOfGraphI_Ex3_G1

So, it looks like we’ve got the following intervals of increasing and decreasing.


From this it looks like  and  are neither relative minimum or relative maximums since the function is increasing on both side of them.  On the other hand,  is a relative maximum and  is a relative minimum.
\end{frame}
%================================================================================ %
\begin{frame} 
	\frametitle{The Shape of a Graph}
For completeness sake here is the graph of the function.
% % ShapeOfGraphI_Ex3_G2
\end{frame}
%================================================================================ %
\begin{frame} 
	\frametitle{The Shape of a Graph}
In the previous example the two critical points where the derivative didn’t exist ended up not being relative extrema.  Do not read anything into this.  They often will be relative extrema.  Check out this example in the Absolute Extrema to see an example of one such critical point.
\end{frame}
%================================================================================ %
\begin{frame} 
	\frametitle{The Shape of a Graph}
Let’s work a couple more examples.

Example 4  Suppose that the elevation above sea level of a road is given by the following function.
\[ E(x) = 500 +cos(\frac{x}{4}) + \sqrt{3} sin(\frac{x}{4})\]
where x is in miles. 

\begin{itemize}
	\item  Assume that if x is positive we are to the east of the initial point of measurement and if x is negative we are to the west of the initial point of measurement. 
	
 \item	If we start 25 miles to the west of the initial point of measurement and drive until we are 25 miles east of the initial point how many miles of our drive were we driving up an incline?
\end{itemize}
\end{frame}
%================================================================================ %
\begin{frame} 
	\frametitle{The Shape of a Graph}
Solution
\begin{itemize}
\item Okay, this is just a really fancy way of asking what the intervals of increasing and decreasing are for the function on the interval [-25,25]. 
\item So, we first need the derivative of the function.
\[ E^{\prime}(x) =  \frac{-1}{4}sin(\frac{x}{4}) + \frac{\sqrt{3}}{4} cos(\frac{x}{4})\]
\end{itemize}


\end{frame}
%================================================================================ %
\begin{frame} 
	\frametitle{The Shape of a Graph}
Setting this equal to zero gives,
\[  \frac{-1}{4}sin(\frac{x}{4}) + \frac{\sqrt{3}}{4} cos(\frac{x}{4}) = 0 \]

\[ tan(\frac{x}{4}) =  \sqrt{3} \]
The solutions to this and hence the critical points are,


I’ll leave it to you to check that the critical points that fall in the interval that we’re after are,

\end{frame}
%================================================================================ %
\begin{frame} 
	\frametitle{The Shape of a Graph}
Here is the number line with the critical points and test points.

% % ShapeOfGraphI_Ex4_G1

So, it looks like the intervals of increasing and decreasing are,

\end{frame}
%================================================================================ %
\begin{frame} 
	\frametitle{The Shape of a Graph}

Notice that we had to end our intervals at -25 and 25 since we’ve done no work outside of these points and so we can’t really say anything about the function outside of the interval [-25,25]. 

\end{frame}
%================================================================================ %
\begin{frame} 
	\frametitle{The Shape of a Graph}
	
From the intervals we can actually answer the question.  We were driving on an incline during the intervals of increasing and so the total number of miles is,

\end{frame}
%================================================================================ %
\begin{frame} 
	\frametitle{The Shape of a Graph}
	
Even though the problem didn’t ask for it we can also classify the critical points that are in the interval [-25,25].

\end{frame}
%================================================================================ %
\begin{frame} 
	\frametitle{The Shape of a Graph}
\textbf{Example 5}
\begin{itemize} 
\item The population of rabbits (in hundreds) after t years in a certain area is given by the following function,

\item Determine if the population ever decreases in the first two years.
\end{itemize}
\end{frame}
%================================================================================ %
\begin{frame} 
	\frametitle{The Shape of a Graph}
	
Solution
\begin{itemize}
\item So, again we are really after the intervals and increasing and decreasing in the interval [0,2].

\item We found the only critical point to this function back in the Critical Points section to be,

\end{itemize}

\end{frame}
%================================================================================ %
\begin{frame} 
	\frametitle{The Shape of a Graph}
	

Here is a number line for the intervals of increasing and decreasing.
% % ShapeOfGraphI_Ex5_G1

So, it looks like the population will decrease for a short period and then continue to increase forever.

Also, while the problem didn’t ask for it we can see that  the single critical point is a relative minimum.

\end{frame}
%================================================================================ %
\begin{frame} 
	\frametitle{The Shape of a Graph}
	
In this section we’ve seen how we can use the first derivative of a function to give us some information about the shape of a graph and how we can use this information in some applications.

Using the first derivative to give us information about a whether a function is increasing or decreasing is a very important application of derivatives and arises on a fairly regular basis in many areas.
\end{frame}
%================================================================================ %
\end{document}