\documentclass{beamer}
% % - http://tutorial.math.lamar.edu/Classes/CalcI/SubstitutionRuleIndefinite.aspx
\usepackage{amsmath}
\usepackage{amssymb}

\begin{document}
%=============================================================================================================================== %
\begin{frame}
	\frametitle{Substitution Rule for Integration}
	\large
This next set of examples, while not particular difficult, can cause trouble if we aren’t paying attention to what we’re doing.
 
Example 3  Evaluate each of the following integrals.
(a)    [Solution]
(b)    [Solution]
(c)    [Solution]
(d)    [Solution]
\end{frame}
%=============================================================================================================================== %
\begin{frame}
	\frametitle{Substitution Rule for Integration}
	\large
Solution
(a) 
We haven’t seen a problem quite like this one yet.  Let’s notice that if we take the denominator and differentiate it we get just a constant and the only thing that we have in the numerator is also a constant.  This is a pretty good indication that we can use the denominator for our substitution so,
\end{frame}
%=============================================================================================================================== %
\begin{frame}
	\frametitle{Substitution Rule for Integration}
	\large                         
The integral is now,
                                                 
 
Remember that we can just factor the 3 in the numerator out of the integral and that makes the integral a little clearer in this case.
 [Return to Problems]
\end{frame}
%=============================================================================================================================== %
\begin{frame}
	\frametitle{Substitution Rule for Integration}
	\large
(b) 
The integral is very similar to the previous one with a couple of minor differences but notice that again if we differentiate the denominator we get something that is different from the numerator by only a multiplicative constant.  Therefore we’ll again take the denominator as our substitution.
                        
 \end{frame}
 %=============================================================================================================================== %
 \begin{frame}
 	\frametitle{Substitution Rule for Integration}
 	\large
The integral is,
                                               
[Return to Problems]
 
(c) 
Now, this one is almost identical to the previous part except we added a power onto the denominator.  Notice however that if we ignore the power and differentiate what’s left we get the same thing as the previous example so we’ll use the same substitution here.
                       
\end{frame}
%=============================================================================================================================== %
\begin{frame}
	\frametitle{Substitution Rule for Integration}
	\large
 
The integral in this case is,
                           
 
Be careful in this case to not turn this into a logarithm.  After working problems like the first two in this set a common error is to turn every rational expression into a logarithm.  If there is a power on the whole denominator then there is a good chance that it isn’t a logarithm.
\end{frame}
%=============================================================================================================================== %
\begin{frame}
	\frametitle{Substitution Rule for Integration}
	\large
The idea that we used in the last three parts to determine the substitution is not a bad idea to remember.  If we’ve got a rational expression try differentiating the denominator (ignoring any powers that are on the whole denominator) and if the result is the numerator or only differs from the numerator by a multiplicative constant then we can usually use that as our substitution.
\end{frame}
%=============================================================================================================================== %
\begin{frame}
	\frametitle{Substitution Rule for Integration}
	\large
[Return to Problems]
 
(d) 
Now, this part is completely different from the first three and yet seems similar to them as well.  In this case if we differentiate the denominator we get a y that is not in the numerator and so we can’t use the denominator as our substitution. 
\end{frame}
%=============================================================================================================================== %
\begin{frame}
	\frametitle{Substitution Rule for Integration}
	\large
In fact, because we have y2 in the denominator and no y in the numerator is an indication of how to work this problem. This integral is going to be an inverse tangent when we are done.  The key to seeing this is to recall the following formula,
                                                      
 \end{frame}
 %=============================================================================================================================== %
 \begin{frame}
 	\frametitle{Substitution Rule for Integration}
 	\large
We clearly don’t have exactly this but we do have something that is similar.  The denominator has a squared term plus a constant and the numerator is just a constant.  So, with a little work and the proper substitution we should be able to get our integral into a form that will allow us to use this formula. 
\end{frame}
%=============================================================================================================================== %
\begin{frame}
	\frametitle{Substitution Rule for Integration}
	\large
There is one part of this formula that is really important and that is the “1+” in the denominator.  That must be there and we’ve got a “4+” but it is easy enough to take care of that.  We’ll just factor a 4 out of the denominator and at the same time we’ll factor the 3 in the numerator out of the integral as well.  Doing this gives,
                                             
\end{frame}
%=============================================================================================================================== %
\begin{frame}
	\frametitle{Substitution Rule for Integration}
	\large 
Notice that in the last step we rewrote things a little in the denominator.  This will help us to see what the substitution needs to be.  In order to get this integral into the formula above we need to end up with a u2 in the denominator.  Our substitution will then need to be something that upon squaring gives us .  With the rewrite we can see what that we’ll need to use the following substitution.
\end{frame}
%=============================================================================================================================== %
\begin{frame}
	\frametitle{Substitution Rule for Integration}
	\large

 
Don’t get excited about the root in the substitution, these will show up on occasion.  Upon plugging our substitution in we get,
                                             
 
After doing the substitution, and factoring any constants out, we get exactly the integral that gives an inverse tangent and so we know that we did the correct substitution for this integral.  The integral is then\end{frame}
%=============================================================================================================================== %
\begin{frame}
	\frametitle{Substitution Rule for Integration}
	\large                                          
[Return to Problems]
 
In this last set of integrals we had four integrals that were similar to each other in many ways and yet all either yielded different answer using the same substitution or used a completely different substitution than one that was similar to it. 
\end{frame}
%=============================================================================================================================== %
\begin{frame}
	\frametitle{Substitution Rule for Integration}
	\large
This is a fairly common occurrence and so you will need to be able to deal with these kinds of issues.  There are many integrals that on the surface look very similar and yet will use a completely different substitution or will yield a completely different answer when using the same substitution.
\end{frame}
%=============================================================================================================================== %
\begin{frame}
	\frametitle{Substitution Rule for Integration}
	\large
Let’s take a look at another set of examples to give us more practice in recognizing these kinds of issues.  Note however that we won’t be putting as much detail into these as we did with the previous examples.
\end{frame}
%=============================================================================================================================== %

\begin{frame}
	\frametitle{Substitution Rule for Integration}
	\large
Example 4  Evaluate each of the following integrals.
(a)    [Solution]
(b)    [Solution]
(c)    [Solution]
(d)    [Solution]
Solution
(a) 
Clearly the derivative of the denominator, ignoring the exponent, differs from the numerator only by a multiplicative constant and so the substitution is,
         
 
After a little manipulation of the differential we get the following integral.
                                             
 
 
[Return to Problems]
(b) 
The only difference between this problem and the previous one is the denominator.  In the previous problem the whole denominator is cubed and in this problem the denominator has no power on it.  The same substitution will work in this problem but because we no longer have the power the problem will be different.
 
So, using the substitution from the previous example the integral is,

So, in this case we get a logarithm from the integral.
[Return to Problems]
 \end{frame}
 %=============================================================================================================================== %
 \begin{frame}
 	\frametitle{Substitution Rule for Integration}
 	\large
(c) 
Here, if we ignore the root we can again see that the derivative of the stuff under the radical differs from the numerator by only a multiplicative constant and so we’ll use that as the substitution.
              
 
The integral is then,
                                               
[Return to Problems]
  \end{frame}
  %=============================================================================================================================== %
  \begin{frame}
  	\frametitle{Substitution Rule for Integration}
  	\large
(d) 
In this case we are missing the x in the numerator and so the substitution from the last part will do us no good here.  This integral is another inverse trig function integral that is similar to the last part of the previous set of problems.  In this case we need to following formula.
                                                    
 \end{frame}
 %=============================================================================================================================== %
 \begin{frame}
 	\frametitle{Substitution Rule for Integration}
 	\large 
The integral in this problem is nearly this.  The only difference is the presence of the coefficient of 4 on the x2.  With the correct substitution this can be dealt with however.  To see what this substitution should be let’s rewrite the integral a little.  We need to figure out what we squared to get 4x2 and that will be our substitution.
 
                                               
 \end{frame}
 %=============================================================================================================================== %
 \begin{frame}
 	\frametitle{Substitution Rule for Integration}
 	\large 
With this rewrite it looks like we can use the following substitution.
                                
 
The integral is then,
                                              
\end{frame}
\end{document}
                                             