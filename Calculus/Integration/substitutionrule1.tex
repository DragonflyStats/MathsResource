\documentclass{beamer}
% % - http://tutorial.math.lamar.edu/Classes/CalcI/SubstitutionRuleIndefinite.aspx
\usepackage{amsmath}
\usepackage{amssymb}

\begin{document}
	\begin{frame}
		Substitution Rule for Indefinite Integrals
After the last section we now know how to do the following integrals.
\end{frame}
%=============================================================================================================================== %
\begin{frame}
	\frametitle{Substitution Rule for Integration}
	\large
	
 
 
However, we can’t do the following integrals.

 
 
All of these look considerably more difficult than the first set.  However, they aren’t too bad once you see how to do them.  Let’s start with the first one.


\end{frame}
%=============================================================================================================================== %
\begin{frame}
	\frametitle{Substitution Rule for Integration}
	\large
	
 
In this case let’s notice that if we let

 
and we compute the differential (you remember how to compute these right?) for this we get,
 

 
\end{frame}
%=============================================================================================================================== %
\begin{frame}
	\frametitle{Substitution Rule for Integration}
	\large
	
Now, let’s go back to our integral and notice that we can eliminate every x that exists in the integral and write the integral completely in terms of u using both the definition of u and its differential.

 
 
In the process of doing this we’ve taken an integral that looked very difficult and with a quick substitution we were able to rewrite the integral into a very simple integral that we can do.
 
\end{frame}
%=============================================================================================================================== %
\begin{frame}
	\frametitle{Substitution Rule for Integration}
	\large
	
Evaluating the integral gives,

 
 
As always we can check our answer with a quick derivative if we’d like to and don’t forget to “back substitute” and get the integral back into terms of the original variable.
 
What we’ve done in the work above is called the Substitution Rule.  Here is the substitution rule in general.
 
\end{frame}
%=============================================================================================================================== %
\begin{frame}
	\frametitle{Substitution Rule for Integration}
	\large
	
Substitution Rule
                             
 
A natural question at this stage is how to identify the correct substitution.  Unfortunately, the answer is it depends on the integral.  However, there is a general rule of thumb that will work for many of the integrals that we’re going to be running across.
 
When faced with an integral we’ll ask ourselves what we know how to integrate.  With the integral above we can quickly recognize that we know how to integrate
\end{frame}
%=============================================================================================================================== %
\begin{frame}
	\frametitle{Substitution Rule for Integration}
	\large
	
However, we didn’t have just the root we also had stuff in front of the root and (more importantly in this case) stuff under the root.  Since we can only integrate roots if there is just an x under the root a good first guess for the substitution is then to make u be the stuff under the root. 
\end{frame}
%=============================================================================================================================== %
\begin{frame}
	\frametitle{Substitution Rule for Integration}
	\large
	
Another way to think of this is to ask yourself what portion of the integrand has an inside function and can you do the integral with that inside function present.  If you can’t then there is a pretty good chance that the inside function will be the substitution.
\end{frame}
%=============================================================================================================================== %
\begin{frame}
	\frametitle{Substitution Rule for Integration}
	\large
	
 
We will have to be careful however.  There are times when using this general rule can get us in trouble or overly complicate the problem.  We’ll eventually see problems where there are more than one “inside function” and/or integrals that will look very similar and yet use completely different substitutions.  The reality is that the only way to really learn how to do substitutions is to just work lots of problems and eventually you’ll start to get a feel for how these work and you’ll find it easier and easier to identify the proper substitutions.
\end{frame}
%=============================================================================================================================== %
\begin{frame}
	\frametitle{Substitution Rule for Integration}
	\large
	
Now, with that out of the way we should ask the following question.  How, do we know if we got the correct substitution?  Well, upon computing the differential and actually performing the substitution every x in the integral (including the x in the dx) must disappear in the substitution process and the only letters left should be u’s (including a du). 
\end{frame}
%=============================================================================================================================== %
\begin{frame}
	\frametitle{Substitution Rule for Integration}
	\large If there are x’s left over then there is a pretty good chance that we chose the wrong substitution.  Unfortunately, however there is at least one case (we’ll be seeing an example of this in the next section) where the correct substitution will actually leave some x’s and we’ll need to do a little more work to get it to work. 
\end{frame}
%=============================================================================================================================== %
\begin{frame}
	\frametitle{Substitution Rule for Integration}
	\large
Again, it cannot be stressed enough at this point that the only way to really learn how to do substitutions is to just work lots of problems. There are lots of different kinds of problems and after working many problems you’ll start to get a real feel for these problems and after you work enough of them you’ll reach the point where you’ll be able to do simple substitutions in your head without having to actually write anything down.
\end{frame}
%=============================================================================================================================== %
\begin{frame}
	\frametitle{Substitution Rule for Integration}
	\large
As a final note we should point out that often (in fact in almost every case) the differential will not appear exactly in the integrand as it did in the example above and sometimes we’ll need to do some manipulation of the integrand and/or the differential to get all the x’s to disappear in the substitution.
\end{frame}
%=============================================================================================================================== %
\begin{frame}
	\frametitle{Substitution Rule for Integration}
	\large
Let’s work some examples so we can get a better idea on how the substitution rule works.
 
Example 1  Evaluate each of the following integrals.
(a)    [Solution]
(b)    [Solution]
(c)    [Solution]
(d)    [Solution]
\end{frame}
%=============================================================================================================================== %
\begin{frame}
	\frametitle{Substitution Rule for Integration}
	\large
Solution
(a) 
In this case we know how to integrate just a cosine so let’s make the substitution the stuff that is inside the cosine.
                                         
So, as with the first example we worked the stuff in front of the cosine appears exactly in the differential.  The integral is then,
                                    
\end{frame}
%=============================================================================================================================== %
\begin{frame}
	\frametitle{Substitution Rule for Integration}
	\large 
Don’t forget to go back to the original variable in the problem.
[Return to Problems]
 
(b) 
Again, we know how to integrate an exponential by itself so it looks like the substitution for this problem should be,

\end{frame}
%=============================================================================================================================== %
\begin{frame}
	\frametitle{Substitution Rule for Integration}
	\large
Now, with the exception of the 3 the stuff in front of the exponential appears exactly in the differential.  Recall however that we can factor the 3 out of the integral and so it won’t cause any problems.  The integral is then,
                                               
[Return to Problems]
\end{frame}
%=============================================================================================================================== %
\begin{frame}
	\frametitle{Substitution Rule for Integration}
	\large
(c) 
In this case it looks like the following should be the substitution.
                                           
Okay, now we have a small problem.  We’ve got an x2 out in front of the parenthesis but we don’t have a “-30”.  This is not really the problem it might appear to be at first.  We will simply rewrite the differential as follows.
\end{frame}
%=============================================================================================================================== %
\begin{frame}
	\frametitle{Substitution Rule for Integration}
	\large                                                  
With this we can now substitute the “x2 dx” away.  In the process we will pick up a constant, but that isn’t a problem since it can always be factored out of the integral.
 
We can now do the integral.
                                        
\end{frame}
%=============================================================================================================================== %
\begin{frame}
	\frametitle{Substitution Rule for Integration}
	\large
Note that in most problems when we pick up a constant as we did in this example we will generally factor it out of the integral in the same step that we substitute it in. 
[Return to Problems]
 
\end{frame}
%=============================================================================================================================== %
\begin{frame}
	\frametitle{Substitution Rule for Integration}
	\large
(d) 
In this example don’t forget to bring the root up to the numerator and change it into fractional exponent form.  Upon doing this we can see that the substitution is,
                        
The integral is then,

[Return to Problems]
 
\end{frame}
%=============================================================================================================================== %
\begin{frame}
	\frametitle{Substitution Rule for Integration}
	\large
In the previous set of examples the substitution was generally pretty clear.  There was exactly one term that had an “inside function” that we also couldn’t integrate.  Let’s take a look at some more complicated problems to make sure we don’t come to expect all substitutions are like those in the previous set of examples.
\end{frame}
%=============================================================================================================================== %
\begin{frame}
	\frametitle{Substitution Rule for Integration}
	\large
Example 2  Evaluate each of the following integrals.
(a)    [Solution]
(b)    [Solution]
(c)    [Solution]

\end{frame}
%=============================================================================================================================== %
\begin{frame}
	\frametitle{Substitution Rule for Integration}
	\large
Solution
(a) 
In this problem there are two “inside functions”.  There is the  that is inside the two trig functions and there is also the term that is raised to the 4th power. 
\end{frame}
%=============================================================================================================================== %
\begin{frame}
	\frametitle{Substitution Rule for Integration}
	\large
There are two ways to proceed with this problem.  The first idea that many students have is substitute the  away.  There is nothing wrong with doing this but it doesn’t eliminate the problem of the term to the 4th power.  That’s still there and if we used this idea we would then need to do a second substitution to deal with that.
\end{frame}
%=============================================================================================================================== %
\begin{frame}
	\frametitle{Substitution Rule for Integration}
	\large
The second (and much easier) way of doing this problem is to just deal with the stuff raised to the 4th power and see what we get.  The substitution in this case would be,
 
\end{frame}
%=============================================================================================================================== %
\begin{frame}
	\frametitle{Substitution Rule for Integration}
	\large         
 
Two things to note here.  First, don’t forget to correctly deal with the “-”.  A common mistake at this point is to lose it.  Secondly, notice that the  turns out to not really be a problem after all.  Because the  was “buried” in the substitution that we actually used it was also taken care of at the same time.  The integral is then,
                            
 \end{frame}
 %=============================================================================================================================== %
 \begin{frame}
 	\frametitle{Substitution Rule for Integration}
 	\large
As seen in this example sometimes there will seem to be two substitutions that will need to be done however, if one of them is buried inside of another substitution then we’ll only really need to do one.  Recognizing this can save a lot of time in working some of these problems.
[Return to Problems]
\end{frame}
%=============================================================================================================================== %
\begin{frame}
	\frametitle{Substitution Rule for Integration}
	\large
(b) 
This one is a little tricky at first.  We can see the correct substitution by recalling that,
                                                      
 
Using this it looks like the correct substitution is,
                
\end{frame}
%=============================================================================================================================== %
\begin{frame}
	\frametitle{Substitution Rule for Integration}
	\large 
Notice that we again had two apparent substitutions in this integral but again the 3z is buried in the substitution we’re using and so we didn’t need to worry about it.
Here is the integral.
                                         
\end{frame}
%=============================================================================================================================== %
\begin{frame}
	\frametitle{Substitution Rule for Integration}
	\large 
Note that the one third in front of the integral came about from the substitution on the differential and we just factored it out to the front of the integral.  This is what we will usually do with these constants.
[Return to Problems]
 
\end{frame}
%=============================================================================================================================== %
\begin{frame}
	\frametitle{Substitution Rule for Integration}
	\large
(c) 
In this case we’ve got a 4t, a secant squared as well as a term cubed.  However, it looks like if we use the following substitution the first two issues are going to be taken care of for us.
                
\end{frame}
%=============================================================================================================================== %
\begin{frame}
	\frametitle{Substitution Rule for Integration}
	\large 
The integral is now,
                                
[Return to Problems]

\end{frame}
%=============================================================================================================================== %
\begin{frame}
	\frametitle{Substitution Rule for Integration}
	\large 
The most important thing to remember in substitution problems is that after the substitution all the original variables need to disappear from the integral.  After the substitution the only variables that should be present in the integral should be the new variable from the substitution (usually u).  Note as well that this includes the variables in the differential!
 
\end{frame}

\end{document}