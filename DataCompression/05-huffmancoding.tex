5 Huffman coding 35
%----------------------------------------------%
Essential reading . . . . . . . . . . . . . . . . . . . . . . . . . . . . . . 35
Further reading . . . . . . . . . . . . . . . . . . . . . . . . . . . . . . . 35
Static Huffman coding . . . . . . . . . . . . . . . . . . . . . . . . . . . 35
Huffman algorithm . . . . . . . . . . . . . . . . . . . . . . . . . . . . . 35
Building the binary tree . . . . . . . . . . . . . . . . . . . . . . . 35
Canonical and minimum-variance . . . . . . . . . . . . . . . . . . 36
Implementation efficiency . . . . . . . . . . . . . . . . . . . . . . . . . . 36
Observation . . . . . . . . . . . . . . . . . . . . . . . . . . . . . . . . . 38
A problem in Huffman codes . . . . . . . . . . . . . . . . . . . . . . . . 38
Extended Huffman coding . . . . . . . . . . . . . . . . . . . . . . . . . 38
Learning outcomes . . . . . . . . . . . . . . . . . . . . . . . . . . . . . 39
%----------------------------------------------%
