\documentclass[]{article}


\begin{document}
\tableofcontents
\newpage

%---------------------------------------------------------%
\section{2010 Question 1}
(a) Draw a flow chart to outline the general stages for developing a compression algorithm. [5]
(b) Discuss the absolute limit of lossless compression with the aid of an example.
Explain why only a proportion of the files of size n can be compressed by one
byte. What is the percentage of the files that can be compressed by one byte? [5]
(c) Consider a source (A, B, C, D) with a probability distribution (0.4, 0.3, 0.2,0.1). 
Discuss whether each of the following statements is true or false. Provide supporting arguments or evidence to justify your answer. [8]
\begin{itemize}
\item[i.] The binary tree representing a fixed length binary code is not optimal.
\item[ii.] The Shannon�Fano code for the given source in the question is optimal.
\end{itemize}
(d) Explain and demonstrate how the compression efficiency of the Shannon�Fano
encoding can be improved by alphabet extension. Use a binary alphabet (A, B)
with the probability of A being 0.3 as an example. [7]
%---------------------------------------------------------%
\section{2010 Question 2}

\begin{itemize}
\item[(a)] Consider the alphabet (A, B, C, D) of a source. Discuss the possibility of finding: [5]
\begin{itemize}
\item[i.] A uniquely decodable binary code in which the codeword for A is of length 2,
that for B of length 1 and for both C and D of length 3.
\item[ii.] A shorter variable length prefix code than the one described in (a)i.
\end{itemize}
Provide evidence or justification for your answers.
\item[(b)] Consider a segment of bitmap image that is represented by the array of characters
below: [10]
\begin{verbatim}
	A D D B B
	C D C A A
	C C C B A
	D D D A A
\end{verbatim}

\begin{itemize}
\item[i.] Compute the probability distribution of the characters.
\item[ii.] What is the minimum length of the encoded file for the segment? Show all
your work and assumptions.
\item[ii.] Assume that the whole image is represented in the same distribution as that
of the segment, and is stored in a text file. What is the minimum number of
bits per character on average?
\end{itemize}
\end{itemize}
\begin{itemize}
\item[(c)] Explain sc = Q + (S � T) / 2, a predictive rule of JPEG. Demonstrate, with the aid
of a small example, how this rule can be applied in preprocessing. Assume the
pixel layout as follows: [5]
T S
El X?
\item[(d)] Consider a binary code that is not a prefix code. Can we conclude that the code
is therefore not uniquely decodable'? Explain your answer. \\Give one example to
support your answer. [5]
\end{itemize}

%---------------------------------------------------------%
\section{2010 Question 3}
\begin{itemize}
\item[(a)] Derive the reflected Grey code for the decimal numeral 12. [5]
\item[(b)] Draw a Howchart to outline the adaptive Huffman algorithm for encoding. [5]
(c) Draw a diagram to outline the LZ77 decoding algorithm. [10]
\\ 
\\
Following the approach of the LZ77 algorithm, decode the tokens (O ,ASCII (A)) ,
(0,AscI1(A)>, <0,Asc1I(B)), <0,Asc1r(A)>, <0,Asc11(c>), <0,Asc1I(c>>,
(5,2) , (6,2), (O,ASCII(A)) , (8,3), (O,ASCII(C)). \\ Assume that the length
of the history buffer is H = 8 and of the lookahead buffer is L : 6. \\ The history
butler is empty initially.

\item[(d)] Explain what each of the variables L, z, s1, p2 represents in the segment of the
Arithmetic decoding algorithm below. Demonstrate how the algorithm works with
the aid of a small example. \\ Assume a source (A, B) with a. probability distribution
(0.2, 0.8).\\
Hint: You may trace variable values at the end of each iteration 0-5 for an input
0.43 as suggested in the table below. \\ Use iteration O to describe the initial state
and add necessary assumptions for a specific source. Finally, derive the decoded
text. [5]
\end{itemize}
%1. L<�Oandd<-1
%2. If x is within [L, L+d=�=p1)
%then output s1, leave L unchanged, and
%set d <� d�=p1
%else if x is within [L+d*p1 , L+d)
%then output s2, set L <� L+d=�=p1 and d <- d*p2
%3. If (the_nu.mbe1�_of_decoded_symbols
%< the_1�equired_n11mbex�_of_sy�mbols)
%then go to step 2.
%Iteration L E d*p1 d*p2 [L, L+d*p1) [L+d*p1, L+d) Output
%-----
%1 -------
%2 -------
%3 -------
%4 -------
%5 -------
%291 0325 ZB 2010 PAGE 4 of4 END OF EXAMINATION
%ULIO/488ZB

\end{document}