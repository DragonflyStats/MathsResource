
\documentclass{beamer}

\usepackage{amsmath}
\usepackage{graphicx}
\usepackage{amssymb}
\begin{document}


\begin{frame}
	\frametitle{Example 2: Evaluating a Function}
	Evaluate the following function for x =1,2 and 5 respectively.
	{
		\Large
	\[ f(x) = \frac{e^x + {e^{-x}}}{2} \]
}
	Remark:  Example 1 was done in previous class.
\end{frame}
%================================================================================== %
\begin{frame}
	
	\frametitle{Example 3: Evaluating a Function}
	Evaluate the function for each of the following values : 0.5,1,1.25,2.
	{
		\Large
	\[f(x) =  \sqrt{1+e^{x}}  \]
}
	Four decimal places will suffice.
\end{frame}
\begin{frame}
	\begin{center}
		\begin{tabular}{|c|c|c|c|}
			\hline \rule[-2ex]{0pt}{5.5ex} x & $e^x$ & $1+e^x$ & $\sqrt{1+e^x}$ \\ \hline
			\hline \rule[-2ex]{0pt}{5.5ex} 0.5 & \phantom{space}  &  &  \\ 
			\hline \rule[-2ex]{0pt}{5.5ex} 1 &  &  &  \\ 
			\hline \rule[-2ex]{0pt}{5.5ex} 1.25 & \phantom{space}  &  &  \\ 
			\hline \rule[-2ex]{0pt}{5.5ex} 2 & \phantom{space}  &  & \phantom{space}  \\ 
			\hline 
		\end{tabular} 
	\end{center}
\end{frame}
%================================================================================ %
\begin{frame}
	\frametitle{Recall : Sets of Numbers}
	\begin{itemize} 
		\item $\mathbb{N}$ Set of all natural numbers
		\item $\mathbb{Z}$ Set of all integers
		\item $\mathbb{Q}$ Set of all rational numbers
		\item $\mathbb{R}$ Set of all real numbers
	\end{itemize}
	
	\bigskip There are, of course, other numbers sets, but we will not be encountering them on the course.
\end{frame}
%========================================================== %

\begin{frame}
	\frametitle{Recall : Sets of Numbers}
	\begin{itemize}
		\item $\mathbb{Z}^{+}$ Set of all positive integers
		\item $\mathbb{Z}^{-}$ Set of all negative integers
		\item $\mathbb{R}^{+}$ Set of all positive real numbers
		\item $\mathbb{R}^{-}$ Set of all negative real numbers
	\end{itemize}
\end{frame}

%=========================================================================================== %
\subsection*{Topic 1 : Special Functions}
\begin{frame}
	\frametitle{Special Functions}
	\begin{itemize}
		% \item The factorial Operator
		\item Absolute Value Function
		\item The Sign Function
		\item Floor and Ceiling Functions
		\item Hyperbolic Functions
	\end{itemize}	
\end{frame}


%---------------------------------------%
\begin{frame}
	\frametitle{Absolute Value Function}
	\Large
	\textbf{Absolute Value Function}
	\begin{itemize}
		\item The absolute value (or modulus) $|x|$ of a real number x is the non-negative value of x without regard to its sign. 
		
		%	\item For example, $|x| = x$ for a positive x, $|x| = -x$ for a negative x (in which case −x is positive), and $|0| = 0$. 
		%	\item For example, the absolute value of 4 is 4, and the absolute value of $-4$ is also 4. 
		%\item The absolute value of a number may be thought of as its distance from zero.
	\end{itemize}
	\[|x| = \begin{cases} x, & \mbox{if }  x \ge 0  \\ -x,  & \mbox{if } x < 0. \end{cases} \]
\end{frame}
%---------------------------------------%
\begin{frame}
	\frametitle{Topic 1 : Absolute Value Function}
	\Large
	\vspace{-1cm}
	\begin{itemize}
		
		\item For a positive x, $|x| = x$ 
		\item For a negative x (in which case −x is positive) $|x| = -x$ 
		\item The absolute value of 0 is 0:  $|0| = 0$. 
		\item For example, the absolute value of 4 is 4, and the absolute value of $-4$ is also 4. 
		\item IMPORTANT:  The input to this function is any real number. The output of this function will always be a positive real numbers.
		%\item The absolute value of a number may be thought of as its distance from zero.
	\end{itemize}
	
	
\end{frame}
\begin{frame}
	\frametitle{Topic 1 : Sign Function}
	\Large
	\textbf{Sign Function}
	\begin{itemize}
		\item The sign function $sng(x)$ of a real number x is a signed value of absolute value of 1, dependent on the sign of $x$. 
		
		%	\item For example, $|x| = x$ for a positive x, $|x| = -x$ for a negative x (in which case −x is positive), and $|0| = 0$. 
		%	\item For example, the absolute value of 4 is 4, and the absolute value of $-4$ is also 4. 
		%\item The absolute value of a number may be thought of as its distance from zero.
	\item IMPORTANT:  The input to this function is any real number. The output of this function will always be either 1 or -1.
	\end{itemize}
	\[ sgn(x) = \begin{cases} 1, & \mbox{if }  x \ge 0  \\ -1,  & \mbox{if } x < 0. \end{cases} \]
\end{frame}
%---------------------------------------%
\begin{frame}
	\frametitle{Topic 1 : Floor and Ceiling Functions}
	\Large
	\begin{itemize}
		\item The floor and ceiling functions map a real number to the largest previous or the smallest following integer, respectively. \item More precisely, \[floor(x) = \lfloor x\rfloor \] is the largest integer not greater than x and \[ceiling(x) =  \lceil x \rceil \] is the smallest integer not less than x.
	\end{itemize}
	
	
\end{frame}

%---------------------------------------%
\begin{frame}
	\frametitle{Topic 1 : Floor and Ceiling Functions}
	\textbf{Examples}
	{
		\LARGE
		\begin{eqnarray}
		\lfloor 3.14 \rfloor =& 3 \\ \bigskip
		\lceil -4.5 \rceil =& -5 \\ \bigskip
		| -4 | =&  4
		\end{eqnarray}
}
{\Large	
	\textbf{Remark:}  Input to the floor and ceiling function can be any really number, but outputs are always integers.
	}
\end{frame}

%====================================================================%
\begin{frame}
\frametitle{Even and Odd functions}
\Large
\textbf{Even Functions}\\
Then f is even if the following equation holds for all x and -x in the domain of f:

\[f(x) = f(-x), \,\]

Geometrically speaking, the graph face of an even function is symmetric with respect to the y-axis, meaning that its graph remains unchanged after reflection about the y-axis.
%Examples of even functions are |x|, x2, x4, cos(x), and cosh(x).
\end{frame}
%================================================================= %
\begin{frame}
\frametitle{Even and Odd functions}
\Large
\textbf{Odd Functions}\\
Let f(x) be a real-valued function of a real variable. Then f is odd if the following equation holds for all x and -x in the domain of f:

\[-f(x) = f(-x), \,\]
or

\[f(x) + f(-x) = 0. \,\]
\end{frame}

\begin{frame}
	\frametitle{Even and Odd functions}
	\begin{itemize}
    \item \textbf{Important:} A function may be neither even nor odd.
	\item Discussion with examples on Blackboard
	\item Examples of Questions from Past Papers done on board
	\end{itemize}
	
\end{frame}
	
%====================================================%
\begin{frame}
	\frametitle{Tech Maths 2}
	\textbf{Cross Multiplication}
	\Large 
	\begin{itemize}
		\item Can simplify an expression by multiplying both the numerator and denominator by same term.
		\item This does not change the value of the expression.
		\item Remark
		{
			\LARGE
			\[ \frac{A}{B} + \frac{X}{Y} = \frac{AY}{BY} + \frac{BX}{BY} = \frac{AY+BX}{BY}\]
		}
	\end{itemize}
\end{frame}
%======================================================================= %


\begin{frame}
	\frametitle{Cross Mutliplication}
	\Large
	\[  \frac{p}{x+a} + \frac{q}{x+b}   = \frac{p(x+b) + q(x+a)}{(x+a)(x+b)} = \frac{(p+q)x+ (pb+aq)}{(x+a)(x+b)} \]
	
	\begin{itemize}
		\item $\{p,q,a,b\} \in R$
	\end{itemize}
	
\end{frame}
%======================================================================= 
\begin{frame}
	\frametitle{Cross Mutliplication}
	\Large
	\[  \frac{4}{x+2} + \frac{2}{x-1}   = \frac{4(x-1) + 2(x+2)}{(x+2)(x-1)} \]
	\[ = \frac{(4+2)x + (4(-1)+(2\times 2)}{(x+2)(x-1)} \]
	\[ = \frac{2x}{x^2+x-2}\]
	
\end{frame}
%====================================================%
\begin{frame}
	\frametitle{Tech Maths 2}
	\textbf{Cross-Multiplication: Example 1}
	% \textbf{Lecture 1B}
	\begin{itemize}
		\item Solve the following Equation for A and B
		\item $A,B \in \mathbb{R}$
	\end{itemize}
	
	\[ \frac{2x + 5}{x^2 - 4x - 12} = \frac{A}{x-6} + \frac{B}{x+2}\]
\end{frame}
%====================================================%
\begin{frame}
	\frametitle{Tech Maths 2}
	%\textbf{Lecture 1B}
	\textbf{Cross-Multiplication: Example 2}
	\begin{itemize}
		\item Solve the following Equation for A and B
		\item $A,B \in \mathbb{R}$
	\end{itemize}
	
	\[ \frac{5}{x^2 - 4x - 12} = \frac{A}{x-6} + \frac{B}{x+2}\]
\end{frame}


%======================================================================= %







\begin{frame}
	\frametitle{ Topic 2 : Laws of Logarithms}
	\Large
	\begin{itemize}
		\item Law 1 : Multiplcation of Logarithms
		\[ Log(a) \times Log(b) = Log(a+b) \]
		\item Law 2 : Division of Logarithms
		\[ \frac{Log(a)}{Log(b)} = Log(a-b) \]
		\item Law 3 : Powers of Logarithms
		\[ Log(a^b) = b \times Log(a) \]
	\end{itemize}
	
	
	
\end{frame}
\end{document}

