\documentclass{beamer}

\usepackage{amsmath}
\usepackage{amssymb}


\usepackage{amsthm, amsmath}
%\usepackage[dvips]{graphicx}



\begin{document}
	
	\begin{frame}
		\frametitle{The Power Rule of Differentiation}
		The power rule is one of the most important differentiation rules in calculus. Since differentiation is linear, polynomials can be differentiated using this rule.
		
		\[ \frac{d}{dx} x^n = nx^{n-1} , \qquad n \neq 0.\]
		
		\[\int\! x^n \,dx= \frac{ x^{n+1}}{n+1} + c, \qquad n \neq -1.\]
		
		\[\int \! x^{-1}\, dx= \ln |x|+ c\]
			\end{frame}
			%----------------------------------------------------%
			\begin{frame}
		\frametitle{Trigonometric Functions}
		
		\[ (\sin x)' = \cos x \,\]
		\[ (\cos x)' = -\sin x \, \]
		
			\end{frame}
			%----------------------------------------------------%
			\begin{frame}
		\frametitle{Integration by parts}
		
		\[ \int u \, dv=uv-\int v \, du. \]
			\end{frame}
			%----------------------------------------------------%
			\begin{frame}
		
		In order to calculate
		
		\[I=\int x\cos (x) \,dx\,\]
		
		let:
		\[u = x \Rightarrow d u = dx\]
		\[dv = \cos(x)\,dx \Rightarrow v = \int\cos(x)\,dx = \sin x\]
		
		then:
			\end{frame}
			%----------------------------------------------------%
			\begin{frame}
		\begin{eqnarray*}
			\int x\cos (x) \,dx & =& \int u \, dv \\
			& = & uv - \int v \, du \\
			& = & x\sin (x) - \int \sin (x) \,dx \\
			& = & x\sin (x) + \cos (x) + C,
		\end{eqnarray*}
		
		
		where $C$ is an arbitrary constant of integration.
			\end{frame}
			%----------------------------------------------------%
			\begin{frame}
		\frametitle{Integration by Substitution: Example}
		
		\begin{eqnarray*}
			& {} \quad \int x \cos(x^2+1) \,dx = \frac{1}{2} \int 2x \cos(x^2+1) \,dx \\
			& {} = \frac{1}{2} \int\cos u\,du \\ & { }= \frac{1}{2}\sin u + C = \frac{1}{2}\sin(x^2+1) + C
		\end{eqnarray*}
			\end{frame}
			%----------------------------------------------------%
			\begin{frame}
		\frametitle{Additivity of integration on intervals}
		\begin{eqnarray*}
			\int_a^c f(x) \, dx &{}= \int_a^b f(x) \, dx - \int_c^b f(x) \, dx \\
			&{} = \int_a^b f(x) \, dx + \int_b^c f(x) \, dx
		\end{eqnarray*}
			\end{frame}
			%----------------------------------------------------%
			\begin{frame}
		\frametitle{Hyperbolic Function}
		
		\begin{eqnarray*}
			\sinh x  & =& \frac{e^x - e^{-x}}{2} \\
			\cosh x  & = & \frac{e^x + e^{-x}}{2} \\
		\end{eqnarray*}
		
		\[( \sinh x )'= \cosh x = \frac{e^x + e^{-x}}{2}\]
		
			\end{frame}
			%----------------------------------------------------%
			\begin{frame}
		%---------------------------------------------------------
		\frametitle{Jacobian}
		\[J=\begin{bmatrix} \dfrac{\partial F_1}{\partial x_1} & \cdots & \dfrac{\partial F_1}{\partial x_n} \\ \vdots & \ddots & \vdots \\ \dfrac{\partial F_m}{\partial x_1} & \cdots & \dfrac{\partial F_m}{\partial x_n}  \end{bmatrix}.\]
			\end{frame}
			%----------------------------------------------------%
			\begin{frame}
		
		\frametitle{Logarithmic Transformation}
		\[ (\ln f)'= \frac{f'}{f}\]
			\end{frame}
			%----------------------------------------------------%
			\begin{frame}
		\frametitle{Quotient Rule}
		
		\[f(x) = \frac{u(x)}{v(x)}\]
		
		
		\[f'(x) = \frac{u'(x)v(x) - u(x)v'(x)}{[v(x)]^2}\]
			\end{frame}
			%----------------------------------------------------%
			\begin{frame}
		\frametitle{Chain Rule}
		
		h(x) = sin(2x) is the composition of the functions
		f(g) = sin g and g(x) = 2x:
		As such we have
		$f^{\prime}(g)$ = cos g and $g^{\prime}(x) = 2$
		
		;
		and so the chain rule tells us that
		\[h^{\prime}(x) = (cos g)(2) = 2 cos(2x)\]
			\end{frame}
		
			%----------------------------------------------------%
\begin{frame}
%------------------------------------------------------------------------%
\frametitle{Maclaurin Series}
\[f(a)+\frac {f'(a)}{1!} (x-a)+ \frac{f''(a)}{2!} (x-a)^2+\frac{f^{(3)}(a)}{3!}(x-a)^3+ \cdots.\]
		
\[\sum_{n=0} ^ {\infty } \frac {f^{(n)}(a)}{n!} \, (x-a)^{n}\]
\end{frame}
%========================================================================================== %
\begin{frame}
		\frametitle{Fundamental Theorem of Calculus}
		The fundamental theorem of calculus states that the integral of a function f over the interval $[a, b]$ can be calculated by finding an antiderivative F of f:
		\[\int_a^b f(x)\,\mathrm dx = F(b) - F(a).\]
		
	\end{frame}	
		
\begin{frame}
%-------------------------------%
\frametitle{ODEs: Integrating factor}
		The integrating factor is a function that is chosen to facilitate the solving of a given equation involving differentials. It is commonly used to solve ordinary differential equations.
		
		\[ y'+ P(x)y = Q(x)\]
		
		the integration factor is
		\[M(x) = e^{\int P(x') dx'}\]
		
\end{frame}
%========================================================================================== %
\begin{frame}		
		\frametitle{ODEs: Example}
		
		Solve the differential equation
		
		\[y'-\frac{2y}{x} = 0.\]
		
		We can see that in this case \[P(x) = \frac{-2}{x}\]
		
		\[ M(x)=e^{\int P(x)\,dx}\]
\end{frame}
%========================================================================================== %
\begin{frame}		
		\[ M(x)=e^{\int \frac{-2}{x}\,dx} = e^{-2 \ln x} = {(e^{\ln x})}^{-2} = x^{-2} \] (Note we do not need to include the integrating constant - we need only a solution, not the general solution)
		
		\[ M(x)=\frac{1}{x^2}.\]
\end{frame}
%========================================================================================== %
\begin{frame}		
		Multiplying both sides by \[M(x)\] we obtain
		
		\[\frac{y'}{x^2} - \frac{2y}{x^3} = 0\]
		
		\[\frac{y'x^3 - 2x^2y}{x^5} = 0\]
		
		\[\frac{x(y'x^2 - 2xy)}{x^5} = 0\]
		
		\[\frac{y'x^2 - 2xy}{x^4} = 0.\]
\end{frame}
%========================================================================================== %
\begin{frame}		
\frametitle{Partial Derivatives: Volume of a Cone}
		
		The volume ''V'' of a cone depends on the cone's height ''h'' and its radius 'r' according to the formula
		\[V(r, h) = \frac{\pi r^2 h}{3}.\]
		The partial derivative of ''V'' with respect to 'r' is
		\[\frac{ \partial V}{\partial r} = \frac{ 2 \pi r h}{3},\]
\end{frame}
	%----------------------------------------------------%
	\begin{frame}
		which represents the rate with which a cone's volume changes if its radius is varied and its height is kept constant.
		The partial derivative with respect to ''h'' is
		\[\frac{ \partial V}{\partial h} = \frac{\pi r^2}{3},\]
		
		which represents the rate with which the volume changes if its height is varied and its radius is kept constant.
\end{frame}
%----------------------------------------------------%
\begin{frame}
\frametitle{Fundamental Theorem of Calculus}
The fundamental theorem of calculus states that the integral of a function f over the interval $[a, b]$ can be calculated by finding an antiderivative F of f:
		\[\int_a^b f(x)\,\mathrm dx = F(b) - F(a).\]
		
		
			\end{frame}
			%----------------------------------------------------%
			\begin{frame}
		\frametitle{Numerical Integration}
		Numerical integration constitutes a broad family of algorithms for calculating the numerical value of a definite integral, and by extension, the term is also sometimes used to describe the numerical solution of differential equations.
		\frametitle{Numerical Integration: Simpson's Rule}
		\[\int_{a}^{b} f(x) \, dx \approx \frac{b-a}{6}\left[f(a) + 4f\left(\frac{a+b}{2}\right)+f(b)\right].\]
		
		
		
			\end{frame}
			%----------------------------------------------------%
			\begin{frame}
		\frametitle{ODEs: Integrating factor}
		The integrating factor is a function that is chosen to facilitate the solving of a given equation involving differentials. It is commonly used to solve ordinary differential equations.
		
		\[ y'+ P(x)y = Q(x)\]
		
		the integration factor is
		\[M(x) = e^{\int P(x') dx'}\]
		
			\end{frame}
			%----------------------------------------------------%
			\begin{frame}
		\frametitle{ODEs: Example}
		
		Solve the differential equation
		
		\[y'-\frac{2y}{x} = 0.\]
			\end{frame}
			%----------------------------------------------------%
			\begin{frame}
		We can see that in this case \[P(x) = \frac{-2}{x}\]
		
		\[ M(x)=e^{\int P(x)\,dx}\]
		
		\[ M(x)=e^{\int \frac{-2}{x}\,dx} = e^{-2 \ln x} = {(e^{\ln x})}^{-2} = x^{-2} \] (Note we do not need to include the integrating constant - we need only a solution, not the general solution)
		
		\[ M(x)=\frac{1}{x^2}.\]
			\end{frame}
			%----------------------------------------------------%
			\begin{frame}
		Multiplying both sides by \[M(x)\] we obtain
		
		\[\frac{y'}{x^2} - \frac{2y}{x^3} = 0\]
		
		\[\frac{y'x^3 - 2x^2y}{x^5} = 0\]
		
		\[\frac{x(y'x^2 - 2xy)}{x^5} = 0\]
		
		\[\frac{y'x^2 - 2xy}{x^4} = 0.\]
		
		
			\end{frame}
			%----------------------------------------------------%
			\begin{frame}
		
		\frametitle{Fundamental Theorem of Calculus}
		The fundamental theorem of calculus states that the integral of a function f over the interval $[a, b]$ can be calculated by finding an antiderivative F of f:
		\[\int_a^b f(x)\,\mathrm dx = F(b) - F(a).\]
			\end{frame}
			%----------------------------------------------------%
			\begin{frame}
		
%		
%		\section{Period of a trigonomteric function}
%		Period of a function is denoted $2l$. (Sometimes it is denoted as
%		$L$, with $L=2l$). \newline When given a trigonometric function in
%		form $f(t) = Cos(kx)$ or $f(t)= Sin(kx)$, the period of the
%		function can be calculated as follows:
%		
%		\begin{eqnarray}
%		2l = \frac{2 \pi}{k}\nonumber
%		\end{eqnarray}
%		
%		
%		\subsection{Example}
%		\begin{eqnarray}f(t) = Cos(\frac{2 \pi x}{3}) \nonumber\\
%		2l \quad=\frac{2\pi}{(\frac{2\pi}{3})}\quad =
%		\frac{1}{(\frac{1}{3})}\quad= \textbf{3}\nonumber
%		\end{eqnarray}
%		
%		\subsection{Example}
%		\begin{eqnarray}f(t) = Sin(\frac{5x}{2}) \nonumber\\
%		2l \quad=\frac{2\pi}{(\frac{5}{2})}\quad = \frac{4\pi}{5}\nonumber
%		\end{eqnarray}
		
%		\section{Even and Odd Function}
%		Even Functions: $Cos(X)$ ,$|X|$ (i.e absolute value of $X$) and
%		$X^2$, $X^4$ etc
%		\newline
%		Odd Functions: $Sin(X)$, $X$, $X^3$ etc
%		\\
%		Functions that are products of two even functions are also
%		\textbf{even} functions.
%		\newline
%		Functions that are products of two odd functions are
%		\textbf{even} functions. (e.g $X \times X^3 = X^4$)
%		\newline
%		Functions that are products of an even function and an
%		odd function are \textbf{odd} functions.
%		\section{Fourier Series - determining the arguments}
%		Given a period $2l$, we must determine the form of the fourier
%		series. $sin( \frac{n x \pi}{l})$
%		\section{Fourier Series}
%		X
		%MA4005
		%MA4006
		%UolEP Maths
		%Clifford Nolan - Maths Class
		
			\end{frame}
			%----------------------------------------------------%
			\begin{frame}
		\frametitle{Integration by Parts}
		\[ \int xe^xdx \]
		
		Let $u = x$  therefore ${du \over dx} =1$ and hence $du=dx$
		
		Let $dv = e^xdx$.
		\[ v= int dv = int e^x dx = e^x\]
		
		\[ I = uv - int vdu \]
		
		\[ I = xe^x -int e^x dx = xe^x - e^x + c \]
			\end{frame}
			%----------------------------------------------------%
			\begin{frame}
		\frametitle{First Order Ordinary Differential Equations}
		
		Solve the first order differential Equation
		
		\[ \frac{dy}{dx} + y = x \]
		
		subject to the boundary condition y(0) = 1.
		
		
		
			\end{frame}
			%----------------------------------------------------%
			\begin{frame}
		
		\frametitle{PFE: Examples}
		
		\[ \int \frac{dx}{x^2-9} = {1 \over 6}ln \left| \frac{x-3}{x+3} \right| + c \]
		
		\[ \int \frac{dx}{x^2 + 7x + 6} = {1 \over 5}ln \left| \frac{x+1}{x+6} \right| + c \]
		
		\[ \int \frac{x dx}{(x-2)^2} = ln \left| x-2 \right| - \frac{2}{x-2} + c \]
		
			\end{frame}
			%----------------------------------------------------%
			\begin{frame}
		\frametitle{Absolute Value Function}
		The absolute Value Function: $|x|$\\
		Integrating Absolute Value Function:
		
		% \[ \int^{\infty}_{\infty} |x| dx = \int^{0}_^{-\infty} x dx \int^{\infty}_0 x dx \]
		\end{frame}
		%----------------------------------------------------%
		\begin{frame}
		\frametitle{Integration of Hyberolic Functions}
		
		\[ \int sinh(x) dx = cosh(x) + c\]
		
		\[ \int cosh(x) dx = sinh(x) + c\]
		
		\[ \int \frac{dx}{\sqrt{x^2+a^2}} dx = sinh^{-1}(x/a) + c \qquad x^2 < a^2\]
		
		\[ \int sinh(x/2) dx = 2cosh(x/2) + c\]
			\end{frame}
			%----------------------------------------------------%
			\begin{frame}
		%------------------------------------%
		\frametitle{Chain Rule : Example}
		\[ y= -cos(2x) \]
		\[ \frac{dy}{dx}= 2\times sin(2x) \]
		
		%------------------------------------%
			\end{frame}
			%----------------------------------------------------%
			\begin{frame}
		\frametitle{Partial Fraction Expansion}
		
		\[ \frac{s+1}{s^2(s+2)} = \frac{As+B}{s^2} + \frac{C}{s+2}\]
		
		Cross-multiply the RHS terms
		
		
		\[ \frac{(As+B)(s+2)}{s^2(s+2)} + \frac{(C)(s^2)}{s^2(s+2)} \]
		
		
		\[  = \frac{As^2+Bs +2As +2B}{s^2(s+2)} + \frac{Cs^2}{s^2(s+2)} \]
		
		
		\[  = \frac{(A+C)s^2+(2A+B)s +2B}{s^2(s+2)} \]
		
		\begin{itemize}\item A+C=0
			\item 2B = 1 B=1/2
			\item A=1/4
			\item C=-1/4
		\end{itemize}
		
		%--------------------------------------------------------%
		
	% %	\subsection{Partial Fraction Expansion}
			\end{frame}
			%----------------------------------------------------%
			\begin{frame}
		\[ \int \frac{dx}{x^2-4} \]
		
		$x^2-4 = (x-2)\times(x+2)$
		
		\[ \frac{1}{x^2-4} = \frac{A}{x-2} + \frac{B}{x+2}  \]
		
		\[ \frac{1}{x^2-4} = \frac{A(x+2)}{(x-2)(x+2)} + \frac{B(x-2)}{(x-2)(x+2)}  \]
		
		\[ 1 = A(X+2) +  B(X-2)  \]
		
		\[ 1 = (A+B)x +  (2A-2B)  \]
			\end{frame}
			%----------------------------------------------------%
			\begin{frame}
		\begin{itemize} \item A+B=0
			\item 2A-2B=1
			\item Solving $A=1/4$ and B= $-1/4$
		\end{itemize}
			\end{frame}
			%----------------------------------------------------%
			\begin{frame}
		\[ \frac{1}{x^2-4} = \frac{1/4}{x-2} + \frac{-1/4}{x+2} \]
		
		\[ \int \frac{dx}{x^2-4} = {1 \over 4}\int \frac{dx}{x-2} -{1 \over 4}\int \frac{dx}{x+2} \]
		\[ \int \frac{dx}{x^2-4} = {1 \over 4}ln|x-2|-{1 \over 4}ln|x+2| \]
		
		
			\end{frame}
			%----------------------------------------------------%
			\begin{frame}
		\frametitle{Integrals: Worked Example}
		
		\[ \int x(x+2)^{7/2}dx = {2 \over 9}x(x+2)^{9/2} - \int {2 \over 9}(x+2)^{9/2}dx\]
		
		\[ {2 \over 9 }x (x+2)^{9/2} -  {4 \over 99}(x+2)^{11/2} + c \]
		
		\end{frame}
		%----------------------------------------------------%
		\begin{frame}
		\frametitle{Integrals: Worked Example 2}
		\[ \int \frac{e^x}{e^{2x}-1}dx  \]
		Letting $e^x=u$ we get $du=e^x dx$
		\[ \int \frac{du}{u^2-1} = \int {du}{(u+1)(u-1)}\]
		%----------------------------------------%
		
	%	\section{Partial Differentiation}
			\end{frame}
			%----------------------------------------------------%
			\begin{frame}
		
		%\[ f(x,y) = \frac{ye^{\frac{x^2}{y^2}}} \]
		
		Determine the partial derivatives $\frac{\partial f}{\partial x } $
		and $\frac{\partial f}{\partial y } $
			\end{frame}
			%----------------------------------------------------%
			\begin{frame}
		
		Verify that
		
		\[x\frac{\partial f}{\partial x } +  y\frac{\partial f}{\partial y }= f(x,y)\]
		
		%\[f_x = y\frac{2x}{y^2} e^{{x^2\overy^2}} = \frac{2x}{y} e^{{x^2\overy^2}} \]
		
		%\[f_x = e^{{x^2\over y^2}}  + y\left(\frac{-2x^2}{y^3} e^{{x^2\over y^2}}\right) = left(1- \frac{2x}{y} \right) e^{{x^2\over y^2}} \]
		
		
			\end{frame}
			%----------------------------------------------------%
			\begin{frame}
				Let
		\[A = \left(
		\begin{array}{cccc}
		1 &0 &1& 2\\
		0 &1 &1 &1\\
		2 &-1& 1 &3\\
		\end{array}
		\right)
		\]
		\begin{itemize}
			\item[(i)] Find the rank of A.
			\item[(ii)] Find a basis for the column space of A.
		\end{itemize}
		
		
		
		
		
		
		
		%--------------------------------------------------%
		
			\end{frame}
			%----------------------------------------------------%
			\begin{frame}
		\frametitle{Substitution}
		
		\[ I= \int  \frac{sin(2x^2)}{3x} dx \]
		
		\begin{itemize}
			\item Let $u = 2x^2$
			\item $du/dx = 4x$
			\item $du =4x dx$
		\end{itemize}
		%--------------------------------------------------%
			\end{frame}
			%----------------------------------------------------%
			\begin{frame}
		
		\frametitle{Integration by Parts}
		
		\[ I= \int udv = uv -  \int vdu \]
		
		\textbf{Example}
		\[ I= \int x^2 e^{2x} dx \]
		
		\begin{itemize}
			\item Let $u = x^2 $. We can say that ${ du \over dx} = 2x$. Furthermore $du=2xdx$
			\item $dv  = e^{2x} dx$
		\end{itemize}
		
		
		
		
		
		
			\end{frame}
			%----------------------------------------------------%
			\begin{frame}
		%----------------------------------------%
		\frametitle{Partial Derivatives}
		% 05a 2011 Maths 1
		
		\[ f(x,y) = y^3 - x^3 -2xy + 5\]
		
		\begin{eqnarray}
		f_x = -3x^2 -2y \\ \nonumber
		f_y = 3y^2 -2x \\ \nonumber
		\end{eqnarray}
		
			\end{frame}
			%----------------------------------------------------%
			\begin{frame}
		\frametitle{Partial Derivatives}
		% 05a 2011 Maths 1
		
		\[ f(x,y) = y^3 - x^3 -2xy +5\]
		
		\begin{eqnarray}
		f_x = -3x^2 -2y \\ \nonumber
		f_y = 3y^2 -2x \\ \nonumber
		f_{xx} = -6x \\ \nonumber
		f_{yy} = 6y \\ \nonumber
		f_{xy} =-2 \\ \nonumber
		\end{eqnarray}
		\end{frame}
		%----------------------------------------------------%
		\begin{frame}
		\frametitle{Optimization}
		
		\begin{itemize}
			\item Local Maxima
			\item Local Minima
			\item Saddle Point
			\item Concavity
			\item Convexity
		\end{itemize}
		\end{frame}
		%----------------------------------------------------%
	\end{document}
	
	
	
	