Signed integers are positive or negative whole numbers. The first bit of the binary representation of a signed integer is used to represent the sign (positive or negative) of the number. This means that there is one less bit available to represent the value of the number, so the maximum size of the number is reduced. 
e.g. An unsigned 8 bit integer could go up to 255 while the 7 bits of a signed integer could only go up to 128; the remaining bit of 0 or 1 must indicate the sign. 

If we take 0 to represent a plus sign and a 1 to represent a minus sign then 01111111 = +128 while 11111111 = -128 


In this system +7  would be 00000111 
and -7  would be 10000111 
Adding we get   0  decimal, 10001110 binary = -14 decimal! 

This is not really satisfactory. 


A system which works better is where we evaluate negative numbers by counting down from zero to the negative number. So -1 decimal would be 11111111 (check this by adding decimal 1 or binary 00000001 to it). 
Looking at some numbers above and below zero using this system: 

00000100 = 4
 00000011 = 3
 00000010 = 2
 00000001 = 1
 00000000 = 0
 11111111 =-1
 11111110 =-2
 11111101 =-3
 11111100 =-4 
 Now pairs of positive and negative numbers of the same integer do add up to zero. Negative numbers represented this way are said to be in Two's complement form
 
 %=========================================================================================%

TwosComplement

Using two's complement to represent negative values has the benefit that subtraction and addition are the same. In your case, you can think of 12 - 7 as 12 + (-7). Hence you only need to find the two's complement representation of -7 and add it to +12:
12  001100
-7  111001   -- to get this, invert all bits of 7 (000111) and add 1
----------
 5 1000101


Then discard the carry (indicates overflow), and you have your result: 000101 which equals to 5 as expected.

For your example of -15 + 2, simply follow the same procedure to get the two's complement representation of -15:
15  001111
    110000   -- inverted bits
    110001   -- add 1


Now do the addition as usual:
-15  110001
  2  000010
-----------
res  110011


To see that res indeed equals -13, you can see that it is negative (MSB set). For the magnitude, convert to positive (invert bits, add 1):
res  110011
     001100  -- inverted bits
     001101  -- add 1


Hence the magnitude is 13 as expected.


%=========================================================================%

Information Representation 

 Computer Arithmetic - Binary Subtraction     
 




 To perform a binary subtraction you first have to represent the number to be subtracted in its negative form. This is known as its two's complement.
The two's complement of a binary number is obtained by:


1.Replacing all the 1's with 0's and the 0's with 1's. This is known as its one's complement.


2.Adding 1 to this number by the rules of binary addition.


 Now you have the two's complement.

Example:
 The decimal subtraction 29 - 7 = 22 is the same as adding (29) + (-7) = 22


1. Convert the number to be subtracted to its two's complement: 



00000111 (decimal 7) 
11111000 (one's complement) 
+ 00000001 (add 1) 
11111001 (two's complement) 

11111001 now represents -7.



2. Add



29 00011101 
+- 7 11111001 
22 (1)00010110 

Note that the final carry 1 is ignored.




 
