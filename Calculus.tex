\documentclass[12pt, a4paper]{article}
\usepackage{natbib}
\usepackage{vmargin}
\usepackage{graphicx}
\usepackage{epsfig}
\usepackage{subfigure}
%\usepackage{amscd}
\usepackage{amssymb}
\usepackage{amsbsy}
\usepackage{amsthm, amsmath}
%\usepackage[dvips]{graphicx}

\renewcommand{\baselinestretch}{1.8}

% left top textwidth textheight headheight % headsep footheight footskip
\setmargins{3.0cm}{2.5cm}{15.5 cm}{23.5cm}{0.5cm}{0cm}{1cm}{1cm}

\pagenumbering{arabic}


\begin{document}
\author{Kevin O'Brien}

MA4005 Syllabus
\begin{itemize}
\item Functions of several variables and partial differentiation.
\item The indefinite integral. Integration techniques: of standard functions, by substitution, by parts and using partial fractions.
\item The definite integral. Finding areas, lengths, surface areas, volumes, and moments of inertial.
\item Numerical integration: trapezoidal rule, Simpson's rule.
\item Ordinary differential equations.
\item First order including linear and separable. Linear second order equations with constant coefficients.
\item Numerical solution by Runge-Kutta. The Laplace transform: tables and theorems and solution of linear ODEs.
\item Fourier series: functions of arbitary period, even and odd functions, half-range expansions.
\item Application of Fourier series to solving ODEs.
\item Matrix representation of and solution of systems of linear equations.
\item Matrix algebra: invertibility, determinants.
\item Vector spaces: linear independence, spanning, bases, row and column spaces, rank.
\item Inner products: norms, orthogonanality. Eigenvalues and eignenvectors.
\item Numerical solution of systems of linear equations. Gauss elimination, LU decomposition, Cholesky decomposition,
iterative methods. Extension to non-linear systems using Newton's method.
\end{itemize}
%-------------------------------%
\subsection{convolution}

\[ (f * g)(t) = \int_0^t f(\tau)g(t-\tau)\,d\tau= F(s) \cdot G(s) \]

%-------------------------------%
\subsection{ODEs: Integrating factor}
The integrating factor is a function that is chosen to facilitate the solving of a given equation involving differentials. It is commonly used to solve ordinary differential equations.

\[ y'+ P(x)y = Q(x)\]

the integration factor is 
\[M(x) = e^{\int P(x') dx'}\]


\subsection*{ODEs: Example}

Solve the differential equation

\[y'-\frac{2y}{x} = 0.\]

We can see that in this case \[P(x) = \frac{-2}{x}\]

\[ M(x)=e^{\int P(x)\,dx}\]

\[ M(x)=e^{\int \frac{-2}{x}\,dx} = e^{-2 \ln x} = {(e^{\ln x})}^{-2} = x^{-2} \] (Note we do not need to include the integrating constant - we need only a solution, not the general solution)

\[ M(x)=\frac{1}{x^2}.\]

Multiplying both sides by \[M(x)\] we obtain

\[\frac{y'}{x^2} - \frac{2y}{x^3} = 0\]

\[\frac{y'x^3 - 2x^2y}{x^5} = 0\]

\[\frac{x(y'x^2 - 2xy)}{x^5} = 0\]

\[\frac{y'x^2 - 2xy}{x^4} = 0.\]


%-------------------------------%
\subsection{Partial Derivatives: Volume of a Cone}

The volume ''V'' of a cone depends on the cone's height ''h'' and its radius 'r' according to the formula
\[V(r, h) = \frac{\pi r^2 h}{3}.\]
The partial derivative of ''V'' with respect to 'r' is
\[\frac{ \partial V}{\partial r} = \frac{ 2 \pi r h}{3},\]

which represents the rate with which a cone's volume changes if its radius is varied and its height is kept constant.
The partial derivative with respect to ''h'' is
\[\frac{ \partial V}{\partial h} = \frac{\pi r^2}{3},\]

which represents the rate with which the volume changes if its height is varied and its radius is kept constant.
\section{Fundamental Theorem of Calculus}
The fundamental theorem of calculus states that the integral of a function f over the interval $[a, b]$ can be calculated by finding an antiderivative F of f:
\[\int_a^b f(x)\,\mathrm dx = F(b) - F(a).\]

\section{Curl}
In vector calculus, the curl is a vector operator that describes the infinitesimal rotation of a 3-dimensional vector field. At every point in the field, the curl of that field is represented by a vector. The attributes of this vector (length and direction) characterize the rotation at that point.
%--------------------------------------------------%
\section{Laplacian Operator}
\[\Delta f = \nabla^2 f = \nabla \cdot \nabla f\]

%--------------------------------------------------%
\section{Notation}
\[ \begin{vmatrix} \mathbf{i} & \mathbf{j} & \mathbf{k} \\
{\frac{\partial}{\partial x}} & {\frac{\partial}{\partial y}} & {\frac{\partial}{\partial z}} \\
 F_x & F_y & F_z \end{vmatrix}\]

\[\left(\frac{\partial F_z}{\partial y}  - \frac{\partial F_y}{\partial z}\right) \mathbf{i} + \left(\frac{\partial F_x}{\partial z} - \frac{\partial F_z}{\partial x}\right) \mathbf{j} + \left(\frac{\partial F_y}{\partial x} - \frac{\partial F_x}{\partial y}\right) \mathbf{k}\]

 %--------------------------------------------------%
\section{Example}
\[f(x,y,z)= \ 2x+3y^2-\sin(z)\]

\[ \nabla f=
\frac{\partial f}{\partial x} \mathbf{i} +
\frac{\partial f}{\partial y} \mathbf{j} +
\frac{\partial f}{\partial z} \mathbf{k}
 = 2\mathbf{i}+ 6y\mathbf{j} -\cos(z)\mathbf{k}\]

\section{Stoke's Theorem}
\[
\iint_{\Sigma} \nabla \times \mathbf{F} \cdot d\mathbf{\Sigma} = \oint_{\partial\Sigma} \mathbf{F} \cdot d \mathbf{r}, \]

\section{Laplace Transforms }
If $g(t)=k \times f(t)$ then $G(S) = k \times F(S)$ where $k$ is a
constant. $\mathcal{f(t)}=F(S)$.
\begin{eqnarray}
f(t) &=& (t+1)^2\\
&=& t^2 +2t +1 \nonumber
\end{eqnarray}


\section{Laplace Transforms Using 1st Shifting Theorem}
\begin{eqnarray}
 g(t) = e^{at}f(t) \quad \Leftrightarrow \quad G(S)= F(S-a) \nonumber
\end{eqnarray}
The function $g(t)$ is presented in a form whereby $a$ and $f(t)$
are easily identifiable. First determine $F(S)$ by finding the
Laplace transform of $f(t)$. Then replace all $S$ terms with
$S-a$.

\section{Laplace Transforms Using 2nd Shifting Theorem}

\begin{eqnarray}
 g(t) = u^{a}f(t-a) \quad \Leftrightarrow \quad G(S)= e^{-aS}F(S) \nonumber
\end{eqnarray}
The function $g(t)$ is presented in a form whereby $a$ and
$f(t-a)$ are easily identifiable. ($U_{a}(t)$ is called the unit
step function). First determine $f(t)$ by replace all $t-a$ terms
in $f(t-a)$ with $t$. Then calculate the laplace transform of
$f(t)$ i.e. $F(S)$. The solutions is in form $G(S)= e^{-aS}F(S)$.

\section{Inverse Laplace Transforms (2 questions) }
Partial fraction expansion is used in questions 4 and 5.
\section{Inverse Laplace Transforms 2}

The denominator has form $S^2 - 2aS + a^2 + k$ which is equivalent
to $(S-a)^2 + k$. Therefore $G(S)$ will have form $F(S-a)$
\newline
The function $G(S)$ may have the form $\frac{S+D}{S^2 +(C+D)S +
CD}$, where C and D are constants. This expression simplifies
$\frac{S+D}{(S+C)(S+D)}$ and again to $\frac{1}{S+C}$. The inverse
laplace transform $g(t)$ can be easily determined.
\section{Convolution}
We are asked to find a function h(t) which is the convolution of
two given functions $f(t)$ and $g(t)$. i.e $h(t)=h*g(t)$.\newline
Importantly $H(S) = F(S)\times G(S)$. We determine the laplace
transforms, F(S) and G(S), and multiply them to determine H(S). We
then find the inverse Laplace transform of H(S) to yield our
solution.
\subsection{Example}
Find $h(t)$ when $h(t) = f*g(t)$, with $f(t)= e^{t}$ and $g(t)=
e^{-t}$.\newline

\begin{eqnarray}
f(t) = e^{t} \quad \Leftrightarrow \quad F(S)= \frac{1}{S-1}
 \nonumber\\
g(t) = e^{-t} \quad \Leftrightarrow \quad G(S)= \frac{1}{S+1}
 \nonumber\\
H(S) = F(S)\times G(S) = \frac{1}{(S-1)(S+1)}
 \nonumber
\end{eqnarray}
\subsection{Example}
Find $h(t)$ when $h(t) = f*g(t)$, with $f(t)= t$ and $g(t)=
t^2$.\newline

\begin{eqnarray}
f(t) = t \quad \Leftrightarrow \quad F(S)= \frac{1}{S^2}
 \nonumber\\
g(t) = t^2 \quad \Leftrightarrow \quad G(S)= \frac{2}{S^3}
 \nonumber\\
H(S) = F(S)\times G(S) = \frac{2}{S^5}
 \nonumber\\
(H(S) \mbox{ is in form }  k\frac{n!}{S^{n+1}} )
 \nonumber
\end{eqnarray}

With $n=4$, $n!= 4! = 24$. Solving for $k$, $k \times n! = 2$.
Therefore $k=\frac{1}{12}$. The solution is $\cal{L}^{-1}[H(S)]$

\section{Period of a trigonomteric function}
Period of a function is denoted $2l$. (Sometimes it is denoted as
$L$, with $L=2l$). \newline When given a trigonometric function in
form $f(t) = Cos(kx)$ or $f(t)= Sin(kx)$, the period of the
function can be calculated as follows:

\begin{eqnarray}
2l = \frac{2 \pi}{k}\nonumber
\end{eqnarray}


\subsection{Example}
\begin{eqnarray}f(t) = Cos(\frac{2 \pi x}{3}) \nonumber\\
2l \quad=\frac{2\pi}{(\frac{2\pi}{3})}\quad =
\frac{1}{(\frac{1}{3})}\quad= \textbf{3}\nonumber
\end{eqnarray}

\subsection{Example}
\begin{eqnarray}f(t) = Sin(\frac{5x}{2}) \nonumber\\
2l \quad=\frac{2\pi}{(\frac{5}{2})}\quad = \frac{4\pi}{5}\nonumber
\end{eqnarray}

\section{Even and Odd Function}
Even Functions: $Cos(X)$ ,$|X|$ (i.e absolute value of $X$) and
$X^2$, $X^4$ etc
\newline
Odd Functions: $Sin(X)$, $X$, $X^3$ etc
\\
Functions that are products of two even functions are also
\textbf{even} functions.
\newline
Functions that are products of two odd functions are
\textbf{even} functions. (e.g $X \times X^3 = X^4$)
\newline
Functions that are products of an even function and an
odd function are \textbf{odd} functions.
\section{Fourier Series - determining the arguments}
Given a period $2l$, we must determine the form of the fourier
series. $sin( \frac{n x \pi}{l})$
\section{Fourier Series}
X


\end{document}
