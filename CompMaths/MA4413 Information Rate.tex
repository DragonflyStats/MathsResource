


%-------------------------------------------------------------------------------------------------------------------------------------------------------------------------------------------%


% page 247 

\begin{frame}
\frametitle{Information Rate}
If the time rate at which source X emits symbols is r (symbols/s), the lnformation rate R of the
source is given by

%\[R ; r11(X) b/s (10.10)\]

\end{frame}



\begin{frame} \frametitle{DISCRETE MEMORYLESS CHANNELS}
\textbf{A. Channel Representation:}\\
\begin{itemize}
\item A communication channel is the path or medium through which the symbols flow to the receiver. \item A discrete memoryless channel (DMC) is a statistical model with an input X and an output Y.
During each unit ofthe time (signaling interval), the channel accepts an input symbol from X, and iu
response it generates an output symbol from Y.\item  The channel is "discrete" when the alphabets of X and
Y are both finite.\item It is ``memoryless" when the current output depends on only the current input and
not on any of the previous inputs.\end{itemize}
\end{frame}
%r. - · yi
%XI: X Ptytnt ykzyr
%rn,. · yn

\begin{frame}
\frametitle{Discrete memoryless channel}
A diagram of a DMC with nt inputs and n outputs is illustrated in Fig. l0-1. The input X consists
of input symbols ir, , X2,   xm. 
The a priori probabilities of these source symbols P(.x,) are assumed to be known. 
The output Y consists of output symbols $\{y_1,y_2,\ldots, y_i \}$

Each possible input-to-output path is indimted along with a conditional probability $P(y_i|x_i)$, where$P(y_i|x_i)$  is the conditional probability of
obtaining output $y_i$ given that the input is $x_i$, and is called a \textbf{\emph{channel transition probability}}.
\end{frame}

\begin{frame}
\frametitle{ Channel Matrix}

A channel is completely specified by the complete set of transition probabilities. Accordingly, the
channel of Fig. 10-l is ol` ten speciiied by the matrix of transition probabilities [P(YlX)l, given by
%P<yilxi> P<yc|Xt>   !’<ytl¤<i>
%W YIXM = P(yllXz> Plyzlxs)   i’<yiI~¤s> (NUI)
%/’<yi|xm) P<ysI·rm>   P<ytI><ml

The matrix [P(YlX)] is called thc channel matrix. Since each input to the channel results in some
output, each row of the channel matrix niust snm to unity. that is,
%ZP(yylx;) : l lor all 1 (10.12)
 \end{frame}


%---------------------------------------------------------------------------------------------------------------------------------------%