An error-correcting code is an algorithm for expressing a sequence of numbers such that any errors which are introduced can be detected and corrected (within certain limitations) based on the remaining numbers. The study of error-correcting codes and the associated mathematics is known as coding theory.

Error detection is much simpler than error correction, and one or more "check" digits are commonly embedded in credit card numbers in order to detect mistakes. Early space probes like Mariner used a type of error-correcting code called a block code, and more recent space probes use convolution codes. Error-correcting codes are also used in CD players, high speed modems, and cellular phones. Modems use error detection when they compute checksums, which are sums of the digits in a given transmission modulo some number. The ISBN used to identify books also incorporates a check digit.

