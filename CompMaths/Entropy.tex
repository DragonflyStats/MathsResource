\documentclass{beamer}

\usepackage{amsmath}
\usepackage{amssymb}

\begin{document}
	
%=========================================================== %
\begin{frame}
	\large
	\frametitle{Entropy}
	\textbf{Information Entrophy}	

We use logarithms to base 2. The entropy will then be measured in
bits. The entropy is a measure of the average uncertainty in the
random variable. It is the number of bits on average required to
describe the random variable.

\end{frame}
%=========================================================== %
\begin{frame}
	\large
	\frametitle{Entropy}
\textbf{Information Entrophy}
Entropy is the uncertainty of a single random variable.

Information entropy is a measure of the uncertainty associated
with a random variable. The term by itself in this context usually
refers to the Shannon entropy, which quantifies, in the sense of
an expected value, the information contained in a message, usually
in units such as bits.

\end{frame}
%=========================================================== %
\begin{frame}
	\large
	\frametitle{Entropy}
 We can
 define conditional entropy $H(X|Y)$, which is the entropy of a
 random variable conditional on the knowledge of another random
 variable. The reduction in uncertainty due to another random
 variable is called the \textbf{\textit{mutual information}}.

\end{frame}
%=========================================================== %
\begin{frame}
	\large
	\frametitle{Entropy}
The Shannon entropy is denoted by $H(X)$ and is defined as
\begin{equation}
   H(X) =   - \sum_{i=1}^np(x_i)\log_b p(x_i).
\end{equation}

\end{frame}
%=========================================================== %
\begin{frame}
	\large
	\frametitle{Entropy}
\textbf{Example}

A source language has 5 symbols A, B, C, D and E.  The associated
probabilities of these symbols are 0.35, 0.25, 0.20, 0.10 and
0.10, respectively.

\begin{itemize}
\item Calculate the entropy of the source language. \item Define a
Huffman binary code for the source language. \item Calculate the
efficiency of this code.
\end{itemize}
\end{frame}
%=========================================================== %

\end{document}