
\documentclass{beamer}

\usepackage{amsmath}
\usepackage{graphicx}
\usepackage{amssymb}
\begin{document}
%================================================================================ %
\begin{frame} 
	\frametitle{Function Concavity}  
 Example 2  Use the second derivative test to classify the critical points of the function,
 
 
 Solution
 Note that all we’re doing here is verifying the results from the first example.  The second derivative is,
 
\end{frame}
%================================================================================ %
\begin{frame} 
	\frametitle{Function Concavity} 
	\begin{itemize}
 \item The three critical points ( ,  , and  ) of this function are all critical points where the first derivative is zero so we know that we at least have a chance that the Second Derivative Test will work.  
 \item The value of the second derivative for each of these are,
 \end{itemize}
\end{frame}
%================================================================================ %
\begin{frame} 
	\frametitle{Function Concavity} 
 The second derivative at  is negative so by the Second Derivative Test this critical point this is a relative maximum as we saw in the first example.  The second derivative at   is positive and so we have a relative minimum here by the Second Derivative Test as we also saw in the first example.
\end{frame}
%================================================================================ %
\begin{frame} 
	\frametitle{Function Concavity} 
 In the case of  the second derivative is zero and so we can’t use the Second Derivative Test to classify this critical point.  Note however, that we do know from the First Derivative Test we used in the first example that in this case the critical point is not a relative extrema.
\end{frame}
%================================================================================ %
\begin{frame} 
	\frametitle{Function Concavity}  
 Let’s work one more example.
 
 Example 3  For the following function find the inflection points and use the second derivative test, if possible, to classify the critical points.  Also, determine the intervals of increase/decrease and the intervals of concave up/concave down and sketch the graph of the function.
\end{frame}
%================================================================================ %
\begin{frame} 
	\frametitle{Function Concavity}  
 
 Solution
 We’ll need the first and second derivatives to get us started.
 
 The critical points are,
\end{frame}
%================================================================================ %
\begin{frame} 
	\frametitle{Function Concavity}  
 Notice as well that we won’t be able to use the second derivative test on  to classify this critical point since the derivative doesn’t exist at this point.  To classify this we’ll need the increasing/decreasing information that we’ll get to sketch the graph.
\end{frame}
%================================================================================ %
\begin{frame} 
	\frametitle{Function Concavity}  
 We can however, use the Second Derivative Test to classify the other critical point so let’s do that before we proceed with the sketching work.  Here is the value of the second derivative at .
 
 
 So, according to the second derivative test  is a relative maximum.
\end{frame}
%================================================================================ %
\begin{frame} 
	\frametitle{Function Concavity}  
 Now let’s proceed with the work to get the sketch of the graph and notice that once we have the increasing/decreasing information we’ll be able to classify .
 
 Here is the number line for the first derivative.
% % Graphic ShapeOfGraphII_Ex3_G1
\end{frame}
%================================================================================ %
\begin{frame} 
	\frametitle{Function Concavity}  
 So, according to the first derivative test we can verify that  is in fact a relative maximum.  We can also see that  is a relative minimum. 
\end{frame}
%================================================================================ %
\begin{frame} 
	\frametitle{Function Concavity} 
 Be careful not to assume that a critical point that can’t be used in the second derivative test won’t be a relative extrema.  We’ve clearly seen now both with this example and in the discussion after we have the test that just because we can’t use the Second Derivative Test or the Test doesn’t tell us anything about a critical point doesn’t mean that the critical point will not be a relative extrema.  This is a common mistake that many students make so be careful when using the Second Derivative Test.
\end{frame}
%================================================================================ %
\begin{frame} 
	\frametitle{Function Concavity}  
 Okay, let’s finish the problem out.  We will need the list of possible inflection points.  These are,
 
 Here is the number line for the second derivative.  Note that we will need this to see if the two points above are in fact inflection points.
 % % ShapeOfGraphII_Ex3_G2
 
\end{frame}
%================================================================================ %
\begin{frame} 
	\frametitle{Function Concavity} 
 
 So, the concavity only changes at  and so this is the only inflection point for this function.
 
 Here is the sketch of the graph.
% % ShapeOfGraphII_Ex3_G3
 
 The change of concavity at  is hard to see, but it is there it’s just a very subtle change in concavity.
 
\end{frame}
\end{document}