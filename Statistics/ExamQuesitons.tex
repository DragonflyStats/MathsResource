\subsection*{Polynomial}

Absorbance= c(0.084, 0.183, 0.326, 0.464, 0.643, 0.707, 0.717, 0.734 ,0.749 ,0.732) ;
Concentration= c(0.123, 0.288, 0.562, 0.921, 1.420, 1.717, 1.921, 2.137 ,2.321, 2.467) ;
plot(Concentration,Absorbance,pch=18,col="red",font.axis=2,font.lab=2)
abline(coef(lm(Absorbance~Concentration)))

Conc.Squared = (Concentration^2)
Conc.Cubed = (Concentration^3)
ModelA = lm(Absorbance~Concentration)
ModelB = lm(Absorbance~Concentration+Conc.Squared)
ModelC = lm(Absorbance~Concentration+Conc.Squared+Conc.Cubed)

%=========================================================================================== 
%============================================================================%
\subsection{Vegetables (ONE WAY ANOVA 4505)}

(c) The following data give the recovery of bromide from spiked samples of vegetable matter, measured using a gas-liquid chromatographic method. The same amount of bromide was added to each specimen. 

The units for all measurements are  mg g-1

\begin{center}
	\begin{tabular}{|c|c|c|c|}
		Tomato: & $\{777 790 759 790 770 758 764\}$ & 774 & 142.6667
		\\ \hline
		
		Cucumber: & $\{782 773 778 765 789 797 782\}$ & 781 &  106\\ \hline
		
		Asparagus : & $\{786 783 781 785 789 797 782 \}$& 785 & \\ \hline
	\end{tabular} 
\end{center}


-------------------------------------------------------------------------------
y<-c(777, 789, 769, 790, 770, 759, 764, 782, 774, 778, 765, 789, 
797, 782, 785, 783, 782, 785, 787, 791, 782)



T<-y[1:7];C<-y[8:14];A<-y[15:21];
mean(T);mean(C);mean(A);

-------------------------------------------------------------------------------
-------------------------------------------------------------------------------



\begin{itemize}
	\item[(i)](3 Marks) Compute the Between Groups Sum of Squares, \textit{Show your workings}
	\item[(ii)](3 Marks) Compute the Within Groups Sum of Squares, \textit{Show your workings}
	\item[(iii)](2 Marks) Compute the Total Sum of Squares,\textit{ show your workings}
	\item[(iv)] (2 Marks) Degrees of Fredom columns
	\item[(v)] (1 Marks) Mean Square
	\item[(vi)] (1 Marks) F test Statistics
\end{itemize}
\begin{tabular}{|c|c|c|c|c|c|}
	\hline Source & DF & SS & MS & F & p-value \\ 
	\hline Between &  &  &  &  &  \\ 
	\hline Within &  &  &  &  &  \\ 
	\hline Total &  &  &  &  &  \\ 
	\hline 
\end{tabular} 
\begin{framed}
	\begin{verbatim}
	> bartlett.test(y~group)
	
	Bartlett test of homogeneity of variances
	
	data:  y by group
	Bartlett's K-squared = 7.9063, df = 2, p-value = 0.01919
	
	\end{verbatim}
\end{framed}
%===============================================================================================%

\

%=================================================================%
\subsection{Pollutant - TWO WAY ANOVA - MA4605}
Q7 These data are from a statement by Texaco, Inc. to the Air and Water Pollution Subcommittee of the Senate Public Works Committee on June 26, 1973. Mr. John McKinley, President of Texaco, cited the Octel filter, developed by Associated Octel Company as effective in reducing pollution. However, questions had been raised about the effects of pollution filters on aspects of vehicle performance, including noise levels. He referred to data presented associated with this question as evidence that the Octel filter was at least as good as a standard silencer in controlling vehicle noise levels. The data constitute a 3-way factorial experiment with 3 replications.

Small Medium Large

Filter Right 810.00 840.00 785.00

Type 820.00 840.00 790.00

“Standard” 820.00 845.00 785.00

Left 835.00 845.00 760.00

835.00 855.00 760.00

835.00 850.00 770.00

Filter Right 820.00 820.00 775.00

Type 820.00 820.00 775.00

“Octel” 820.00 825.00 775.00

Left 825.00 815.00 770.00

825.00 825.00 760.00

825.00 825.00 765.00

----------------------------------------------------------


%========================================================== %

Main effects plots, interaction plots, and MINITAB output relevant to these data are given below.

(a) Explain what you understand by “the main effect of a factor”. Write down the model corresponding to the ANOVA table below. Explain why the main effect for the factor “Side” can not be dropped from the model despite its high p-value ?

[8 marks]

(b) In your answer sheet draw an interaction plot for the factors “Type” and “Size” when (i) considering only data collected on the “Right Side” of automobiles, (ii) considering only data collected on the “Left Side” of automobiles.

[4 marks]

(c) Using the interaction plots from part (b), interpret the statistically significant three-way interaction found in the ANOVA table.

[4 marks]

(d) Comment on the overall validity of the claim made by John McKinley.

[4 marks]

%=================================================================%
\newpage

%=================================================================%
% %- process.capability(qcc(SPC13+1000,"xbar"),spec.limits=c(1225,1235))

