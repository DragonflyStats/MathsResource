\documentclass{beamer}

\usepackage{amsmath}
\usepackage{amssymb}
\usepackage{graphics}

\begin{document}
\begin{frame}
{
\huge
\[ \mbox{Discrete Mathematics} \]
\[ \mbox{Indices} \]
}
{
\Large
\[ \mbox{www.MathsResource.com} \]
}
\end{frame}
%-----------------------------------------------------------%
\begin{frame}
\frametitle{Indices}
\LARGE
\vspace{-1cm}
\textbf{Rules of Indices}
\begin{enumerate}
\item[1.] $a^m\times a^n=a^{m+n}$ \bigskip  \vspace{1.5cm}
\item[2.] ${a^m\over a^n} =  a^{m-n}$ \bigskip \vspace{1.5cm}
\item[3.] $a^0 = 1$ for all $a$ \textbf{except for 0}.
\end{enumerate}
\end{frame}
%-----------------------------------------------------------%
\begin{frame}
\frametitle{Indices}
\LARGE
\vspace{-1cm}
\textbf{Rules of Indices}
\begin{enumerate}
\item[4.] $(a^m)^n=a^{mn}$ \bigskip \vspace{1.5cm}
\item[5.] $a^{-n} = \frac{1}{a^n}$ $a \neq 0 $ \bigskip \vspace{1.5cm}
\item[6.] $a^{{m \over n}} = \sqrt[n]{a^m}$
\end{enumerate}
\end{frame}
%-----------------------------------------------------------%
\begin{frame}
\frametitle{Indices}
\Large
\vspace{-3cm}
Example 1: $256^{3\over 4}$


\end{frame}
%-----------------------------------------------------------%
\begin{frame}
\frametitle{Indices}
\Large
Example 1: $256^{3\over 4}$

\begin{itemize}
\item $256^{3\over 4}$ = 
$256^{3\over 4}$ 
\item \phantom{$256^{3\over 4}$} = $(4^4)^{3\over 4}$
\item 
\end{itemize}

\end{frame}
%-----------------------------------------------------------%
\begin{frame}
\frametitle{Indices}
\end{frame}
%-----------------------------------------------------------%
\end{document}
