\documentclass[a4]{beamer}
\usepackage{amssymb}
\usepackage{graphicx}
\usepackage{subfigure}
\usepackage{newlfont}
\usepackage{amsmath,amsthm,amsfonts}
%\usepackage{beamerthemesplit}
\usepackage{pgf,pgfarrows,pgfnodes,pgfautomata,pgfheaps,pgfshade}
\usepackage{mathptmx}  % Font Family
\usepackage{helvet}   % Font Family
\usepackage{color}

\mode<presentation> {
 \usetheme{Default} % was Frankfurt
 \useinnertheme{rounded}
 \useoutertheme{infolines}
 \usefonttheme{serif}
 %\usecolortheme{wolverine}
% \usecolortheme{rose}
\usefonttheme{structurebold}
}

\setbeamercovered{dynamic}

\title[MathsCast]{MathsCasts - Dynamic Maths Support) \\ {\normalsize Statistics}}
\author[Kevin O'Brien]{Kevin O'Brien \\ {\scriptsize Kevin.obrien@ul.ie}}
\date{Summer 2011}
\institute[Maths \& Stats]{Dept. of Mathematics \& Statistics, \\ University \textit{of} Limerick}

\renewcommand{\arraystretch}{1.5}

\begin{document}

\frame{
%\frametitle{Formula}
\vspace{1.3cm}
{\Large
\begin{center}
Period of a Trigonometric Function
\end{center}
}
\bigskip
{\large
\begin{center}
kobriendublin.wordpress.com
\end{center}
}

}



\frame{
\frametitle{Formula}
\vspace{-2.3cm}
\Large
When you have a trigonometric function in the form of \emph{cos(kx)} or \emph{sin(kx)}, the period of the function $L$ is computed using the coefficient $k$:
\\\bigskip
\begin{center}
$\mbox{\fontsize{24}{30}\selectfont $L = \frac{2 \pi}{k}$}$
\end{center}
}

%----------------------------------------------------------------------%


\frame{
\frametitle{Example 1}
\vspace{-1.3cm}
\Large
Compute the period of the function $f(x)$, where
\[f(x) = cos(4 \pi x).\]

\textbf{Solution}:
Here the coefficient $k$ is \hspace{2cm}\\
%Here the coefficient $k$ is $4 \pi$.\\\bigskip
The period of the function is therefore :
\begin{center}
\[\hspace{-5.3cm}L = \frac{2 \pi}{k} = \]
\end{center}
}
%----------------------------------------------------------------------%

\frame{
\frametitle{Example 2}
\vspace{-1.3cm}
\Large
Compute the period of the function $f(x)$, where
\[f(x) = 2 sin(0.5 x).\]

\textbf{Solution}\\
Here the coefficient $k$ is $0.5$\\
The period of the function is therefore \[\hspace{-5.3cm}\frac{2 \pi}{k} = \]
}

%------------------------------------------------------------------------%
\frame{
\frametitle{Example 3}
\Large
Compute the period of the function $f(x)$, where
\[f(x) =  sin (\frac{\pi x}{2}).\]

\textbf{Solution}\\
Here the coefficient $k$ is \\
The period of the function is therefore\\ \[\hspace{-5.3cm}\frac{2 \pi}{k} = \]


}

%---The period function of $sin (\frac{\pi x}{2})$ is $4$
%------------------------------------------------------------------------%
\frame{
\frametitle{Example 4}
\Large
Compute the period of the function $f(x)$, where
\[f(x) = sin \left(\frac{ x}{2}\right).\]

\textbf{Solution}\\
Here the coefficient $k$ is \\
The period of the function is therefore\\ \[\hspace{-5.3cm}\frac{2 \pi}{k} = \]
%The period of function $sin (\frac{ x}{2})$ is $4\pi$
}

\end{document}




%----------------------------------------------%
\frame{
\frametitle{Introduction}
When dealing with trigonometric functions, we are often required to compute the period of that function.
}

%--------------------------------------------------------------------------%
\frame{

\frametitle{Example 3}
half period $l$
}
%------------------------------------------------------------------------%
\frame{
\frametitle{Definition}


Consider a trigonometric function

\[ cos ( n \pi x ) \]


\begin{itemize}
\item T$
\item
\end{itemize}
}


\end{document} 