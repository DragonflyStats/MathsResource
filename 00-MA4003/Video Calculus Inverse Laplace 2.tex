


\documentclass{beamer}

\usepackage{amsmath}
\usepackage{amssymb}
\usepackage{graphics}
\usepackage{framed}

\begin{document}


%-----------------------------------------%
\begin{frame}
\Huge
\[\mbox{Calculus For Engineers}\]
\LARGE
\[\mbox{Inverse Laplace Transforms}\]

\Large
\[\mbox{kobriendublin.wordpress.com}\]
\[\mbox{Twitter: @StatsLabDublin}\]
\end{frame}


%-----------------------------------------%
%-----------------------------------------%
% Example 2 [MT 2009 Q4] Find the inverse Laplace transform of G(s)
\begin{frame}
\frametitle{Inverse Laplace Transforms}
\LARGE
\vspace{-2cm}
Find the Inverse Laplace Transform of the following expression
\[ G(s) = \frac{s}{s^2 -2s + 1} \]
\end{frame}
%-----------------------------------------%
\begin{frame} 
\frametitle{Inverse Laplace Transforms}

Factorise the denominator: \[s ^2 - 2s + 1 = (s - 1)^2\]
\end{frame}
%-----------------------------------------%
\begin{frame} 
\frametitle{Inverse Laplace Transforms}
Rewrite G(s) as follows: \[G(s) = \frac{s}{(s-1)^2} \]

Can we use the form $F(s-a)$ with $a=1$
s = (s-1)+1
The expression is now in form $F(s-a)$, with $a = 1$.
\end{frame}
%-----------------------------------------%
\begin{frame} 
\frametitle{Inverse Laplace Transforms}
\[ F(s) = \frac{s+1}{s^2} = \frac{s}{s^2} + \frac{1}{s^2} = \frac{1}{s} + \frac{1}{s^2}\]

\[ f(t) = L^{-1}[\frac{1}{s}] + L^{-1}[\frac{1}{s^2}] = 1 + t\]

\[ L^{-1}[G(s)] = e^{-t}(t+1 ) \]

\end{frame}
%-----------------------------------------%
\end{document}
