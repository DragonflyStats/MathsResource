\documentclass[a4]{beamer}
\usepackage{amssymb}
\usepackage{graphicx}
\usepackage{subfigure}
\usepackage{newlfont}
\usepackage{amsmath,amsthm,amsfonts}
%\usepackage{beamerthemesplit}
\usepackage{pgf,pgfarrows,pgfnodes,pgfautomata,pgfheaps,pgfshade}
\usepackage{mathptmx}  % Font Family
\usepackage{helvet}   % Font Family
\usepackage{color}

\mode<presentation> {
 \usetheme{Default} % was Frankfurt
 \useinnertheme{rounded}
 \useoutertheme{infolines}
 \usefonttheme{serif}
 %\usecolortheme{wolverine}
% \usecolortheme{rose}
\usefonttheme{structurebold}
}

\setbeamercovered{dynamic}

\title[MathsCast]{MathsCasts - Dynamic Maths Support) \\ {\normalsize Engineering Maths 3}}
\author[Kevin O'Brien]{Kevin O'Brien \\ {\scriptsize Kevin.obrien@ul.ie}}
\date{Summer 2011}
\institute[Maths \& Stats]{Dept. of Mathematics \& Statistics, \\ University \textit{of} Limerick}

\renewcommand{\arraystretch}{1.5}

\begin{document}

\begin{frame}
\titlepage
\end{frame}

\section{Fourier Coefficients}
\frame{


}
%------------------------------------------------------------------------%
\frame{
\frametitle{Example 1 }
\vspace{-2cm}
Compute the coefficient $a_0$ in the \emph{Fourier} Series for the periodic function \[f(x) \;=\; x^3 \] if $-1 <x<1$, with period 2.

}
%------------------------------------------------------------------------%


\frame{\frametitle{Is the Function odd or even?}

\begin{itemize}
\item If the function is odd, then the terms $a_0$ and $a_n$ will be zero.
\item If the function is even, then the terms  $b_n$ will be zero.
\end{itemize}

$f(x) = x^3$  is an \textbf{\emph{odd}} function, so necessarily $a_0$ is zero.\\

Lets do it from first principles.





}


%------------------------------------------------------------------------%
\frame{
\frametitle{Example 2 }
\vspace{-2cm}
Compute the coefficient $a_0$ in the \emph{Fourier} Series for the periodic function \[f(x) \;=\; x^4 \] if $-1 <x<1$, with period 2.

}
%------------------------------------------------------------------------%


\frame{\frametitle{Is the Function odd or even?}

\begin{itemize}
\item If the function is odd, then the terms $a_0$ and $a_n$ will be zero.
\item If the function is even, then the terms  $b_n$ will be zero.
\end{itemize}

$f(x) = x^4$  is an \textbf{\emph{even}} function, so necessarily $a_0$ is not going to equal to zero.\\

Lets do it from first principles.
}

%--------------------------------------------------------------------------------%
\frame{

\[ a_o = {1 \over l } \int\limits^{1}_{-1} f(x) dx \]

\[ a_o = {1 \over l } \int\limits^{1}_{-1} x^4 dx  = \left[{x^5 \over 5}\]^{1}_{-1} \right] \[

\[ \left[{x^5 \over 5}\right]^{1}_{-1} = \left[(1)^5 \over 5}\right] - \left[(-1)^5 \over 5}\right] \]

}


\end{document}
