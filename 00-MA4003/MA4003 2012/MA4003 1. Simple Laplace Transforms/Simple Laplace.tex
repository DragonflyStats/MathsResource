
\documentclass[a4]{beamer}
\usepackage{amssymb}
\usepackage{graphicx}
\usepackage{subfigure}
\usepackage{newlfont}
\usepackage{amsmath,amsthm,amsfonts}
%\usepackage{beamerthemesplit}
\usepackage{pgf,pgfarrows,pgfnodes,pgfautomata,pgfheaps,pgfshade}
\usepackage{mathptmx}  % Font Family
\usepackage{helvet}   % Font Family
\usepackage{color}

\mode<presentation> {
 \usetheme{Default} % was Frankfurt
 \useinnertheme{rounded}
 \useoutertheme{infolines}
 \usefonttheme{serif}
 %\usecolortheme{wolverine}
% \usecolortheme{rose}
\usefonttheme{structurebold}
}

\setbeamercovered{dynamic}

\title[MathsCast]{MathsCasts - Dynamic Maths Support) \\ {\normalsize Statistics}}
\author[Kevin O'Brien]{Kevin O'Brien \\ {\scriptsize Kevin.obrien@ul.ie}}
\date{Summer 2011}
\institute[Maths \& Stats]{Dept. of Mathematics \& Statistics, \\ University \textit{of} Limerick}

\renewcommand{\arraystretch}{1.5}

\begin{document}


%-----------------------------------------------------------------------------------------------------------------%
\frame{
\frametitle{Laplace Transform}

\begin{itemize}
\item It turns out that many problems are greatly simplified when converted to the complex frequency domain. 

\item For example, integration and differentiation in the time domain become simple algebraic expressions in the complex frequency domain.

\item The Laplace transform converts a problem between these two domains.

\item This type of calculation can be done entirely in the time domain, but it requires solving differential equations,
which is challenging and time-consuming.

\item The Laplace transform technique is a huge improvement over working directly with differential equations.

\end{itemize}
}
%-----------------------------------------------------------------------------------------------------------------%
\frame{
\frametitle{Laplace Transform}

\begin{itemize}
\item The \emph{Laplace} transform of $f(t)$ is denoted $F(s)$
\item $f(t)$ is a function in the time domain
\item $F(S)$ is the function in the complex frequency domain that corresponds to $F(s)$
\item The \emph{ Laplace} transform operator is denoted $\mathcal{L}$
\end{itemize}

}
%-----------------------------------------------------------------------------------------------------------------%
\frame{
\frametitle{Laplace Transforms}

\begin{itemize}
\item We will use a set of tables of \emph{Laplace} transforms.
\end{itemize}
}

%-----------------------------------------------------------------------------------------------------------------%
\frame{
\frametitle{Laplace Transforms :  Example 1}
\Large
\vspace{-3cm}
Compute the Laplace transforms of the time domain function $ f(t)$  
\[ f(t) = t^4 \]



}
%-----------------------------------------------------------------------------------------------------------------%
\frame{
\frametitle{Laplace Transforms :  Example 1}
\Large
\vspace{-2.2cm}
\begin{table}[ht]
\caption{Laplace Transforms Tables} % title of Table
\centering 
\begin{tabular}{|ccc |ccc| } 
\hline %inserts double horizontal lines
&f(t)& && F(s)& \\  \hline \hline &\ldots&& &\ldots& \\ \hline
&$t^n$ &&& $ {n! \over s^{n+1}}$ &\\ \hline 
 &\ldots&& &\ldots& \\ \hline
\end{tabular}
\end{table}


}

%-----------------------------------------------------------------------------------------------------------------%
\frame{
\frametitle{Laplace Transforms :  Example 2}
\Large
\vspace{-3cm}
Compute the Laplace transforms of the time domain function $ f(t)$  
\[ f(t) = 4\; \mbox{sin}(2t) \]



}
%-----------------------------------------------------------------------------------------------------------------%
\frame{
\frametitle{Laplace Transforms :  Example 2}
\Large
\vspace{-2.2cm}
\begin{table}[ht]
\caption{Laplace Transforms Tables} % title of Table
\centering 
\begin{tabular}{|ccc |ccc| } 
\hline %inserts double horizontal lines
&f(t)& && F(s)& \\  \hline \hline &\ldots&& &\ldots& \\ \hline
&$\mbox{ sin }(at)$ &&& $ {a^2 \over s^2 + a^2}$ &\\ \hline 
 &\ldots&& &\ldots& \\ \hline
\end{tabular}
\end{table}


}
%-----------------------------------------------------------------------------------------------------------------%
\frame{
\frametitle{Laplace Transforms :  Example 3}
\Large
\vspace{-3cm}
Compute the Laplace transforms of the time domain function $ f(t)$  
\[ f(t) = {e^{-t} \over 2} \]



}
%-----------------------------------------------------------------------------------------------------------------%
\frame{
\frametitle{Laplace Transforms :  Example 3}
\Large
\vspace{-2.2cm}
\begin{table}[ht]
\caption{Laplace Transforms Tables} % title of Table
\centering 
\begin{tabular}{|ccc |ccc| } 
\hline %inserts double horizontal lines
&f(t)& && F(s)& \\  \hline \hline &\ldots&& &\ldots& \\ \hline
&$e^{at}$ &&& $ {1 \over s - a}$ &\\ \hline 
 &\ldots&& &\ldots& \\ \hline
\end{tabular}
\end{table}


}
%-----------------------------------------------------------------------------------------------------------------%
\end{document}