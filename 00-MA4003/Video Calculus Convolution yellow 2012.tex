\documentclass{beamer}

\usepackage{graphics}
\usepackage{amsmath}
\usepackage{amssymb}

\begin{document}

%Slide 1 Title Slide
\begin{frame}
\bigskip
{
\Huge
\[ \mbox{Calculus for Engineers}\]
\[ \mbox{Inverse Laplace Transforms}\]
}
{
\Large
\[ \mbox{kobriendublin.wordpress.com}\]
\[ \mbox{Youtube: StatsLabDublin}\]
}
\end{frame}

\begin{frame}
\frametitle{Convolution}
The convolution of $e^t$ with $e^-t$ (also denoted $e^t \ast e^{-t}$)

\begin{itemize}
\item[1] Determine the Laplace transforms of both terms (find $G_1(s)$ and $G_2(s)$).
\item[2] Multiply these terms ( find $G_1(s) \times G_2(s)$.
\item[3] Determine the inverse Laplace transform of that product ( $\mathcal{L}^{-1}$ ($G_1(s) \times G_2(s)) $.
\end{itemize}

\end{frame}
%---------------------------------------------------%
\begin{frame}
\frametitle{Convolution : Step 1}

Determine the Laplace transform of $e^t$ (using tables)
\[
G_1(s) = \mathcal{L}[ e^t ] = \frac{1}{s+1}
\]

Determine the Laplace transform of $e^t$ (using tables)
\[
G_1(s) = \mathcal{L}[ e^{-t} ] = \frac{1}{s-1}
\]
(Note: see Formula sheet entry 6)
\end{frame}

%---------------------------------------------------%


\begin{frame}
\frametitle{Convolution : Step 2}
\Large
Multiply the terms $G_1(s)$ and $G_2(s)$

\[ G_1(s)\cdot G_2(s) = \frac{1}{s+1}\times \frac{1}{s-1} \]

\[ G_1(s)\cdot G_2(s) =  \frac{1}{s^2-1}\]

(Note: see next slide for workings)
\end{frame}

%---------------------------------------------------%
\begin{frame}
\frametitle{Convolution : Step 2 (Workings) }
\LARGE

Using the cross-multiplication technique

\[ \frac{1}{s+1}\times \frac{1}{s-1} = \frac{1}{s+1}\times \frac{1}{s-1} \]


\end{frame}
%---------------------------------------------------%

\begin{frame}
\frametitle{Convolution : Step 3}
Find the inverse laplace transform of $G_1(s) \times G_2(s)$




\end{frame}


\begin{frame}
\frametitle{Convolution : Step 3}
\Large

See Formula Sheet Entry 6.
\[e^t \ast e^{-t} = \mbox{sinh}(t) \]



\end{frame}


%---------------------------------------------------%
\end{document}