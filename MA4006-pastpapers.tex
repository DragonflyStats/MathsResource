Spring 2009 Q5b
 
Use Laplace transforms to solve the heat equation

            ut=c2uxx
 
on the semi-infinite domain
 
         0 x <,14cm  0t <,
 
subject to
(i) ux(0, t) = -1, t > 0;
(ii) u(x, t) -> 0, as x -> 1;
(iii) u(x, 0) = 0, x > 0.

%-------------------------------------------------------%

 

 
Spring 2009 Q6 c
 
Consider the wave equation utt=c2uxx for the semi-infinite string
0x < and t0, with
u(x, 0) =xe-5(x-1)2
ut(x, 0) = 0; u(0, t) = 0:
Formulate the discretised problem using an explicit finite difference method on a uniform mesh (taking central differences to approximate the 2nd derivatives).

%--------------------------------------------------%
 
Spring 2008 Q4d
 
Consider the partial differential equation
                uxx + 3uxt + 2utt = 0 :
Use the transformation = x-t, = x -t2 to reduce the equation to a simpler form and determine the general solution u(x; t).

Hence find the particular solution that satisfies the boundary conditions    

u(x; 0) = x2 ; ut(x; 0) = 0 :


%----------------------------------------------------%
Repeat 2008 Q3a
The vector Field
 
F(x; y; z) =(Ax3y2z ) i +(z3+Bx4yz) j +(3yz2-x4y2) k
 
is conservative.
 
Find values of the constants A and B.
 
Hence find a scalar field  such that F=.

%----------------------------------------------------%


Repeat 2008 Q2a
 
Find the work done by the force field

        F(x; y; z) = xz i - yz k ;

in moving a particle along the line segment from the point (-1; 2; 0) to the point (3; 0; 1).

Solution

%----------------------------------------------------%
 

MA4006 Engineering Mathematics 5 Spring 2007
 
Find the directional derivative of the scalar valued function

f(x; y; z) = 3x2y + z3x -4x
 
at the point (1;¡1; 0) in the direction of 2i ¡ j + k.
