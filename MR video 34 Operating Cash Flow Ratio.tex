\documentclass{beamer}

\usepackage{amsmath}
\usepackage{amssymb}

\begin{document}


\begin{frame}
\Large
\[
\mbox{Continuous Compounding}
\]
\end{frame}

%-------------------------------------------------------%

\begin{frame}
\frametitle{Operating Cash Flow Ratio}

A measure of how well current liabilities are covered by the cash flow generated from a company's operations. 

Formula: 

Operating Cash Flow Ratio
 
\end{frame}

%-------------------------------------------------------%

\begin{frame}
\frametitle{Operating Cash Flow Ratio} 
Investopedia explains 'Operating Cash Flow Ratio'
The operating cash flow ratio can gauge a company's liquidity in the short term. Using cash flow as opposed to income is sometimes a better indication of liquidity simply because, as we know, cash is how bills are normally paid  
\end{frame}



\end{document}

Maths Resource

http://www.csusm.edu/mathlab/documents/M132BusCalcFormulas%20r1-12e.pdf
http://www.math.ubc.ca/~chau/elasticity.pdf
http://www.textbooksonline.tn.nic.in/books/12/std12-bm-em-1.pdf
