\documentclass[]{report}

\voffset=-1.5cm
\oddsidemargin=0.0cm
\textwidth = 480pt

\usepackage{framed}
\usepackage{subfiles}
\usepackage{graphics}
\usepackage{enumerate}
\usepackage{newlfont}
\usepackage{eurosym}
\usepackage{amsmath,amsthm,amsfonts}
\usepackage{amsmath}
\usepackage{color}
\usepackage{amssymb}
\usepackage{multicol}
\usepackage[dvipsnames]{xcolor}
\usepackage{graphicx}
\begin{document}
	

\subsection*{Question 1}

Let's assume that a sequence of bits (binary numbers) is transmitted and, at the other end, decoded; the decoder has a 10\% chance reading a bit incorrectly (i.e., reading a 0 as 1 or vice versa). 
Let $X$ be the number of errors in the sequence received (i.e., the decoded sequence). Calculate the probability that there are: \\[-0.2cm]

\begin{enumerate}[(i)]
\item \emph{No} errors in a 20-bit string.  \item Less than three errors in a 10-bit string.  \item More than 10 errors in a 50-bit string \item More than 10 errors  a 100-bit string (hint: use tables).  \item Calculate the average number of errors in a 100-bit string. Calculate the standard deviation also.
\end{enumerate}

\subsection*{Question 2}

	Commuter trains have a probability 0.1 of delay
	between Dublin and Maynooth. Supposing that the delays are all independent,
	what is the probability that out of 10 journeys between Dublin and
	Mullinar more than 8 do not have a delay.
	\begin{enumerate}[(i)]
		\item Reconsider the question : What is the probability that there is less than 2 delays.
		\item $X$ is the variable for `delays', with Binomial parameters $n=10$, $p=0.1$
		\item $P(X < 2) = P(X \leq 1) = P(X=0)+P(X=1)$
		\item $P(X=0)$
		\[P(X=0)= {10 \choose 0} \times 0.1^0  \times 0.9^10 = 0.34868\]
		\item $P(X=1)$
		\[P(X=1)= {10 \choose 1} \times 0.1^1  \times 0.9^9 = 0.38742\]
		\item $P(X < 2) = = 0.38742 + 0.34868 = 0.73610.$
	\end{enumerate}
	
	
\subsection*{Question 3}
A biased coin yields `Tails' on $48\%$ of throws. Consider an experiment that consists of throwing this coin 11 times.
\begin{itemize}
	\item[(i)] Evaluate the following term $^{11}C_2$.
	\item[(ii] Compute the probability of getting two `Tails' in this experiment.
\end{itemize}

\subsection*{Question 4}
An inspector of computer parts selects a random sample of components
from a large batch to decide whether or not to audit the full batch.

%---------------------------------------------%
\begin{itemize}
	\item[(i)] lf 20\% or more of the sample is defective, the entire batch is
	inspected, Calculate the probability of this happening if it is
	thought that the population contains 4\% defective components and
	a sample of 20 is selected.
	\item[(ii)] lf 10\% or more of the sample is defective, the entire batch is
	inspected. Calculate the probability of this happening if it is
	thought that the population contains 4\% defective components and
	a sample of 50 is selected.
	(10 marks)
\end{itemize}


\subsection*{Question 5}
Under what circumstances is it appropriate to use the binomial distribution when calculating probabilities?					(1 mark)

\subsection*{Question 6}
Flextronics supply PCB boards to Dell.  You are a production manager with Dell.  There is a constant probability of 0.01 that a board will be defective.  You select 20 boards at random.  What is the probability that:
\begin{itemize}
	\item[(i)]	0 boards will be defective
	\item[(ii)]	1 or more boards will be defective
	\item[(iii)]	2 or less boards will be defective			
	(6 marks)
\end{itemize}

\subsection*{Question 7}
Suppose that a student is taking a multiple-choice exam in which each question has four choices.
Suppose that she has no knowledge of the correct answer to any of the questions. Furthermore suppose that she selects one of the possible choices at random as her answer.
\begin{enumerate}[(i)]
\item lf there are five muItiple—choice questions on the exam, what is the probability that she will answer four questions correctly.
\item What is the probability that she will answer none of the questions correctly?
\item What is the probability that she will answer at least two questions correctly?
\end{enumerate}


%===============================================================%


\subsection*{Question 8}
A company owns 400 laptops. Each laptop has a 2\% probability of not working. You randomly select 20 laptops for your salespeople.
\begin{enumerate}[(i)]
\item What distribution do you feel is appropriate here. Justify your answer. 
\item What is the probability that they will all work?
\item What is the probability that only one will be broken?
\item What is the probability that at least two will be broken? 
\end{enumerate}
\end{document}	
