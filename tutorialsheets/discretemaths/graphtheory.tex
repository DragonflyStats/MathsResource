\documentclass[]{article}
\voffset=-1.5cm
\oddsidemargin=0.0cm
\textwidth = 470pt
\usepackage[utf8]{inputenc}
\usepackage[english]{babel}
\usepackage{framed}

\usepackage{multicol}
\usepackage{amsmath}
\usepackage{amssymb}
\usepackage{enumerate}
\usepackage{multicol}




%opening
\title{Binary Arithmetic - Tutorial Sheet}
\author{www.MathsResource.com}

\begin{document}
\section*{Graph Theory - Tutorial Sheet}
\subsection*{Question 1}
\begin{itemize}
\item[(a)] Let G be a graph and let v be a vertex of G. Say what is meant by the degree
of v. % \newline [1 Mark]
\item[(b)] A graph is called k-regular if each of its vertices has degree k. Construct an
example of:
\begin{itemize}
\item[(i)] a 2-regular graph with 5 vertices, % [2 Marks]
\item[(ii)] a 3-regular graph with 6 vertices. % [2 Marks]
\end{itemize}
\item[(c)] State, without proving, a result connecting the degrees of the vertices of
a graph G with the number of its edges. % [1 Mark]
\item[(d)] Use the result in part (c) to find the number of edges of a 3-regular graph with 10
vertices % [2 Marks]
\item[(e)] Explain why it is not possible to construct a 3-regular graph with 9 vertices. % [2 Marks]

\end{itemize}


\subsection*{Question 2}

%\frametitle{Graph Theory}

Draw the graph \textbf{G}, which has the vertices $v_1$,$v_2$,$v_3$,$\ldots$,$v_7$, and the adjacency list:

\begin{itemize}
\item[$v_1$]: $v_2$,$v_4$
\item[$v_2$]: $v_1$,$v_3$
\item[$v_3$]: $v_2$,$v_4$
\item[$v_4$]: $v_1$,$v_3$,$v_5$
\item[$v_5$]: $v_4$,$v_6$
\item[$v_6$]: $v_5$,$v_7$
\item[$v_7$]: $v_5$,$v_6$
\end{itemize}
%Draw the graph \textbf{G}.

\subsection*{Question 3}
Given the following definitions for simple, connected graphs:
\begin{itemize}
\item $K_n$ is a graph on $n$ vertices where each pair of vertices is connected by an edge;
\item $C_n$ is the graph with vertices $v_1, v_2, v_3, \dots, v_n$ and edges $\{v_1,v_2\}, \{v_2,v_3\}, \dots\{v_n, v_1\}$;
\item $W_n$ is the graph obtained from $C_n$ by adding an extra vertex,$v_{n+1}$, and edges
from this to each of the original vertices in $C_n$.
\end{itemize}
(a) Draw $K_4$, $C_4$, and $W_4$. 
%----------------------------------------------------------------%


%----------------------------- %
\subsection*{Question 4}
\begin{enumerate}[(a)]
\item A simple, connected graph has 7 vertices, all having the same degree d.
State the possible values of d and for each value also give the number of edges
in the corresponding graph.
\item  Another simple, connected graph has 6 vertices, all having the same degree, n.\\ Draw such a graph when n = 3 and state the other possible values of n.
\end{enumerate}

\end{document}
