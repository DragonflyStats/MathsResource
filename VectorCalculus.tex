MS4613

MS4613
Syllabus
2008 Q1
2008 Q5
Orthogonal Projection
Conservative vector fields
2007 Q3
Surface Integrals
Vector Calculations

Syllabus

[Linear algebra:] definition of determinant; evaluation of determinants by row and column
expansions; symmetries of determinant; eigenvalues and eigenvectors; characteristic
equation; degenerate eigenvalues.

[Vectorial Mechanics:] vector functions of time; laws of mechanics in vector form;
derivative of dot and cross products; angular momentum and torque in vector form; line
integrals and work; conservation of energy and potential function; applications to planetary
dynamics.

[Vector Calculus:] scalar and vector fields; contour maps, directional derivative and gradient
vector of scalar fields; divergence and curl of vector field; applications in electromagnetism
and fluid mechanics; vector identities; surface and volume integrals; Gauss's and Stoke's
theorems.

[Tensor Algebra and Calculus:] Review of matrix algebra introducing suffix notation;
transformation properties of tensors; symmetric and anti-symmetric tensors, with special
reference to examples from mechanics and electromagnetics;
the Levi-Civita tensor.

2008 Q1
Let (0, 0, 0), (−1, 2,−4) and (2, 6,−3) be the coordinates of three points which define a triangle.
(a) Calculate the lengths of the sides of this triangle. 3%
(b) Calculate any angle you wish from the above triangle. 2%
(c) Use elementary geometry to calculate the area of the triangle. 2%
(d) Use the vector cross product to verify the area calculated in part (c). 3%

2008 Q5
Calculate the unit outer normal vector to the ellipsoidal surface x2a2+y2b2+z2c2= 1
at the point (a/3, b/3, c/3).

In the special case a = b = c where the ellipsoidal surface is a spherical surface, 
calculate the unit outer normal at (x, y, z) and interpret your result. 

Orthogonal Projection
Define what is meant by the orthogonal projection of one line onto another. 
If OA has direction cosines l,m,n and P has coordinates (x, y,z) then prove that the orthogonal projection of OP on OA is
lx + my + nz.
Conservative vector fields
Define what is meant by a conservative vector field and prove that every irrotational vector is conservative.


2007 Q3
(a) Define what is meant by arc-length of a differentiable curve. 
(b) Define what is meant by the curvature (s) of a curve at distance s along the curve. 
(c) Compute the curvature at a general point of the parabola: r(x) = (x,x2)
(d) At which point on the above parabola the curvature is maximum.

(e) What happens to the curvature of the parabola as x? Interpret your answer.

Surface Integrals
Use the divergence theorem to write the surface integral
su. dS
as a volume integral if S is the spherex2+y2+z2=a2, and u=(x3,y3,z3)

Vector Calculations
Calculate the vector



i+j+k


x

