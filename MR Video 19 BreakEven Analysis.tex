\documentclass{beamer}

\usepackage{amsmath}
\usepackage{amssymb}

\begin{document}


\begin{frame}
\Huge
\[\mbox{Financial Mathematics}\]
\LARGE
\[\mbox{Break Even Analysis}\]

\Large
\[\mbox{www.stats-lab.com}\]
\[\mbox{Twitter: @StatsLabDublin}\]

\end{frame}


%------------------------------------------------%
\begin{frame}
\frametitle{Break Even Analysis}
\Large
In the \textit{\textbf{Cost-Volume-Profit}} Analysis model, the break-even point (in terms of Unit Sales (Q)) can be directly computed in terms of \textit{\textbf{Total Revenue}} (TR) and \textit{\textbf{Total Costs}} (TC) as:
\begin{eqnarray}
\text{TR} &=  \text{TC}\\
\text{P}\times \text{Q} &= \text{TFC} + \text{V} \times \text{Q}\\
\text{P}\times \text{Q} - \text{V} \times \text{Q} &= \text{TFC}\\
\left(\text{P} - \text{V}\right)\times \text{Q} &= \text{TFC}\\
\text{X} &= \frac{\text{TFC}}{\text{P} - \text{V}}
\end{eqnarray}
where:
\begin{itemize}
\item TFC is Total Fixed Costs,
\item P is Unit Sale Price, 
\item V is Unit Variable Cost.
\end{itemize}

\end{frame}

%------------------------------------------------%
\begin{frame}
\frametitle{Break Even Analysis}

The Break-Even Point can alternatively be computed as the point where Contribution equals Fixed Costs.
The quantity, $\left(\text{P} - \text{V}\right)$, is of interest in its own right, and is called the Unit Contribution Margin (C): it is the marginal profit per unit, or alternatively the portion of each sale that contributes to Fixed Costs. 
\end{frame}

%------------------------------------------------%
\begin{frame}
\frametitle{Break Even Analysis}

\end{frame}



\end{document}

Maths Resource

http://www.csusm.edu/mathlab/documents/M132BusCalcFormulas%20r1-12e.pdf
http://www.math.ubc.ca/~chau/elasticity.pdf
http://www.textbooksonline.tn.nic.in/books/12/std12-bm-em-1.pdf
