\documentclass[12pt]{article}

\begin{document}



\section{Integration}
\begin{abstract}
Business Maths : Section 103
\end{abstract}
\begin{itemize}
%-----------------------------------------------------------------%
\item Integration is an important concept in mathematics and, together with its inverse, differentiation, is one of the two main operations in calculus. Given a function f of a real variable x and an interval [a, b] of the real line, the definite integral
\[\int_a^b \! f(x)\,dx\]
is defined informally to be the signed area of the region in the xy-plane bounded by the graph of f, the x-axis, and the vertical lines x = a and x = b, such that area above the x-axis adds to the total, and that below the x-axis subtracts from the total.
%-----------------------------------------------------------------%
\item The term integral may also refer to the related notion of the antiderivative, a function F whose derivative is the given function f. In this case, it is called an indefinite integral and is written:
\[F(x) = \int f(x)\,dx.\]
However, the integrals discussed in this article are termed definite integrals.
%-----------------------------------------------------------------%
\item The principles of integration were formulated independently by Isaac Newton and Gottfried Leibniz in the late 17th century. Through the fundamental theorem of calculus, which they independently developed, integration is connected with differentiation: if f is a continuous real-valued function defined on a closed interval [a, b], then, once an antiderivative F of f is known, the definite integral of f over that interval is given by
\[\int_a^b \! f(x)\,dx = F(b) - F(a).\]
%-----------------------------------------------------------------%
\end{itemize}
\end{document}
