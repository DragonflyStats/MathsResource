
\documentclass[12pt]{article}
%\usepackage[final]{pdfpages}

\usepackage{graphicx}
\graphicspath{{/Users/kevinhayes/Documents/teaching/images/}}

\usepackage{tikz}
\usetikzlibrary{arrows}

\newcommand{\bbr}{\Bbb{R}}
\newcommand{\zn}{\Bbb{Z}^n}

%\usepackage{epsfig}
%\usepackage{subfigure}
\usepackage{amscd}
\usepackage{amssymb}
\usepackage{amsbsy}
\usepackage{amsthm}
\usepackage{natbib}
\usepackage{amsbsy}
\usepackage{enumerate}
\usepackage{amsmath}
\usepackage{eurosym}
%\usepackage{beamerarticle}
\usepackage{txfonts}
\usepackage{fancyvrb}
\usepackage{fancyhdr}
\usepackage{natbib}
\bibliographystyle{chicago}

\usepackage{vmargin}
% left top textwidth textheight headheight
% headsep footheight footskip
\setmargins{2.0cm}{2.5cm}{16 cm}{22cm}{0.5cm}{0cm}{1cm}{1cm}
\renewcommand{\baselinestretch}{1.3}


\pagenumbering{arabic}

\begin{document}

\subsection{1.1.4 The hexadecimal number system}
\begin{itemize}
    \item Another number system commonly used in computer science is the hexadecimal system based
on 16. 
\item This system has the following advantages: it enables most numbers to be recorded using
substantially fewer digits than are necessary in base 2; there is a very easy method of converting
between hexadecimal and binary, which as we have seen is a “natural” language for computers to
use.
\item In order to express a number in the hexadecimal system we need 16 digits. it is customary to
use the digits 0,1,...,9 followed by the letters A,B,C,D,E, and F as symbols to represent the
numbers 10, 11, 12,13,141 and 15 respectively. 
\item The hexadecimal digits are known as hexits and
the hexadecimal system is often called hex, for short.
\item We convert from hex to decimal in the usual way, using expanded notation in hexits.
\end{itemize}

Example 1.14 We first illustrate the number (5AE7)1@ using hexadecimal boxes.
163 162 161 1
Writing it in expanded notation, we have
(5AE7)16 =.- 5(163) +10(1s2) + 14(1s1) + 7(1)
= 23271. U
Arithmetic in hex
Hexadecimal arithmetic is very similar to decimal arithmetic, except that we have 16 hexits and
carry, or borrow, l6’s instead of 1O’s. l/Ve work two illustrative examples here.
Example 1.15 We add (9D7)15 + (4BA)15. Hexits carried into the next column are entered in
the row above the horizontal line.
%========================%
Thus (9D7]15 + (4BA)16 : (E91)15. The decimal equivalent is 2519 +1210 = 3729. D
Example 1.16 We subtract (F04)15 — (CD9)1@. The revised tow when we have borrowed a 16 is
shown in the row above the horizontal line. Note that all the numbers in the table are in hex, so
that the “I4” in the revised row is (l4)15, representing the decimal number 20.
:oO"1‘lCl3
i\:Uo*'1'.1
mum;


CHAPTER. 1. NUMBERS SYSTEMS 5
Adding the 22 column, we obtain (1)2 and so this time there is nothing to carry. Adding the
23 column, we obtain (11)-3 and so we record 1 and carry 1 into the 24 column. Adding the
2" column, we obtain (1)9 and the process terminates,
Thus we obtain (1011)-3-:-(1111);-1-(11); = (11101); In decimal numbers, this addition is 11+15+3,
and the sum we obtained is 29. El
Binary subtraction
Let us first revise how We deal with “borrowing” in base 10.
\subsection{Example 1.6} We perform the subtraction 4003 —~ 597.

%====================%
Since 7 is greater than 3, we must borrow a 10. In this case, we do this from the 103 column. We
rewrite the borrowed 1000 as 990 + 10. So we reduce the entry in the 103 column from 4 to 3,
replace the zeros by 9’s in the next two columns and replace 3 by 10‘+3 in the units column. The
resulting row is recorded above the horizontal line. We can now obtain the answer by subtracting
597 from the row above the horizontal line. U
We shall now carry out the same process in binary, where we have to borrow a. 2 when we have to
perform the subtraction (0)2 ~ (1)2 in any column. The (0)3 then becomes (10);, and we have
(10)2 ~ (1)2 = (1)2-
This process is illustrated in the example below.
\subsection{Exaniple 1.7} ll/e subtract (110l000)9 — (l01011)2.
\begin{itemize}
    \item \textbf{Step 1}
1 1 0 0 1 l 10
1 1 0 1 0 O O
1 0 1 0 1 1
As in the previous example, we have to borrow to perform the subtraction in the units
column. The first column with a non-zero entry is the 23 column. We borrow (1000); from
this column and rewrite it as (110); + (10); (effectively, we are borrowing an S and rewriting
it as 6 + 2). The revised row is recorded above the horizontal line.
\item Step 2 We replace the row representing the number we are subtracting from by the row above
the horizontal line at the end of Step 1. We can now perform the subtraction in the first
three columns as shown below. When we reach the 23 column; we have to borrow again. We
repeat the procedure described in Step 1. The revised top row is shown above the horizontal
line.
1 0 1 10
1 1 0 0 1 1 10
101011
101
\item  Step 3 We replace the row representing the number we are subtracting from by the row above
the horizontal line and perform the subtraction in the 23 and 24 columns. When we reach
the 25 column, we have to borrow from the next column. The revised top row is shown above
the horizontal line‘
0 10
1 0 1 10 1 1 ,10
1 0 1 0 1 1
11101




\end{itemize}





\end{document}
