
\documentclass[]{article}
\voffset=-1.5cm
\oddsidemargin=0.0cm
\textwidth = 470pt
\usepackage[utf8]{inputenc}
\usepackage[english]{babel}
\usepackage{framed}

\usepackage{multicol}
\usepackage{amsmath}
\usepackage{amssymb}
\usepackage{enumerate}
\begin{document}

% % Part A (4 Marks)
% 1 Floor and Ceiling Functions
% 2 Evaluate Function for two values
% 2 Cross Multiplication

% % PArt B Limits
% Simple one
% Divide by Zero Problem
% Divide by Infinity Problem

% Part C sequences and Series
% 2 Marks Sum of integers
% 3 Geometric Sequence

\Large
\section*{Examinable Material Checklist}

\textbf{Important:} Past Paper short questions and lecture note examples are indicative of what is going to be asked.

\begin{itemize}
	\item Evaluation of Functions
	\item Cross Multiplication
	\item Floor and Ceiliing Functions
	\item Exponentials and Logarithms
	\begin{itemize}
		\item Solving Equations with Exponentials and Logarithms terms (See Tutorial Sheet Question 8)
	\end{itemize}
	\item Difference of Two Squares and Two Cubes
	\begin{itemize}
		\item Formula for Difference of Two Cubes in Formula
	\end{itemize}
	\item Evaluating Limits ( Conventional Evaluation )
	\item Evaluating Limits ( Indeterminate Answers - Factorize and Simplify )
	\item Evaluating Limits ( Indeterminate Answers - Divide by Highest power )
	\item Piece Wise Functions
	\begin{itemize}
		\item Evaluating Functions from the Left 
		\item Evaluating Function from the Right
		\item Are Functions Continuous?
	\end{itemize}
	\item \textit{(Trigonometric Functions are not examinable for the mid-term)}
	\item Arithmetic Sequences - Determining the $n-th$ term
	\item Arithmetic Series - Summation of $n$ terms
	\item Geometric Sequences - Determining the common ratio $r$
	\item Geometric Sequences - Determining the $n-th$ term
	\item Geometric Series - Summation
	\item Summation of Integers (See Tutorial Sheet Question 7)
	\item Partition of Summations (See Tutorial Sheet Question 7)
	\item Sum to Infinity of a Geometric Series
	\item Repeating Decimals - Expressing them as a fraction
	\item Telescoping Series
\end{itemize}

\newpage
\section*{Examination Procedures}

\begin{itemize} 
\item This exam will start at 11:05, and will last 45 minutes.

\item Each Question will be worth Either 1 or 2 Marks. There are 15 Marks worth of questions.
\item All Questions must be attempted (LENS students please see below)

\item Write all of your answers in the exam script. Write the script number on any other documents you submit.

\item It is your responsibility to return the script to collection box. An audit of scripts will take place immediately after the exam. If your script is account for in that audit,  you are deemed to be absent, and will receive no marks.

\item \textbf{LENS Student}
Specifically approved LENS students have to answer any selection of questions that have an aggregate mark of 12 Marks.  
\begin{itemize}
\item They may skip any three 1-Mark Questions
\item OR - They may skip a 1-Mark Question and a 2-Mark Question
\item The mark will be rescaled by 125 \%.
\end{itemize}


\end{itemize}

%============================================================================ %

\newpage


\LARGE
\section*{Formula Sheet}

\subsection*{Sum and Difference of Two Cubes}
\[ a^3 + b^3 = (a-b)(a^2 - ab + b^2)\]
\[ a^3 - b^3 = (a-b)(a^2 + ab + b^2)\]
\subsection*{Sequences and Series}
\[ \sum_{i=1}^{n} i = \frac{n(n+1)}{2}\]

\[ S_n = \frac{n}{2} \left(2a + (n-1) d \right)\]

\[ S_n = a\left(\frac{1-r^n}{1-r}\right)\]

\[ S_\infty = \frac{a}{1-r}\]
\newpage

\section*{Sample Paper 1 - Finished }

%================================================================================== %
\subsection*{Part A Fundamentals of Mathematics (5 Marks) } %1AREADY

\begin{itemize}
%========================================================%

	\item[(i)] (2 Marks) Determine the values of A and B from the following expression
 \[  \frac{2x + 1}{x^2 - 4x + 3} = \frac{A}{x-3} + \frac{B}{x-1}\]
% %	\item[(ii)] (2 Marks) Determine whether or not the function \[f(x) = x cos(x)\] is odd, even or neither.


	\item[(ii)] (2 Marks) Evaluate the function for the values of  $ x = \{-1, 0.5 , 1 \}$

\[  f(x) = \frac{e^x + e^{-x} }{2} \]


\item[(iii)] (1 Mark )  Compute the floor and ceiling functions for $x= -1.25$.
\end{itemize}
%======================================================== %
%======================================================== %
\subsection*{Part B Exponentials and Logarithms (2 Marks)}

\begin{itemize}
	
	
	\item[(i)] (1 Mark) Find the value of $x$
	\[e^{2x-5} = 3. \]
	
	
	
	\item [(ii)](1 Mark) Find the value of $x$
	
	\[log_2(x + 1) + log_2(5) = 3\]
	
\end{itemize}
\newpage
%======================================================== %
\subsection*{Part C Limits of Functions (3 Marks)}

\begin{enumerate}
	\item[(i)]  Compute the limit of the following function
	
	\[\lim_{x \to 3 } \frac{x^2-12}{x+3}\]
	
	\item[(ii)]  Compute the limit of the following function
	
	\[\lim_{x \to 3 } \frac{x^2+x-12}{x-3}\]
	
	\item[(iii)]  Compute the limit of the following function
	\[ \lim_{x \to \infty } \frac{2x^2 - 8x }{4x^2 - 7} \]	

\end{enumerate}
\newpage

\subsection*{Part D Sequences and Series ( 5 Marks)}

\begin{enumerate}
	\item[(i)](1 Mark)  Compute the following summation
	
	\[ \sum_{i=1}^{75} i \]
	
	\item[(ii)](1 Mark) Compute the following summation
	
	\[ \sum_{i=33}^{75} i \]
	
	
	\item[(iii)](1 Mark) Determine the common ratio $r$ of the following geometric sequence. 
	
	
	\[ 6, 2, 0.666 , 0.222\ldots \]
	
	\item[(iv)](1 Mark) Compute the sum to infinity of the following geometric series
	
	\[ 6 + 2 + 0.666 + 0.22 +\ldots \]
	
	\item[(v)] (1 Mark) Express the following repeating decimal number as fractions.
	
	\[0.2424242424....\]
\end{enumerate}


\newpage

\section*{Sample Paper 2}
%======================================================== %
%================================================================================== %
\subsection*{Part A Fundamentals of Mathematics (5 Marks) } %1AREADY

\begin{itemize}
	\item[(i)] (2 Marks) Determine the values of A and B from the following expression
	\[  \frac{3}{(3n+1)(3n+4)} = \frac{A}{n+1} + \frac{B}{3n+4}\]
	
	% \item[(ii)] (2 Marks) Determine whether or not the function \[f(x) = x^3 + x^5 \] is odd, even or neither.
	
	\item[(ii)] (2 Marks) Evaluate the function for the values of x $=\{ -1, 0.5 , 1 \}$ % GOOD
	\[  f(x) = \frac{e^x - e^{-x} }{2} \]
	
	
	\item[(iii)] (1 Mark )  Compute the floor and ceiling functions for $x= -1.25$. % GOOD
\end{itemize}
\newpage
%================================================================================== %
\subsection*{Part B Limits of Functions (3 Marks)}

\begin{itemize}
	\item[(i)] (1 Mark)  Compute the limit of the following function  % GOOD
	\[ \lim\limits_{x \rightarrow 8} \frac{x^2-9}{11}\]
	
	\item[(ii)] (1 Mark)  Compute the limit of the following function % GOOD
	
	\[\lim_{x \to 3 } \frac{x^2-9}{x-3}\]
	
	\item[(iii)] (1 Mark)  Compute the limit of the following function % GOOD
	\[ \lim_{x \to \infty } \frac{6x^2 - 9x }{3x^2 - 7x^3} \]	
	
\end{itemize}
\newpage

\subsection*{Part C Sequences and Series (7 Marks)}

\begin{itemize}
	\item[(i)] (1 Mark)  Compute the following summation
	
	\[ \sum_{i=1}^{45} i \]
	
	\item[(ii)] (1 Mark) Find the sum of the following geometric series: 
\[3 + 6 + 12 + 24 + \ldots + 1536\]
	
	\item[(iii)] (1 Mark) Determine the common ratio $r$ of the following geometric sequence. 
	
	\[ 5, 1,0.2,0.04\ldots \]
	
	\item[(iv)] (1 Mark) Compute the sum to infinity of the following geometric series
	
	\[ 5 + 1 + 0.2 + 0.04\ldots \]
	
	\item[(v)] (2 Marks) Find the sum of the following telescoping series
	\[  \sum^{\infty}_{n=1}   \frac{3}{(3n+1)(3n+4)}  \]
	
	\item[(vi)] (1 Mark) Express the following repeating decimal as a fraction
	\[0.4545454545.....\]
\end{itemize}


%========================================================= %
\newpage 
\section*{Sample Paper 3}
%================================================================================== %
\subsection*{Part A Fundamentals of Mathematics, Logarithms and Exponentials (5 Marks) } %3AREADY
\begin{enumerate}

	\item[(i)] (2 Marks) Determine the values of A and B from the following expression
\[  \frac{5}{x^2 - 1} = \frac{A}{x-1} + \frac{B}{x+1}\]

%	\item[(ii)] (2 Marks) Determine whether or not the function $f(x) = e^x cos(x)$ is odd, even or neither.

%	\item[(ii)] (2 Marks) Evaluate the function for the values of x = $ \{-1, 0.5 , 1 \}$
% \[ f(x) = x sin(x) \]

\item[(ii)] (1 Mark )  Compute the floor and ceiling functions for $x= 1.25$. % GOOD


	
	
	\item[(iii)] (1 Mark) Find the value of $x$
	\[ln(e^x+2) = 4\]
	
	
	
	\item[(iv)] (1 Mark) Find the value of $x$
	
	\[log_3(2x - 1) + log_3(5) = 3\]
	
\end{enumerate}

\newpage
%================================================================================== %
\subsection*{Part B Limits of Functions (3 Marks)}

\begin{itemize}
	\item[(i)] (1 Mark)  Compute the limit of the following function  % GOOD
	\[ \lim\limits_{x \rightarrow 7} \frac{x^2+3}{11}\]
	
	\item[(ii)] (1 Mark)  Compute the limit of the following function % GOOD
	
	\[\lim_{x \to 4 } \frac{x^2-9x-20}{x-4}\]
	
	\item[(iii)] (1 Mark)  Compute the limit of the following function % GOOD
	\[ \lim_{x \to \infty } \frac{6x^2 - 9x^3 }{3x^3 - 7x} \]	
\end{itemize}
%================================================================================== %
\subsection*{Part C Piece Wise Functions (4 Marks)}
Consider the following piecewise function:
\[
f(x) = 
\bigg \{ 
\begin{array}{cc}
(x-3)^2 &  x <3 \\ 
x-2 & x \geq 3
\end{array} 
\]

\begin{itemize}
\item[(i)] (1 Mark) Evaluate the following limit \[\lim_{x \to 3^{-} } f(x) \]
\item[(ii)] (1 Mark) Evaluate the following limit \[\lim_{x \to 3^{+} } f(x) \]
\item[(iii)] (2 Marks) Is $f(x)$
continuous at $x=3$? Justify your answer.
\end{itemize}

	
\newpage


\subsection*{Part D Sequences and Series (3 Marks)}

\begin{enumerate}
	\item[(i)](1 Mark)  Compute the following summation
	
	\[ \sum_{i=1}^{65} i \]
	
%		\item [(ii)](1 Mark) Compute the following summation
%		
%		\[ \sum_{i=37}^{65} i \]
%		
%		\item[(iii)](1 Mark) What is the 15th term of the following geometric sequence
%		
%		\[ 6 , 12 , 24 , 48 ,\ldots \] 
%		\item[(iv)] (1 Mark)Compute the sum of the first 15 terms of of the following geometric series
%		
%		\[ 6 + 12 + 24 + 48 +\ldots \]
		\item[(ii)] (2 Marks)
		In an arithmetic sequence, three consecutive terms have a sum of 15 and a product of 80.
		Find the second of the three terms, and common difference $d$ for these terms.
		\\
		\textit{Hint : Write terms as $x-d,x,x+d$  - the second term is $x$}
\end{enumerate}
%======================================================== %
%\begin{enumerate}
%
%	\item (2 Marks) CUT Find the domain and the range of the function: 
%\[f(x) = 7 + 2 sin x \]
%
%	\item (2 Marks) CUT Find the x and y intercepts of f(x).
%
%\end{enumerate}

\newpage

\begin{enumerate}[(i)]
\item (2 Marks) Determine if the function $f(x) = x^3sin(x)$ is an even function, an odd function or neither.

\item (1 Mark) Given $f(x) = sqrt{2x-8}$, find $f(2x^2+4)$ and simplify the answer.

\item (2 Marks) Find $g^{-1} (x)$ the inverse of the function $g(x) = e^{3x}$

%=========================================================%

\item (2 Marks) Given the functions $g(x) = x^2+1$ and $f(x) = (x-1)/2$ determine the values of $f \circ g(1)$ and $g \circ f(x1)$

\item (2 Marks) Find the Maclaurin Series of $e^{x}$ up to and including the term containing $x^6$
Use this answer to evaluate $e^{-0.4}$ and $e^{3}$

%=========================================================%

\item (2 Marks) Find the domain and the range of the function:
\[f(x) = 7 + 2 sin (x)\]

\item 
(1 Mark) Consider the function $f(x) = x^2 -8x  + 7$. Find the y intercept of the function f(x). 

\end{enumerate}
%======================================================== %
\end{document}
