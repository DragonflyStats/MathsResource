\documentclass{article}
\usepackage[utf8]{inputenc}
\usepackage{enumerate}
\title{MatrixTutorial1}
\author{kobriendublin }
\date{October 2017}

\begin{document}


\section{Cert FM  April 2005 Q 9}
Cert FM  April 2005 Q 9

% April 2005 Question 9
% September 2005 Question 4
% September 2005 Question 5
% Extra Material

\subsection*{April 2005 Question 9}
The one-year forward rate of interest at time t = 1 year is 5% per annum effective.

The gross redemption yield of a two-year fixed interest stock issued at time t = 0 which pays coupons of 3% per annum annually in arrear and is redeemed at 102 is 5.5% per annum effective.

The issue price at time t = 0 of a three-year fixed interest stock bearing coupons of  10% per annum payable annually in arrear and redeemed at par is £108.9 per £100 
nominal.

(i) Calculate the one-year spot rate per annum effective at time t = 0. [4]

(ii) Calculate the one-year forward rate per annum effective at time t = 2 years.[3]

(iii) Calculate the two-year par yield at time t = 0. [3]



From two year stock information:

\begin{eqnarray}
Price &=& 3a_{-2} + 102v^2 at 5.5\%\\
&=& (3 \times 1.84632) + (102 \times 0.89845)\\
&=& 97.181\\
\end{eqnarray}


\[97.181 = \frac{3}{1+ i}+ \frac{3 + 102}{(1+ i)(1+f_{(1,1))}}\]

\[97.1811 =\frac{103}{1+i_1}\]

$i_1= 5.9877 \%$
%-----------------------------------------%
\subsection{September 2005 Question 4}

\begin{itemize}
\item The force of interest $\delta(t)$ at time t is $a +bt^2$ 
where a and b are constants. 
\item 
An amount of £200 invested at time t = 0 accumulates to £210 
at time t = 5 and £230 at time t = 10. 
\item 
Determine a and b. [5]
\end{itemize}


    

    

ln(1.05) = 5a + 41.667b
ln(1.15) = 10a + 333.333b

a= 0.008352

b = 0.000168

%%%%%%%%%%%%%%%%%%%%%%%%%%%%%%%%%%%%%%%%%%%%%%%%%%%%%%%%%%%%




\subsection*{September 2005 Question 5}

Calculate the present value of  £100 over ten years at the following rates of 
interest/discount:
(a)  a rate of interest of 5\% per annum convertible monthly
(b)  a rate of discount of 5\% per annum convertible monthly
(c)  a force of interest of 5\% per annum

(ii) A 91-day treasury bill is bought for $98.91 and is redeemed at $100.  
Calculate the annual effective rate of interest obtained from the bill.  [3]

\begin{enumerate}[(a)]
\item  $100 \times (1 +0.05/12)^{-12/10}= 60.716 $
\item  $100 \times (1 -0.05/12)^{-12/10}= 60.590 $

\item  $100 \times e^{-10\delta}= 60.6531$
\end{enumerate}

%--------------------%

(ii) 98.91 = 100(1 +i)^{-91/365}

\[ ln(1 + i) = \frac{-365}{91} \times ln \frac{98.91}{100} = 0.04396\]

therefore 
            i = 0.04494





\end{document}
