Statement of the equation[edit]
Note: u(x,y,z,t) is not velocity. It is an arbitrary function being considered; often it is temperature.
For a function u(x,y,z,t) of three spatial variables (x,y,z) (see cartesian coordinates) and the time variable t, the heat equation is
\frac{\partial u}{\partial t} -\alpha\left(\frac{\partial^2u}{\partial x^2}+\frac{\partial^2u}{\partial y^2}+\frac{\partial^2u}{\partial z^2}\right)=0
More generally in any coordinate system:


\frac{\partial u}{\partial t} - \alpha \nabla^2 u=0 
 

where α is a positive constant, and Δ or ∇2 denotes the Laplace operator. In the physical problem of temperature variation, u(x,y,z,t) is the temperature and α is the thermal diffusivity. For the mathematical treatment it is sufficient to consider the case α = 1.

The heat equation is of fundamental importance in diverse scientific fields. In mathematics, it is the prototypical parabolic partial differential equation. In probability theory, the heat equation is connected with the study of Brownian motion via the Fokker–Planck equation. In financial mathematics it is used to solve the Black–Scholes partial differential equation. The diffusion equation, a more general version of the heat equation, arises in connection with the study of chemical diffusion and other related processes.
