In mathematics, a partial differential equation (PDE) is a differential equation that contains unknown multivariable functions and their partial derivatives. (This is in contrast to ordinary differential equations, which deal with functions of a single variable and their derivatives.) PDEs are used to formulate problems involving functions of several variables, and are either solved by hand, or used to create a relevant computer model.

PDEs can be used to describe a wide variety of phenomena such as sound, heat, electrostatics, electrodynamics, fluid flow, elasticity, or quantum mechanics. These seemingly distinct physical phenomena can be formalised similarly in terms of PDEs. Just as ordinary differential equations often model one-dimensional dynamical systems, partial differential equations often model multidimensional systems. PDEs find their generalisation in stochastic partial differential equations.


Partial differential equations (PDEs) are equations that involve rates of change with respect to continuous variables. The position of a rigid body is specified by six numbers, but the configuration of a fluid is given by the continuous distribution of several parameters, such as the temperature, pressure, and so forth. The dynamics for the rigid body take place in a finite-dimensional configuration space; the dynamics for the fluid occur in an infinite-dimensional configuration space. This distinction usually makes PDEs much harder to solve than ordinary differential equations (ODEs), but here again there will be simple solutions for linear problems. Classic domains where PDEs are used include acoustics, fluid flow, electrodynamics, and heat transfer.

A partial differential equation (PDE) for the function u(x_1, \cdots, x_n) is an equation of the form
F \left (x_1, \ldots, x_n, u, \frac{\partial u}{\partial x_1}, \ldots, \frac{\partial u}{\partial x_n}, \frac{\partial^2 u}{\partial x_1 \partial x_1}, \ldots, \frac{\partial^2 u}{\partial x_1 \partial x_n}, \ldots \right) = 0.
If F is a linear function of u and its derivatives, then the PDE is called linear. Common examples of linear PDEs include the heat equation, the wave equation, Laplace's equation, Helmholtz equation, Klein–Gordon equation, and Poisson's equation.

A relatively simple PDE is
\frac{\partial u}{\partial x}(x,y) = 0.~
This relation implies that the function u(x,y) is independent of x. However, the equation gives no information on the function's dependence on the variable y. Hence the general solution of this equation is
u(x,y) = f(y),
where f is an arbitrary function of y. The analogous ordinary differential equation is
\frac{\mathrm{d} u}{\mathrm{d} x}(x) = 0,
which has the solution
u(x) = c,
where c is any constant value. These two examples illustrate that general solutions of ordinary differential equations (ODEs) involve arbitrary constants, but solutions of PDEs involve arbitrary functions. A solution of a PDE is generally not unique; additional conditions must generally be specified on the boundary of the region where the solution is defined. For instance, in the simple example above, the function f(y) can be determined if u is specified on the line x = 0.
