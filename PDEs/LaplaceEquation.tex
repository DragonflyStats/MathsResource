Laplace's equation is a second-order partial differential equation named after Pierre-Simon Laplace who first studied its properties. This is often written as:
\nabla^2 \varphi = 0 \qquad\mbox{or}\qquad \Delta\varphi = 0
where ∆ = ∇2 is the Laplace operator and φ is a scalar function.

Laplace's equation and Poisson's equation are the most simple examples of elliptic partial differential equations. The general theory of solutions to Laplace's equation is known as potential theory. The solutions of Laplace's equation are the harmonic functions, which are important in many fields of science, notably the fields of electromagnetism, astronomy, and fluid dynamics, because they can be used to accurately describe the behavior of electric, gravitational, and fluid potentials. In the study of heat conduction, the Laplace equation is the steady-state heat equation.
