Introduction[edit]

The wave equation is a hyperbolic partial differential equation. It typically concerns a time variable t, one or more spatial variables x1, x2, …, xn, and a scalar function u = u (x1, x2, …, xn; t), whose values could model the displacement of a wave. The wave equation for u is
{ \partial^2 u \over \partial t^2 } = c^2 \nabla^2 u 
where ∇2 is the (spatial) Laplacian and where c is a fixed constant.

Solutions of this equation that are initially zero outside some restricted region propagate out from the region at a fixed speed in all spatial directions, as do physical waves from a localized disturbance; the constant c is identified with the propagation speed of the wave. This equation is linear, as the sum of any two solutions is again a solution: in physics this property is called the superposition principle.

The equation alone does not specify a solution; a unique solution is usually obtained by setting a problem with further conditions, such as initial conditions, which prescribe the value and velocity of the wave. Another important class of problems specifies boundary conditions, for which the solutions represent standing waves, or harmonics, analogous to the harmonics of musical instruments.

The wave equation, and also modifications of it, are found in elastic physics, quantum mechanics, plasma physics and general relativity, for example.
