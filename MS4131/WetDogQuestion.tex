

\documentclass[a4paper,12pt]{article}
%%%%%%%%%%%%%%%%%%%%%%%%%%%%%%%%%%%%%%%%%%%%%%%%%%%%%%%%%%%%%%%%%%%%%%%%%%%%%%%%%%%%%%%%%%%%%%%%%%%%%%%%%%%%%%%%%%%%%%%%%%%%%%%%%%%%%%%%%%%%%%%%%%%%%%%%%%%%%%%%%%%%%%%%%%%%%%%%%%%%%%%%%%%%%%%%%%%%%%%%%%%%%%%%%%%%%%%%%%%%%%%%%%%%%%%%%%%%%%%%%%%%%%%%%%%%
\usepackage{eurosym}
\usepackage{vmargin}
\usepackage{amsmath}
\usepackage{graphics}
\usepackage{epsfig}
\usepackage{subfigure}
\usepackage{fancyhdr}
%\usepackage{listings}
\usepackage{amsmath,enumerate,ifthen}
\usepackage{amssymb}
\usepackage{framed}
\usepackage{times}
%\usepackage{bigints}


\setcounter{MaxMatrixCols}{10}
%TCIDATA{OutputFilter=LATEX.DLL}
%TCIDATA{Version=5.00.0.2570}
%TCIDATA{<META NAME="SaveForMode" CONTENT="1">}
%TCIDATA{LastRevised=Wednesday, February 23, 2011 13:24:34}
%TCIDATA{<META NAME="GraphicsSave" CONTENT="32">}
%TCIDATA{Language=American English}

%\pagestyle{fancy}
%\setmarginsrb{20mm}{0mm}{20mm}{25mm}{12mm}{11mm}{0mm}{11mm}
%\lhead{MA4413} \rhead{Mr. Kevin O'Brien}
%\chead{Statistics For Computing}
%\input{tcilatex}

\begin{document}
\begin{center}
\includegraphics[scale=0.65]{shieldtransparent2}
\end{center}

\begin{center}
\vspace{1cm}
\large \bf {FACULTY OF SCIENCE AND ENGINEERING} \\[0.5cm]
\normalsize DEPARTMENT OF MATHEMATICS AND STATISTICS \\[1.25cm]
\large \bf {END OF SEMESTER EXAMINATION PAPER 2015} \\[1.5cm]
\end{center}

\begin{tabular}{ll}
MODULE CODE: MS4131 & SEMESTER: Spring 2016 \\[1cm]
MODULE TITLE: Linear Algebra 1 & DURATION OF EXAM: 2.5 hours \\[1cm]
LECTURER: Mr. Kevin O'Brien & GRADING SCHEME: 80 marks \\
& \phantom{GRADING SCHEME:} \footnotesize {100\% of module grade} \\[0.8cm]
EXTERNAL EXAMINER: Prof. J. King & \\
\end{tabular}
\bigskip
\begin{center}
{\bf INSTRUCTIONS TO CANDIDATES}
\end{center}

{\noindent \\ Scientific calculators approved by the University of Limerick can be used. \\
%Formula sheet and statistical tables are provided at the end of the exam paper.\\
Students must attempt any 4 questions from 5.}
%==========================================================================================%
\newpage
\section*{Question 1}

	\subsection*{Part A. Matrix Multiplication (4 Marks)}	
	Given the matrices 
	$$
	A=\left(\begin{array}{ccccc} 
	2&2&1&\!\!\!-1&4\end{array}
	\right); \quad
	B =\left(\begin{array}{cc} 
	1&4\\0&5\\\!\!\!-1&0\\7&1\\1&3\end{array}
	\right); \quad
	C=\left(\begin{array}{c} 1\\2\end{array}
	\right)
	$$
	calculate the products $AB$ and $CA$.
	%----------------------------------------%
	\subsection*{Part B. Diagonal Matrices (3 Marks)}	
	Consider the following diagonal matrix D. In terms of the values $a$, $b$ and $c$ ;
	
	
	\[D = \left(\begin{array}{ccc}
		a & 0 & 0 \\ 
		0 & b & 0 \\ 
		0 & 0 & c
	\end{array} \right)\]
	\begin{itemize}
		\item[(i)] (1 Mark) Write an expression for the trace of the matrix D.
		\item[(ii)] (1 Mark) State the inverse of $D$, i.e. $D^{-1}$.
		\item[(iii)] (1 Mark) State the matrix $D^3$.
	\end{itemize}
	\smallskip
	%----------------------------------------%
	\subsection*{Part C. Matrix Multiplication (4 Marks)}	
		Suppose A is a lower triangular matrix of the form;
		\[A = \left(
		\begin{matrix}
		a & 0 & 0 \\
		b & c & 0 \\
		d & e & f \\
		\end{matrix} \right)
		\]
		
		\begin{itemize}
			\item[(i)] (1 Mark) State the transpose of A.
			\item[(ii)] (2 Marks) Compute B where B = $ A \times A^{T}$. 
			\item[(iii)] (1 Mark) B is a symmetric matrix. What is meant by this?
		\end{itemize}
%==========================================================================================%
\newpage
\section*{Question 2}
	%===========================================================================%
	\subsection*{Part A. Fundamental Theorem of Invertible Matrices (5 Marks)}
	The Fundamental Theorem of Invertible Matrices states that a set of mathematical expressions concerning a $n\times n$ matrix $A$ are each equivalent to one another.
	
	\begin{itemize}
		\item[(i)] ($4 \times 1$ Mark)
		State any four of these expressions.
		%\item[(i)] (1 Mark) What is the trace of a square matrix
		\item[(ii)] (1 Mark) What is the rank of a matrix.
	\end{itemize}
	\subsection*{Part B. Invertible Matrices (5 Marks)}	
	\noindent	Show that if $A$ is an $n\times n$ invertible matrix that satisfies 
	$$
	9A^3+A^2-3A=0
	$$
	where $A^n=\underbrace{A\ldots A}_{\textrm{$n$ times}}$, %_{\underbrace\textrm$n $ times$, 
	$I$ is the $n\times n$  identity matrix and $0$ is the $n\times n$  zero matrix,
	then the inverse of $A$ is given by  %\marks{4}
	$$
	A^{-1}=\frac13I+3A.
	$$
	\subsection*{Part C. Inverting a Matrix with E.R.O.s (5 Marks)}	
	Find the inverse of the  matrix 
	
	\begin{equation*}
	A=\left( \begin{array}{rrr}
	1 & -2 & 4\\
	1 & -4 & 1\\
	-3 & 0 & -1
	\end{array} \right).
	\end{equation*}
	using elementary row operations.
%==========================================================================================%
\newpage
\section*{Question 3}
	%=================================================================%
	\subsection*{Part A. System of Linear Equations}
 Consider the linear system
		\begin{align*}
		x_1 + x_3 &= 4\\
		2x_1 + 4x_2 + x_3 &= -3\\
		x_2 + 3x_3 &= 7.
		\end{align*}
		\begin{itemize}
			\item[(i)] (1 Mark) Write down the coefficient matrix and the augmented matrix of this system. %\marks{4}
			
			\item[(ii)] (1 Mark) What can you say about the solution set of the system? Justify your answer. %\marks{4}
			
			\item[(iii)] (4 Marks) Solve the system of equations, using any appropriate method. %\marks{6}
		\end{itemize}
%		\item Consider the homogeneous system:
%		\begin{align*}
%		x_1 + x_3 &= 0\\
%		2x_1 + 4x_2 + x_3 &= 0\\
%		x_2 + 3x_3 &= 0.
%		\end{align*}
%		What can you say about its solution set?%\marks{4}

%==========================================================================================%
\newpage
\section*{Question 4}
	\subsection*{Part A. Vector Calculations (7 Marks)}
	Consider the three vectors in $\mathbb{R}^3$:
	$$
	u = (1, 2, 0), \quad v = (0, 1, 3),\quad w = (1, 2, 3).
	$$
		\begin{itemize}
			\item[(i)] Evaluate $\|u\|$, $\|v\|$, $u\cdot v$, $u\times v$ and the angle between $u$ and $v$. %\marks{3}
		
		\medskip\item[(ii)] Calculate the scalar triple product  $u\cdot(v \times w)$.%\marks{3}
	\end{itemize}
%==========================================================================================%
\newpage
\section*{Question 5}

\end{document}




