\documentclass[12pt,a4paper]{article}
\setlength{\hoffset}{-0.8in} \setlength{\voffset}{-0.55in}
\setlength{\textheight}{240mm}
\setlength{\textwidth}{150mm}
\setlength{\headheight}{14pt}
% \usepackage{fancybox,latexsym,epic,eepic,epsfig}
\usepackage{amsmath,enumerate,ifthen}
\usepackage{amssymb}
\usepackage{framed}
\usepackage{multicol}
\usepackage{times}
\usepackage{fancyhdr}\usepackage{graphicx}
% \input def.tex

\usepackage{pgfpages}
% \pgfpagesuselayout{2 on 1}[a4paper, landscape]

\newcommand{\R}{\mathbb{R}}
\newcommand{\N}{\mathbb{N}}
\newcommand{\bs}{\bigskip}


\begin{document}
\section*{Question 1 (25 Marks)}

\subsection*{Part A}
	Given the matrices 
	$$
	A=\left(\begin{array}{ccccc} 2&3&0&\!\!\!-1&4\end{array}\right); 
	B =\left(\begin{array}{cc} 1&4\\0&5\\\!\!\!-1&0\\4&1\\1&0\end{array}\right); 
	C=\left(\begin{array}{c} 3\\1\end{array}\right)
	$$
	calculate the products $AB$ and $CA$.
\subsection*{Part B}
For the matrices below, evaluate the following expressions where it is possible.
\begin{equation*}
A=\left[ \begin{array}{cc} 1  & 2 \\ 3 & 4 \end{array}\right],
B=\left[ \begin{array}{cc} -2  & 0 \\ 1 & -7 \end{array}\right],
C=\left[ \begin{array}{ccc} 3  & 2 & -2 \\ 4 & 8 & 2 \end{array}\right],
D=\left[ \begin{array}{ccc} 3  & 2 & -2 \\ 4 & 8 & 2 \end{array}\right],
\end{equation*}
\begin{equation*}
E=\left[ \begin{array}{ccc} 1  & 2 & 3 \\ 4 & 5 & 6 \\ 7 & 8 & 9 \end{array}\right],
F=\left[ \begin{array}{ccc} -1  & 0 & 2 \\ 3 & 4 & 1 \\  3 & 1 & 0 \end{array}\right],     
\end{equation*}
\begin{multicols}{2}
	\begin{enumerate}
		\item $2A+3B$
		\item $3C-D$
		\item $8A+4C$
		\item $2000A+3000B$
		\item $E-F$
		\item $\det(A)+\det(B)$
		\item $\det(A+B)$
		\item $\det(C)$
		%			\item $\det(E)$
		%			\item $A\vec{x}$
		%			\item $B\vec{x}$
		%			\item $A\vec{y}+B\vec{x}$
		%			\item $A\vec{u}$
		%			\item $C\vec{x}$
		%			\item $C\vec{w}$
		%			\item $E\vec{u}$
		%			\item $E\vec{w}-\vec{F}\vec{w}$
		%			
	\end{enumerate}
\end{multicols}
\subsection*{Part A. Addition and Subtraction of Matrice}
\begin{itemize}
	\item[(a)] 
\item[(b)]
	Suppose A is a lower triangular matrix of the form 
	\[\left(
	\begin{matrix}
	a & 0 & 0 \\
	b & c & 0 \\
	d & e & f \\
	\end{matrix} \right)
	\]
	
	\begin{itemize}
\item[(i)] State the transpose of A.
\item[(ii)] Compute B where B = $ A \times A^{T}$
\item[(iii)] B is a symmetric matrix. What is meant by this?
	\end{itemize}
	
		
		\item[(c)] Let A and B be $m \times n$ matrices. Then:
		
		\begin{itemize}
			\item[(i)] $(kA)^T = kA^T$
			\item[(ii)] $(A+B)^T = A^T + B^T$
			\item[(iii)] $(AB)^T = B^TA^T$
		\end{itemize}
		
		
\item[(d)]For a square matrix A show that:
		
		\begin{itemize}
			\item[(i)] $AA^T$ and $A+A^T$ are symmetric
			\item[(ii)] $A-A^T$ is skew symmetric
			\item[(iii)] A can be expressed as the sum of a symmetric matrix, $\frac{1}{2}(A+A^T)$ and a skew 
			symmetric matrix $\frac{1}{2}(A-A^T)$
		\end{itemize}
		\end{itemize}
		

	
\subsection*{Invertible Matrices}	
\noindent	Show that if $A$ is an $n\times n$ invertible matrix that satisfies 
	$$
	9A^3+A^2-3A=0
	$$
	where $A^n=\underbrace{A\ldots A}_{\textrm{$n$ times}}$, %_{\underbrace\textrm$n $ times$, 
	$I$ is the $n\times n$  identity matrix and $0$ is the $n\times n$  zero matrix,
	then the inverse of $A$ is given by  %\marks{4}
	$$
	A^{-1}=\frac13I+3A.
	$$\vfill
\end{document}