\documentclass[12pt,a4paper]{article}
\setlength{\hoffset}{-0.8in} \setlength{\voffset}{-0.55in}
\setlength{\textheight}{240mm}
\setlength{\textwidth}{150mm}
\setlength{\headheight}{14pt}
% \usepackage{fancybox,latexsym,epic,eepic,epsfig}
\usepackage{amsmath,enumerate,ifthen}
\usepackage{amssymb}
\usepackage{framed}
\usepackage{times}
\usepackage{fancyhdr}\usepackage{graphicx}
% \input def.tex

\usepackage{pgfpages}
% \pgfpagesuselayout{2 on 1}[a4paper, landscape]

\newcommand{\R}{\mathbb{R}}
\newcommand{\N}{\mathbb{N}}
\newcommand{\bs}{\bigskip}


\begin{document}
	\section*{Question 2 (25 Marks)}
	
	%===========================================================================%
	\subsection*{Part A. Fundamental Theorem of Invertible Matices}
	\begin{itemize}
		\item[(i)] The fundamental Theorem of Invertible Matrices states that a set of mathematical expressions concerning a $n\times n$ matrix $A$ are each equivalent to one another.
		State any four of these expressions.
		\item[(i)] (1 Mark) What is the trace of a square matrix
		\item[(ii)] (1 Mark) What is the Rank of a matrix.
	\end{itemize}
	
	%==============================================%
	
	Reduce the following matrix to row echelon form
	
	%----------------------------------------%
	
	Consider the following diagonal matrix A. In terms of the values $a$,$b$ and $c$ ;
	\begin{itemize}
		\item[(i)] Write an expression for the trace of the matrix
		\item[(ii)] State the inverse of $A$, i.e. $A^{-1}$
		\item[(iii)] State the matrix $A^3$
	\end{itemize}

	
	%----------------------------------------------%
\subsection*{Part B. Inverting a Matrix using Co-Factors}
	% Matrix of Minors
	% Determinant of a 3 by 3 matrix
	
	
	
	
	Given the  matrix 
	\begin{equation*}
	A=\left( \begin{array}{rrr}
	-1 & 2 & 0\\
	1 & 1 & 0\\
	2 & -1 & 2
	\end{array} \right).
	\end{equation*}
	calculate
	\begin{itemize}
		\item the determinant of $A$;
		\item the cofactor matrix of $A$;
		\item and {\bf hence} the inverse matrix $A^{-1}$ .
	\end{itemize}
	
	
	
	\begin{itemize}	
		\item Evaluate the  minors and cofactors of $A$, for $A$ given by
		%
		%\begin{eqnarray*}
		%& &\!\!\!\!\!\!\!\!A=\left( \begin{array}{ccc}
		%2 & 3 & \!\!\!\!-1\\
		%0 & 1 & 3\\
		%\!\!\!\!-3 & 0 & 0\end{array}\right);\qquad A=\left( \begin{array}{ccc}
		%1 & 5 & 4\\
		%0 & 1 & 1\\
		%0 & 0 & 7
		%\end{array} \right);\\
		%& &\!\!\!\!\!\!\!\!A=\left( \begin{array}{ccc}
		%1 & 3 & 4\\
		%8 & 0 & 1\\
		%0 & 0 & 0\end{array}\right);\qquad A=\left( \begin{array}{ccc}
		%1 & 3 & 4\\
		%0 & 1 & 1\\
		%\!\!\!\!-6 & 0 & \!\!\!\!-1
		%\end{array} \right).
		%\end{eqnarray*}
		and hence, in each case, construct the cofactor matrix $\textnormal{Cof(A)}$ of $A$.
	\end{itemize}
	
	
	
	Find the inverse of the  matrix 
	
	\begin{equation*}
	A=\left( \begin{array}{rrr}
	1 & -2 & 4\\
	1 & -4 & 1\\
	-3 & 0 & -1
	\end{array} \right).
	\end{equation*}
	using elementary row operations.
	
	
	2. Let a triangular matrix be a square matrix with either all $(i,j)$ entries zero for either $i<j$ 
	(in which case it is called an lower triangular matrix) or for $j<i$ (in which case it is called an upper triangular matrix). 
	Show that any triangular matrix satisfying $AA^T = A^TA$ is a diagonal matrix.
	
	
	
	%4. Suppose A is a m×n matrix and x is a $n \times 1$ column vector. Show that if 
	%\[x=\begin{pmatrix}x_1\\x_2\\\vdots\\x_n\end{pmatrix}\] and 
	%\[A=\begin{pmatrix}c_1 & c_2 &\cdots c_n \end{pmatrix} where c_j=\begin{pmatrix}A_{1j}\\A_{2j}\\\vdots\\A_{mj}\end{pmatrix}\]
	%then \[Ax = x_1c_1+x_2c_2+\cdots x_nc_n. \]
	
	This is also expressed by saying that Ax is a linear combination of the columns of A.
	%=================================================================%
	\subsection*{Part C. System of Linear Equations}
		
	\begin{enumerate}
		\item Consider the linear system
		\begin{align*}
		x_1 + x_3 &= 4\\
		2x_1 + 4x_2 + x_3 &= -3\\
		x_2 + 3x_3 &= 7.
		\end{align*}
		\begin{enumerate}
			\item Write down the coefficient matrix and the augmented matrix of this system. %\marks{4}
			
			\item What can you say about the solution set of the system? Justify your answer. %\marks{4}
			
			\item Solve the system of equations, using any appropriate method. %\marks{6}
		\end{enumerate}
		\item Consider the homogeneous system:
		\begin{align*}
		x_1 + x_3 &= 0\\
		2x_1 + 4x_2 + x_3 &= 0\\
		x_2 + 3x_3 &= 0.
		\end{align*}
		What can you say about its solution set?%\marks{4}
	\end{enumerate}

	\begin{enumerate}
		\item Prove that, for any $u,\:v\:\in\mathbb{R}^3$, %\marks{8}
		$$(u\times v)\times w= (u\cdot w)v - (v\cdot w)u.$$
		(It is sufficient to verify this property for one component.)
		\item
		Consider the three vectors in $\mathbb{R}^3$:
		$$
		u = (1, 2, 0), \quad v = (0, 1, 3),\quad w = (1, 2, 3).
		$$
		\begin{enumerate}
			\item  Evaluate $\|u\|$, $\|v\|$, $u\cdot v$, $u\times v$ and the angle between $u$ and $v$. %\marks{3}
			
			\medskip\item Calculate the scalar triple product  $u\cdot(v \times w)$.%\marks{3}
		\end{enumerate}
	\end{enumerate}
	
	%============================================================%
	Fundamental Theorem of Invertible Matrices
	Rank
	Trace
	
	
	
	
	

	
\end{document}