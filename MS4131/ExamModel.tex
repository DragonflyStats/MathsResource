\documentclass[12pt,a4paper]{article}
\setlength{\hoffset}{-0.8in} \setlength{\voffset}{-0.55in}
\setlength{\textheight}{240mm}
\setlength{\textwidth}{150mm}
\setlength{\headheight}{14pt}
% \usepackage{fancybox,latexsym,epic,eepic,epsfig}
\usepackage{amsmath,enumerate,ifthen}
\usepackage{amssymb}

\usepackage{times}
\usepackage{fancyhdr}\usepackage{graphicx}
% \input def.tex

\usepackage{pgfpages}
% \pgfpagesuselayout{2 on 1}[a4paper, landscape]

\newcommand{\R}{\mathbb{R}}
\newcommand{\N}{\mathbb{N}}
\newcommand{\bs}{\bigskip}


%%%%%%%%%%%%%%%%%%%%%%%%%%%%%%%%%%%%%%%%%%%%%%%%%%%%%%%%%%%%%%%%%%%%%%%%%%%%%%
%%%%%%%%%%%%%%%%%%%%%%%%%%%%%%%%%%%%%%%%%%%%%%%%%%%%%%%%%%%%%%%%%%%%%%%%%%%%%%
%%%%%%%%%%%%%%%%%%%%%%%%%%%%%%%%%%%%%%%%%%%%%%%%%%%%%%%%%%%%%%%%%%%%%%%%%%%%%%
%%%                                                                        %%%
%%%      Edit the definition of \modcode, \modtitle, \examiner, \term,     %%%
%%%                             \duration, \percentage, \extern            %%%
%%%                                                                        %%%
%%%     as necessary. Delete the \% in the definition of \percentage       %%%
%%%     if it is not a percentage!                                         %%%
%%%                                                                        %%%
%%%     Edit the definition of \instructions as necessary.                 %%%
%%%                                                                        %%%
%%%                                                                        %%%
%%%                                                                        %%%
%%%%%%%%%%%%%%%%%%%%%%%%%%%%%%%%%%%%%%%%%%%%%%%%%%%%%%%%%%%%%%%%%%%%%%%%%%%%%%
%%%%%%%%%%%%%%%%%%%%%%%%%%%%%%%%%%%%%%%%%%%%%%%%%%%%%%%%%%%%%%%%%%%%%%%%%%%%%%
%%%%%%%%%%%%%%%%%%%%%%%%%%%%%%%%%%%%%%%%%%%%%%%%%%%%%%%%%%%%%%%%%%%%%%%%%%%%%%


\def\modcode{MS4117}
\def\modtitle{Discrete Mathematics 2}
\def\examiner{Dr A. Hegarty}
\def\term{Autumn 2014}
\def\duration{2$\frac 12$ hours}
\def\percentage{80 \%}
\def\extern{Prof. J. King}
\def\instructions{
	\bigskip
	
	\bf Full marks for correct answers to {\bf four} questions.
	
	\bigskip\bf The graphs for each question are displayed in larger format at the end of the \\
	examination booklet.}


%%%%%%%%%%%%%%%%%%%%%%%%%%%%%%%%%%%%%%%%%%%%%%%%%%%%%%%%%%%%%%%%%%%%%%%%%%%%%%
%%%%%%%%%%%%%%%%%%%%%%%%%%%%%%%%%%%%%%%%%%%%%%%%%%%%%%%%%%%%%%%%%%%%%%%%%%%%%%
%%%%%%%%%%%%%%%%%%%%%%%%%%%%%%%%%%%%%%%%%%%%%%%%%%%%%%%%%%%%%%%%%%%%%%%%%%%%%%
%%%                                                                        %%%
%%%      Now go to the section marked START HERE                           %%%
%%%      and start typing. Note the use of nested enumerate environments   %%%
%%%      to generate question/section/subsection numbers.                  %%%
%%%                                                                        %%%
%%%                                                                        %%%
%%%                                                                        %%%
%%%%%%%%%%%%%%%%%%%%%%%%%%%%%%%%%%%%%%%%%%%%%%%%%%%%%%%%%%%%%%%%%%%%%%%%%%%%%%
%%%%%%%%%%%%%%%%%%%%%%%%%%%%%%%%%%%%%%%%%%%%%%%%%%%%%%%%%%%%%%%%%%%%%%%%%%%%%%
%%%%%%%%%%%%%%%%%%%%%%%%%%%%%%%%%%%%%%%%%%%%%%%%%%%%%%%%%%%%%%%%%%%%%%%%%%%%%%




\def\marks#1{\hfill\rlap{~~~~~~~~~~~#1}}
\def\question#1{\bigbreak{Q.#1}}
\def\examsection#1{\medbreak{#1}}
\def\subsect#1{\smallbreak{#1}}


\def\examtitle{\hsize 6.5true in
	\begin{center}
		\includegraphics[width=0.5\textwidth]{uni}\\
		
	\end{center}
	
	\begin{center}
		Faculty of Science and Engineering\\
		Department of Mathematics and Statistics
	\end{center}
	
	% \hfill \\ \\
	
	% \begin{tabular}[h]{ll}
	% \small MODULE CODE: \modcode&\small SEMESTER: \term \\ \\
	% \small MODULE TITLE: \modtitle&\small DURATION OF EXAMINATION: \duration \\ \\
	% \small LECTURER: \examiner&\small PERCENTAGE OF TOTAL MARKS: \percentage \\ \\
	% \small INTERNAL EXAMINER: \extern \\ \\
	% \multicolumn{2}{l}{\begin{minipage}[h]{16cm}INSTRUCTIONS TO CANDIDATES: \instructions\end{minipage}}
	% \end{tabular}
	\begin{tabular}[h]{ll}
		MODULE CODE: \modcode  & SEMESTER: \term \\[0.5cm]
		MODULE TITLE: \modtitle& DURATION OF EXAMINATION: \duration \\[0.5cm]
		LECTURER: \examiner    & PERCENTAGE OF TOTAL MARKS: \percentage \\[0.5cm]
		EXTERNAL EXAMINER: \extern \\[1cm]
		\multicolumn{2}{l}{\begin{minipage}[h]{\textwidth}INSTRUCTIONS TO CANDIDATES: \instructions\end{minipage}}
	\end{tabular}
}

\begin{document}
	
	%=============================================================%
	
	
\subsection*{Question 1}
1) Basic Exercises
	Matrix Multiplication
	Arranging Formulaes
	Identity Matrix Equations
	Proofs
	
	
	\begin{itemize}
\item What is the trace of a matrix
\item What is meant by a Symmetric Matrix
\item The Fundamental Theorem of Invertible Matrices asserts that 
	\end{itemize}
A = 	\[\begin{matrix}{ccc}
	\hline a & 0 & 0 \\ 
	\hline b & c  & 0  \\ 
	\hline d & e &  f \\ 
	\hline 
	\end{matrix} \]
	
	Compute $B = A \times A^{T}$

	% Cholesky Calculation
Given the following matrices
\begin{equation*}
 \!\!\!\!\!\!\!\!\!\!\!\!\!\!\!\!A=\left( \begin{array}{ccc}
1 & 4 & \!\!\!-2\\
2 & 1 & 0\end{array} \right);\qquad B=\left( \begin{array}{cc}
3 & 2\\
1 & 4 \end{array} \right);\qquad
C=\left( \begin{array}{ccc}
\!\!\!\!-1 & \!\!\!\!-3 & 4\\
2 & 4 & \!\!\!\!-3\end{array}\right);\qquad D=\left( \begin{array}{cc}
3 & 1\\
4 & 0\\
\!\!\!\!-2 & 5 \end{array} \right),
\end{equation*}
calculate (if possible) the following operations (justify your
answer for any operation you think may not be performed)

(a) $AB$; (b)  $BA$; (c) $A+C$; (d)  $BC$; (e) $(A+2C)D$.


%=======================================================%



%=======================================================%
%-http://www.cse.scu.edu/~atkinson/teaching/su09/247/exam5.pdf

%If matrix A is diagonalizable, then its transpose A
%T must be diagonalizable as well.
%Solution: True; if A = SDS−1
%, then A
%T = S
%T D(S
%T
%)
%−1
.
%============================================================%
Fundamental Theorem of Invertible Matrices
Rank
Trace

%=================================================================%
\subsection*{Quesiton 2}

2) Inverting a Matrix using Co-Factors
	Matrix of Minors
	Determinant of a 3 by 3 matrix

\begin{itemize}	
\item Evaluate the  minors and cofactors of $A$, for $A$ given by

\begin{eqnarray*}
& &\!\!\!\!\!\!\!\!A=\left( \begin{array}{ccc}
2 & 3 & \!\!\!\!-1\\
0 & 1 & 3\\
\!\!\!\!-3 & 0 & 0\end{array}\right);\qquad A=\left( \begin{array}{ccc}
1 & 5 & 4\\
0 & 1 & 1\\
0 & 0 & 7
\end{array} \right);\\
& &\!\!\!\!\!\!\!\!A=\left( \begin{array}{ccc}
1 & 3 & 4\\
8 & 0 & 1\\
0 & 0 & 0\end{array}\right);\qquad A=\left( \begin{array}{ccc}
1 & 3 & 4\\
0 & 1 & 1\\
\!\!\!\!-6 & 0 & \!\!\!\!-1
\end{array} \right).
\end{eqnarray*}
and hence, in each case, construct the cofactor matrix $\textnormal{Cof(A)}$ of $A$.
\end{itemize}

1. Let A and B be $m \times n$ matrices. Then:

\begin{itemize}
\item[(i)] $(kA)^T = kA^T$
\item[(ii)] $(A+B)^T = A^T + B^T$
\item[(iii)] $(AB)^T = B^TA^T$
\end{itemize}

2. Let a triangular matrix be a square matrix with either all $(i,j)$ entries zero for either $i<j$ 
(in which case it is called an lower triangular matrix) or for $j<i$ (in which case it is called an upper triangular matrix). 
Show that any triangular matrix satisfying $AA^T = A^TA$ is a diagonal matrix.

3. For a square matrix A show that:

\begin{itemize}
\item[(i)] $AA^T$ and $A+A^T$ are symmetric
\item[(ii)] $A-A^T$ is skew symmetric
\item[(iii)] A can be expressed as the sum of a symmetric matrix, $\frac{1}{2}(A+A^T)$ and a skew 
symmetric matrix $\frac{1}{2}(A-A^T)$
\end{itemize}

4. Suppose A is a m×n matrix and x is a $n \times 1$ column vector. Show that if 
\[x=\begin{pmatrix}x_1\\x_2\\\vdots\\x_n\end{pmatrix}\] and 
\[A=\begin{pmatrix}c_1 & c_2 &\cdots c_n \end{pmatrix} where c_j=\begin{pmatrix}A_{1j}\\A_{2j}\\\vdots\\A_{mj}\end{pmatrix}\]
 then \[Ax = x_1c_1+x_2c_2+\cdots x_nc_n. \]

This is also expressed by saying that Ax is a linear combination of the columns of A.
%=================================================================%
\subsection*{Quesiton 3}

3) Planes
	Distance
%=================================================================%
\subsection*{Quesiton 4}
4) Vectors / Systems of Linear Equations
	Cross Product
	Scalar Triple Product


 Find the inverse of the  matrix 

\begin{equation*}
A=\left( \begin{array}{rrr}
1 & -2 & 4\\
1 & -4 & 1\\
-3 & 0 & -1
\end{array} \right).
\end{equation*}
using elementary row operations.

 Given the  matrix 
\begin{equation*}
A=\left( \begin{array}{rrr}
-1 & 2 & 0\\
1 & 1 & 0\\
2 & -1 & 2
\end{array} \right).
\end{equation*}
calculate
\begin{itemize}
\item the determinant of $A$;
\item the cofactor matrix of $A$;
\item and {\bf hence} the inverse matrix $A^{-1}$ .
\end{itemize}
%=================================================================%
\subsection*{Quesiton 5}
5) Eigenvalues / Diagonalization
	Characteristic Polynomial
	Power Formula
\begin{itemize}
\item Given
$u,\:u^{\prime},\:v,\:v^{\prime},w\:w^{\prime}$, with

\begin{eqnarray*}
& & u=(1,3,0);\qquad u^{\prime}=(-3,1,5)\\
& & v=(5,0,4);\qquad v^{\prime}=(-4,3,5)\\
& & w=(3,2,7);\qquad w^{\prime}=(1,0,1),
\end{eqnarray*}

calculate $u\cdot u^{\prime}$, $v\cdot v^{\prime}$, $w\cdot
w^{\prime}$. Which of the pairs are orthogonal vectors?

\vspace{0.25cm}\item Calculate the (Euclidean) norm of the
following vectors

\begin{eqnarray*}
& & u=(1,2)\\
& & v=(3,0)\\
& & w=(4,0,3)\\
& & 0=(0,0,0).
\end{eqnarray*}
%================================================%
\item Calculate the scalar triple product

\[u\cdot (v\times w)\]

for

\begin{enumerate}
\item $u=(1,3,5)$; $v=(0,5,3)$; $w=(3,0,7)$;

\item $u=(0,1,2)$; $v=(5,0,1)$; $w=(2,2,2).$
\end{enumerate}
\vfill\item Are the points 
$$P_1=(1,2,0), \quad P_2=(3,5,0),\quad P_3=(7,3,0), \quad P_4=(-5,3,0)
$$
coplanar? If yes, what is the equation of the plane containing them?

\vfill\item
\begin{enumerate}
\item Find the equation of the line $\ell$ in $\mathbb{R}^2$, which  passes through the 
points $(2,1)$ and $(1,3)$ .

\item  Let $Q = (1, -3)$  be a point in $\mathbb{R}^2$.
\begin{enumerate}
\item Verify that $Q$ does not lie on the line $\ell$.

\item Find the distance between the point $Q$ and the line $\ell$.
\end{enumerate}
\end{enumerate}\newpage
\vspace{0.25cm}

\end{itemize}
\end{document}

\item Let $A$ be an $3\times 3$ matrix. Prove that for any vectors $u$ and $v$  in $\mathbb{R}^3$ ($u$, $v$ are $3\times 1$ column vectors) we have
