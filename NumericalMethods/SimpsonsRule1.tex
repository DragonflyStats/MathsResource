\documentclass{beamer}
\usepackage{amsmath}
\usepackage{amssymb}

\begin{document}
	%%% - http://leavingcertmaths.blogspot.ie/2008/04/simpsons-rule.html
%===========================================================================%
\begin{frame}[fragile]
\frametitle{Simpson's Rule}



Today we looked at the application of Simpson's rule for working out the area of an irregular shape.

The shape is divided into an even number of strips which are of equal width.

\end{frame}
%===========================================================================%
\begin{frame}[fragile]

You are given a shape like this (taken from LC paper 2005).
It is divided into an even number of strips (in this case 6) and each strip is bounded by a strip length. As there are 6 strips there are 7 strip lengths.

Note that the last strip length is zero.


First step is to label each strip length as odd or even.

%IMAGE

\end{frame}
%===========================================================================%
\begin{frame}[fragile]


Now apply Simpson's rule. 
The rule is that the area of this shape
= w/3(First + last + 2(sum of odds) + 4(sum of evens)) 
Filling this in:
\begin{verbatim}
area = 18/3 (10 + 0 + 2(30 + 36) + 4 (25 + 38 + 22))
= 6(10 + 132 + 340)
= 6(482)
= 2892m²
\end{verbatim}
\end{frame}
%===========================================================================%
\begin{frame}[fragile]
	\frametitle{Simpson's Rule}
	\Large
[As it is easy to make an error in the adding and multiplying, it is a good habit to do a rough check - to make sure that your answer is in the right ballpark. You can check this by approximating the average length of strip and multiplying by the total width. Width = 18 x 6 = 108 Average length = 26 (rough estimate) Ballpark for area = 108 x 26 = 2808m%sup2;]


In this diagram, there are 4 (an even number) strips, and there are two sets of measurements for each strip.
\end{frame}
%===========================================================================%
\begin{frame}[fragile]
	\frametitle{Simpson's Rule}
	\Large
Label the strip lengths as odd or even and add the upper length to the lower one.

... giving you this
The total lengths are in green. You then sub these into the Simpson's Rule formula:

\end{frame}
%===========================================================================%
\begin{frame}[fragile]

\begin{verbatim}
Area = 5/3(0 + 0 + 2(11) + 4(8 + 6))
= 5/3(22+ 56)
= 5/3(78)
= 130
\end{verbatim}
\end{frame}
%===========================================================================%
\begin{frame}[fragile]
Again ballpark estimate would be: 20x 7 = 140.


\end{frame}
%===========================================================================%
\begin{frame}[fragile]
\frametitle{Simpson's rule with Missing Values}
There are two possibilities - either the width of each strip is missing or one of the strip lengths is missing. In either case, you will be given the area and you use it to build an equation around your unknown.

In this example the strip width is not given.
Let's say that you are told that the
area = 1432m²
\end{frame}
%===========================================================================%
\begin{frame}[fragile]
	\frametitle{Simpson's Rule}
	\Large

You set up an equation like this:
1432 = x/3(0 + 0 + 2(37) + 4(30+41))
1432 = x/3(358)
4296 = 358x
x = 12

\end{frame}
%===========================================================================%
\begin{frame}[fragile]
\frametitle{Simpson's Rule}
\Large


The missing value could be in the strip length.
In fact there could be two missing values, both expressed in terms of x. For example.

In this example, the area = 270m²
\end{frame}
%===========================================================================%
\begin{frame}[fragile]
	\frametitle{Simpson's Rule}
	\Large

Set up an equation:

270 = 3/3(12 + x + 2(2x+18) + 4(18+22+8))
270 = 1(12 + x + 4x + 36 + 192)
270 = 5x + 240
300 = 5x
x = 6
\end{frame}
%===========================================================================%
\begin{frame}[fragile]
	\frametitle{Simpson's Rule}
	\Large

Note that there are a few variants of Simpson's rule.
You could be asked to work out the area between the graph of a function and the x-axis (see 2002 question).
You could be asked to copy a diagram, measure it and work out the area. 
\end{frame}
%===========================================================================%


\end{document}