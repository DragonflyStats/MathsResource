\documentclass{beamer}

\usepackage{amsmath}
\usepackage{amssymb}

\begin{document}
%================================================== %
\begin{frame}
\frametitle{Numerical Integration}
Numerical integration constitutes a broad family of algorithms for calculating the numerical value of a definite integral, and by extension, the term is also sometimes used to describe the numerical solution of differential equations.
\end{frame}

%================================================== %
\begin{frame}
\frametitle{Numerical Integration}
Numerical integration constitutes a broad family of algorithms for calculating the numerical value of a definite integral, and by extension, the term is also sometimes used to describe the numerical solution of differential equations.
This article focuses on calculation of definite integrals. The term numerical quadrature (often abbreviated to quadrature) is more or less a synonym for numerical integration, especially as applied to one-dimensional integrals.
\end{frame}

%================================================== %
\begin{frame}
	\frametitle{Numerical Integration}
Numerical integration over more than one dimension is sometimes incorrectly described as cubature, since the meaning of quadrature is understood for higher-dimensional integration as well.

\begin{itemize}
\item Trapezoidal Rule
\item Simpson’s Rule
\end{itemize}

\end{frame}

%================================================== %
\begin{frame}
\frametitle{Numerical Integration: Simpson's Rule}
\[\int_{a}^{b} f(x) \, dx \approx \frac{b-a}{6}\left[f(a) + 4f\left(\frac{a+b}{2}\right)+f(b)\right].\]

\end{frame}
%================================================== %
\begin{frame}
\frametitle{Trapezoidal Rule}
\[\int_{a}^{b} f(x)\, dx \approx (b-a)\frac{f(a) + f(b)}{2}\]

\end{frame}
%================================================== %
\end{document}