\documentclass{beamer}
\usepackage{array}
\usepackage{booktabs}
\usepackage{amsmath}
\usepackage{amssymb}
\usepackage{graphics}
\setlength{\heavyrulewidth}{1.5pt}
\setlength{\abovetopsep}{4pt}


\begin{document}

\begin{frame}
\bigskip
{
\Huge
\[ \mbox{Set Theory} \]

}
{
\LARGE
\[ \mbox{www.MathsResource.com} \]

}
\end{frame}
%---------------------------------------%
Question 2B 2010 Zone A


\begin{itemize}
\item Let A and B be subsets of the a universal set $U$.
\item Use membership tables to prove that $(A \cup B^{\prime})^{\prime}$ = $A^{\prime} \cap B$
\item Shade the regions corresponding to this set on a Venn Diagram
\end{itemize}

\begin{array}{|c|c|| c | c| c|}
A	&	B	&	$B^{\prime}$	&	$A \cup B^{\prime}$	&	$(A \cup B^{\prime})^{\prime}$	\\ \hline
0	&	0	&	1	&	1	&	0	\\
0	&	1	&	0	&	0	&	1	\\
1	&	0	&	1	&	1	&	0	\\
1	&	1	&	0	&	1	&	0	\\
\end{array}									


\begin{array}{|c|c|| c | c| }									
A	&	B	&	$A^{\prime}$	&	$A^{\prime} \cap B$	\\	\hline	
0	&	0	&	1	&	0	\\		
0	&	1	&	1	&	1	\\		
1	&	0	&	0	&	0	\\		
1	&	1	&	0	&	0	\\		
\end{array}



%-------------------------------------------%
% 2010 Zone A Q 1c
Given the universal set $U$ and subsets $A$ and $B$, list the set $(A \cup B^{\prime})^{\prime}$
\begin{itemize}
\item $U=\{1,2,\ldots,8,9\}$
\item $A=\{2,4,6,8\}$
\item $B=\{ 4,5,6,7\}$
\item $B^{\prime}=\{ 1, 2, 3, 8, 9  \}$
\item $A \cup B^{\prime}=\{ 1, 2, 3,4, 6, 8, 9  \}$
\item $(A \cup B^{\prime})^{\prime}=\{ 5,7 \}$
\end{itemize}

%------------------------------------------$
2010 Zone B Q 1

5n+1 Rules of Inclusion method

$A = \{5n+1: n \element Z \}$
\subsection*{Floating Point Notation}
(Demonstration on white board)

\subsection*{2011 Zone A question 1d}

Showing your workings, express the repeating decimal 0.012012012012...
as a rational number in its simplest form.


begin{itemize}
\item x = 0.012012012012...
\item 10x = 0.12012012012... (not particularly useful )
\item 100x = 1.2012012012... (not particularly useful either)
\item 1000x= 12.012012012... (very useful)
\item 999x = 12
\item x= 12/999 = 4/333 (Answer!)
\end(itemize}

%------------------------------------------$

\subsection*{2008  Zone A question2a}
$B = \{3n-1 :n \in Z^{+} \}$
Describe the set B using the listing method

\begin{itemize}
\item Let $n=1$. Consequently $3(1)-1 =2$
\item Let $n=2$. Likewise $3(2)-1 =5$
\item Let $n=3$. $3(3)-1 = 8 $
\item The repeated differences are 3. The next few values are 11, 14 and 17
\item So by the listing method $B= \{2,5,8,11,14,17,\ldots\}$
\end{itemize}

$A = \{3,5,7,9,ldots \}$
Describe the set A using the rules of inclusion method

\begin{itemize}
\item The repeated differences are 2.
\item We can say the rule has the form $2n+k$
\item For the first value n=1. Therefore $2+k=3$
\item Checking this , for the second value , n=2. Therefore $4+k=5$
\item Clearly k = 1.
\item $A = \{2n+1 :n \in Z^{+} \}$
\item So by the listing method $B= \{2,5,8,11,14,17,\ldots\}$
\end{itemize}
