
VaVaVoom
The plant in Austria produces 80% of the cars.
The plant in Belgium produces 20% of the cars.
A randomly chosen car was build at Austrian plant
A randomly chosen car was built at the Belgian plant
S: A randomly chosen car has standard
The mean and standard deviation of the following
We are told the following piece of information \bar{x} = 44
So what is the coefficient of determination?
The duration, in months, of the construction phase of a number of motorway projects
were collected and tabulated as follows
Calculate the mean value of the durations.
Calculate the variance of the data set.
Calculate the standard deviation.
Calculate the coefficient of variation.
Firstly, what is the sample size?
\cv= {s over \bar{x}} \times 100%
\bar{x} = {49 + 55 + 43 + 45 + 41 +33 + 42 \over 7}
\bar{x} = 44
s^2 = 47
s = \sqrt{47}

{(49-44)^2 + (55-44)^2 + (43-44)^2 + (45-44)^2 + (41-44)^2 + (33-44)^2 + (42-44)^2 \over 7-1}

25 + 121 + 1 + 1 + 9 + 121 + 4
={282/6} = 47
s = \sqrt{47}

= 15.58%
-----------------------------------------------------------------------
4 & 6 & 8 & 9 & 17 & 17 & 18 & 19 & 20 & 22 \\
22 & 27 & 28 & 29 & 31 & 35 & 38 & 39 & 40 & 46 \\
48 & 56 & 56 & 57 & 57 & 58 & 58 & 60 & 61 & 62 \\
64 & 66 & 68 & 69 & 74 & 75 & 78 & 79 & 80 & 82 \\

lower fence?
Upper fence?
Any values above or below fences?

----------------------------------------------------------------------
A motor dealership which specializes in agricultural machinery sells on vehicle every 2 days, on average

In this question the unit period is one day. The company expects to sell, on average, 0.5 vehicles every day.
The Poisson mean $m$ is therefore 0.5.

P(X \geq 1)

Go to your Poisson tables, and search for the $m=0.5$ column.
We are interested in the probability of \textbf{exactly} one vehicle sold on a particular day.
From the tables we can easily work out P(X \geq 1), but this is probability of one or more vehicles being sold.
This is not the same thing.
P(X \geq 1) = P(X =1) + P( X=2) + P(X=3) + \ldots
P(X \geq 1) = P(X=1) + P(X \geq 2)
From tables
P(X \geq 1)
P(X \geq 2)

Six day working week?
our unit period is now six days.
How many vehicles do we expect to sell in 6 days?
answer = 3
$m=3$

P(X\geq 4)

e^{-6/5} = 0.3011942
e^{-4/5} = 0.449329
e^{-5/5} = 0.3678794

P(B|A) = { P(A \mbox{ and } B) \over P(A) }
 
P(A|B) = { P( A \mbox{ and }B) \over P(B) }

The exponential distribution
The average lifespan ppf a laptop is5 year. You may assume that the lifespan of laptop computers follows an exponential distribution.
What is the probability that the lifespan of the laptop will be at least 6 years.
What is the probability that the lifespan of the laptop will not exceed 4 years.
What is the probability that the lifespan of the laptop will be between 5 years and 6 years.
It is a one tailed test
$H_o$  : $\mu = 80 $
$H_a$  : $\mu \neq 80$
The significance level is 5% (or 0.05)
what is the column to use?
what is the degrees of freedom
Is it a large sample or a small sample?

\bar{x} =
s^2 = 16
CV =

\sqrt{3}{1.09 \times 1.08 \times 1.07}

\sqrt{ \frac{\hat{p} 1- \hat{p}}{n} }

\sqrt{ \frac{\hat{p_1} 1- \hat{p_1}}{n_1} + \frac{\hat{p_2} 1- \hat{p_2}}{n_2}}

H_o
H_a
The F-test
H_0: Both variances are equal
H_a : The variances are different.
Compute the test statistic.
Divide the larger variance by the smaller variance.
The degrees of freedom are as follows
 
\nu_1 size of sample with larger variance
\nu_2 size of sample with smaller variance

There are 5 values tabulated
We use the one for a significance level of 0.05
Carefully read the tables.

Binomial Distribution
There are n independent trials
The probability of a success is
\sum_x
\sum_y
\sum_xy
\sum_x^2
\sum_y^2#
n=10

Fading
>Transitions
>Fade Thru Black
>Drag and Drop
>Right Click on Arrow - Change transition from 3 to 5
>Both First Slide and End Slide
>Change the final slide to 2 second.
August 2011

Correlation
P-values
MathsCast 1

1) Set Exam Paper
2) MCS
3) MathsCast
4) UseR Conference
5) WorkShop
6) Asterix Extras


Please advise if you are still recruiting extras for Asterix film.

If so, please find attached a full length photo of my housemate Konstantin Lurz

19 Crescent Court
Dooradoyle 
Limerick
085 735 0249

Age 19
Height 178 cm
Neck 39cm / 15.5 inches
Chest 94 cm /37 inches
Waist 81cm / 32 Inches
Leg 81cm/32 inches
Foot size 11



Please advise if you are still recruiting extras for Asterix film.

I was in the Reserve Defence Force Medical Corps for 6 years.

I was a Saxon soldier combat extra in the 2003 production of King Arthur, shot in Ballymore Eustace.

I worked for 37 shooting days on battle scenes.

Our company was commended by the Extras Co-ordinator Andrew Ward for our good attitude and work ethic. 








If so, please find attached a full length photo of my housemate Konstantin Lurz

19 Crescent Court
Dooradoyle 
Limerick
085 735 0249

I am based in Limerick, but will look at temporary accommodation  in the vicinity of the unit base should he be selected.




Ma4402
 in the following sequence, evaluate the  term a4
a_n = 2n^2 + 3n-7
a_n =n+2 
:: 5 to n 

question 2
use the rafio test to show that this.sequen e is convergenf.
Use the series ro estomate the value of to thee decimql places.
Writinf your resulta into the table below.

question 3
kepler's problem cor.an orbir with eccentrity and mean anomaly 1 onvolves find the eoot of the function 
The angle is.measured in radians.


3

1) Confidence Interval for a proportion
 
n = 400
 
240 
 
 
\hat{P} = {240 \over 400} = 0.60
 
 
S.E. \;(\hat{P}) = \sqrt{ { \hat{P} \;\times \;(1-\hat{P}) \over n}}
 

 
 
S.E. \;(\hat{P}) = \sqrt{ { 0.60 \;\times \; 0.40 \over 400}}
 
S.E. \;(\hat{P}) = \sqrt{ { 60% \;\times \; 40% \over 400}}


Correlation
The Pearson's Product Moment Correlation Coefficient tells us how well two sets of continuous data correlate to each other. The value can fall between 0.00 (no correlation) and 1.00 (perfect correlation). A p value tells us if the Pearson's is significant or not. Generally p values under 0.05 are considered significant.


P-values
SPSS only reports the p-value to the third decimal place. If you double click on the p-value in your output it should show you more decimal places (version 14+). You should never report the p-value as 0 if SPSS gives a p-value as.000 report it as p<.001.

In short, however, this is a significant value. You should reject H0.

MathsCast 1

Correlation  Coefficient
what is the linear coeffi ient.of the aboce data set
we are gicen the follpwing pieces of data

\begin{itemize}
\item
\item \sum x
\item \sum y \item \sum xy \item \sum x2 \item sum y2
\end{itemize}
one important piece of.informafion we are not given direcrly is the sample size n
but we can work this out easily by looking at the data sef.

There are ten pairs of data so n is 10

lets compute the top half of the equation first

so we get 365.10

now lets look at the first half of the bottom
then we get the square root of this

The correlation coefficient is herefore 0.89

Interpret this: This is a very high strong positive linear relationship.

\sqrt{\hat{p} \times (1-\hat{p} ) \over n}
