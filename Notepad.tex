

1) Confidence Interval for a proportion
 
n = 400
 
240 
 
 
\hat{P} = {240 \over 400} = 0.60
 
 
S.E. \;(\hat{P}) = \sqrt{ { \hat{P} \;\times \;(1-\hat{P}) \over n}}
 

 
 
S.E. \;(\hat{P}) = \sqrt{ { 0.60 \;\times \; 0.40 \over 400}}
 
S.E. \;(\hat{P}) = \sqrt{ { 60% \;\times \; 40% \over 400}}


Correlation
The Pearson's Product Moment Correlation Coefficient tells us how well two sets of continuous data correlate to each other. The value can fall between 0.00 (no correlation) and 1.00 (perfect correlation). A p value tells us if the Pearson's is significant or not. Generally p values under 0.05 are considered significant.


P-values
SPSS only reports the p-value to the third decimal place. If you double click on the p-value in your output it should show you more decimal places (version 14+). You should never report the p-value as 0 if SPSS gives a p-value as.000 report it as p<.001.

In short, however, this is a significant value. You should reject H0.

MathsCast 1

Correlation  Coefficient
what is the linear coeffi ient.of the aboce data set
we are gicen the follpwing pieces of data

\begin{itemize}
\item
\item \sum x
\item \sum y \item \sum xy \item \sum x2 \item sum y2
\end{itemize}
one important piece of.informafion we are not given direcrly is the sample size n
but we can work this out easily by looking at the data sef.

There are ten pairs of data so n is 10

lets compute the top half of the equation first

so we get 365.10

now lets look at the first half of the bottom
then we get the square root of this

The correlation coefficient is herefore 0.89

Interpret this: This is a very high strong positive linear relationship.

\sqrt{\hat{p} \times (1-\hat{p} ) \over n}
