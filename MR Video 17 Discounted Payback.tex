
\documentclass{beamer}

\usepackage{amsmath}
\usepackage{amssymb}
\usepackage{graphics}

\begin{document}
%-------------------------------------------------- %
\begin{frame}
\bigskip
{
\Huge
\[ \mbox{Financial Mathematics}  \]
\huge
\[ \mbox{
 Period}  \]
}

{
\LARGE
\[ \mbox{www.Stats-Lab.com}  \]
\[ \mbox{Twitter: @StatsLabDublin} \]
}
\end{frame}
%-------------------------------------------------- %
\begin{frame}
\frametitle{Discounted Payback Period}
A capital budgeting procedure used to determine the profitability of a project. In contrast to an NPV analysis, which provides the overall value of an project, a discounted payback period gives the number of years it takes to break even from undertaking the initial expenditure. Future cash flows are considered are discounted to time "zero." This procedure is similar to a payback period; however, the payback period only measure how long it take for the initial cash outflow to be paid back, ignoring the time value of money.
\end{frame}
\end{document}