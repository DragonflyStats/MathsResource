\documentclass[12pt]{article}
%\usepackage[final]{pdfpages}

\usepackage{graphicx}
\graphicspath{{/Users/kevinhayes/Documents/teaching/images/}}

\usepackage{tikz}
\usetikzlibrary{arrows}

\newcommand{\bbr}{\Bbb{R}}
\newcommand{\zn}{\Bbb{Z}^n}

%\usepackage{epsfig}
%\usepackage{subfigure}
\usepackage{amscd}
\usepackage{amssymb}
\usepackage{amsbsy}
\usepackage{amsthm}
\usepackage{natbib}
\usepackage{amsbsy}
\usepackage{enumerate}
\usepackage{amsmath}
\usepackage{eurosym}
%\usepackage{beamerarticle}
\usepackage{txfonts}
\usepackage{fancyvrb}
\usepackage{fancyhdr}
\usepackage{natbib}
\bibliographystyle{chicago}

\usepackage{vmargin}
% left top textwidth textheight headheight
% headsep footheight footskip
\setmargins{2.0cm}{2.5cm}{16 cm}{22cm}{0.5cm}{0cm}{1cm}{1cm}
\renewcommand{\baselinestretch}{1.3}


\pagenumbering{arabic}
\begin{document}
	\thispagestyle{empty}
	
	% ******* Title page *******
	% **************************
	\def\marks#1{\marginpar{#1}}	

	\def\modcode{MS4315}
	\def\modtitle{Operations Research 2}
	\def\examiner{Dr. M. Burke \& Mr. \ K.\ O'Brien }
	\def\term{Repeat 2017}%
	%Annual Repeats 2014}
	\def\duration{2$\frac{1}{2}$ hours}
	\def\percentage{100\%}%75\%}
	\def\extern{Prof. J. King}%Prof.\ Brendan Murphy}
	\def\instructions{\bf \\ Answer four questions correctly for full marks.\\
		Answer two	questions from Q1--Q3 and two questions from Q4--Q6.\\
		50\% of the
		marks are for the two questions from Q1--Q3.\\ 50\% for the two
		questions from Q4--Q6..}
	
	\def\examtitle{\hsize 6.5true in
		\begin{center}
			{    \Large UNIVERSITY \textit{of} LIMERICK }\\
			\textsc{OLLSCOIL LUIMNIGH}
		\end{center}
		
		\begin{center}
			Faculty of Science and Engineering
		\end{center}
		
		\hfill \\ \\
		
		\begin{tabular}[h]{ll}
			{\bf REPEAT ASSESSMENT PAPER}\\ \\
			\small MODULE CODE: \modcode&\small SEMESTER: \term \\ \\
			\small MODULE TITLE: \modtitle&\small DURATION OF EXAMINATION: \duration \\ \\
			\small LECTURER: \examiner&\small PERCENTAGE OF TOTAL MARKS: \percentage \\ \\
			\small EXTERNAL EXAMINER: \extern \\ \\
			\multicolumn{2}{l}{
				\begin{minipage}[h]{15.5cm}
					INSTRUCTIONS TO CANDIDATES: \instructions
				\end{minipage}}
			\end{tabular}
		}
		
		\hspace{2.8in}\includegraphics[width=0.7in]{images/shieldtransparent2}
		\examtitle
		
		
		
		\newpage 
		
		%\renewcommand\theenumi{(\alph{enumi})}
		%%renewcommand\theenumii{\roman{enumii}}
		%\def\modcode{MA4007}
		%\def\modtitle{Experimental Design}
		%\def\examiner{Dr.~K.~Coffey }
		%\def\term{Autumn 2010}
		
		\pagestyle{fancy} \setcounter{page}{1}
		\renewcommand{\headrulewidth}{0.4pt}
		\renewcommand{\headsep}{20pt}
		\lhead{\modcode \quad \modtitle \qquad \examiner} \chead{}
		\rhead{\term \setlength{\unitlength}{\baselineskip}}
		
		
		
		
		\newpage
        %==================================%
        \subsection*{Question 1}
        % IEDS
        \begin{enumerate}
        	
        	\item[(a)] In terms of a strategic (matrix) game, define what is meant by a dominant strategy and a dominated strategy? Describe the technique of iterated elimination of dominated strategies. What is the technique used for ? \marks{5 \%}
        	\item[(b)] By removing all strategies which are dominated by strict pure or mixed strategies, derive a reduced version of the following 2-player matrix game:
        	
        	\begin{center}
        		
        		\begin{tabular}{|c|c|c|c|}
        			\hline
        			& S        &T       & U    \\
        			\hline
        			A & (5,3) & (1,2) & (2,1) \\
        			\hline
        			B &(1,-1)& (0,0)& (3,-3) \\
        			\hline
        			C & (2,-2) & (3,0)& (6,3) \\
        			\hline
        		\end{tabular}
        		
        		
        	\end{center} \marks{10 \%}
        	
        	\item[(c)] Find all the \textit{Nash} Equilibria of the above game.\marks{10 \%}
        	
        \end{enumerate}
        
        %===================================%
        \subsection*{Question 2}
        
        % game tree - extensive form
        \begin{itemize}
   
        \item In a game show, contestants M\'{a}ire and S\'{e}amus start the last round with \euro{500} and \euro{400} respectively. Each must decide to pass or play. If a player passes, they keep their money but if they opt to play they win or lose \euro{200} each with probability 1/2. These outcomes are independent of each other. The player with the most money at the end of the round gets a bonus of \euro{300}.
        \end{itemize}
        \begin{enumerate}
        	\item[(a)] If M\'{a}ire goes first and S\'{e}amus sees her move, draw the game tree. \marks{9 \%}
        	\item[(b)] Show that the strategic form of the game is
        	
        	\begin{center}
        		
        		\begin{tabular}{|c|c|c|}
        			\hline
        			& Pass         &Play       \\
        			\hline
        			Pass & (8,4) &$\left(\frac{13}{2}, \frac{11}{2}\right)$  \\
        			\hline
        			Play & $\left(\frac{13}{2}, \frac{11}{2}\right)$& $\left(\frac{29}{4}, \frac{23}{4}\right)$ \\
        			\hline
        		\end{tabular}
        	\end{center}
        	where payoffs are expected values in 00's. \marks{8 \%}
        	
        	\item[(c)] Solve the game. \marks{8 \%}
        \end{enumerate}
        
        
        
        %===================================%
        \subsection*{Question 3}
        
        % Duopoly
        

        \begin{enumerate}
        	\item[(a)] The costs incurred by a firm in a production period are $$ c = 50 + 2x$$ where $x$ is the number of items produced in that period. The items sell at a price of $$ p = 14- \frac{x}{50}$$ each. Find the level of production that maximises the firm's profits when the firm has a monopoly. \marks {6 \%}
        	\item[(b)]   If two identical firms supply the market with $x_i, \; i = 1,2$ items each at a cost per period of $$c_i = 50 + 2x_i$$ respectively and sell each item at a price of $$p = 14 - \frac{x_1 +x_2}{50},$$ analyse the resulting one shot \textit{Cournot} game. \marks{9 \%}
        	\item[(c)]  If this 2-firm game is to be played repeatedly, consider the following ``cooperative'' strategy: a firm produces half of the optimal level associated with a monopoly (see part (a) for as long as the other firm does the same, and if the other firm deviates, it reverts to the single shot \textit{Cournot} strategy thereafter. Does it ever pay to defect from the cooperative strategy? In particular, using the discount factor $\omega$ per period, when is this cooperative strategy a \textit{Nash} equilibrium ?  \marks{ 10 \%}
        \end{enumerate}
        
        
        
        \newpage
		
		%==================================%
		
		\subsection*{Question 4}
		
		%% Simplex Tableau
		
		\begin{itemize}
			
			
			
		
		\item[(a)]
		\begin{enumerate}[(i)]
			\item Big O-notation is used to classify algorithms according to their relative complexity. Compare the complexity of algorithms of order $\mathrm{O}(\log n)$, $\mathrm{O}(n)$, $\mathrm{O}(n\log n)$, $\mathrm{O}(2^n)$and $\mathrm{O}(n!)$. Illustrate your answer with a sketch. \marks{2 \%}
			%	\item Classify the Binary Search Tree algorithm using Big $\mathrm{O}$-notation. Justify your answer. \marks{2 \%}
		%	\item Compare and contract Big O-Notation, Big Omega Notation and Big Theta Notation. \marks{2 \% }
			
			\item What is meant by Combinatorial Explosion? Why is it relevant for Binary Integer Problems? \marks{2 \%  }
			
		\end{enumerate}	
			
			%------------------------------------------------------------%
			\item[(b)]  Consider the Integer Linear Program (IP):
			\begin{eqnarray*}
				\max 5 x_1 +4 x_2\\
				x_1+x_2&\le 5\\
				10x_1+6x_2 &\le 45\\
				x_1, x_2 \le 0 \, \text{and integer.}
			\end{eqnarray*}
			The corresponding Simplex Tableau (transforming the max problem into a min problem)  is:
			\begin{center}
				\begin{tabular}[h]{|l|rrrr|}\hline
					0&-5&-4&0&0\\\hline
					5&1&1&1&0\\
					45&10&6&0&1\\\hline
				\end{tabular}
			\end{center}
%			
%			{\bf N.B. The Simplex Method and the Dual Simplex Method are stated on the last page  of this paper.}
			
			\begin{enumerate}[(i)]
				\item Apply one iteration of the Simplex Method and show that the  Simplex Tableau now takes the form:\marks{4 \%}
				\begin{center}
					\begin{tabular}[h]{|l|rrrr|}\hline
						22.5&0&-1&0&0.5\\\hline
						0.5&0&0.4&1&-0.1\\
						4.5&1&0.6&0&0.1\\\hline
					\end{tabular}
				\end{center}
				\item After a second iteration of  the Simplex Method the  Simplex Tableau now takes the form: ({\bf N.B.do not perform the arithmetic!})
				\begin{center}
					\begin{tabular}[h]{|l|rrrr|}\hline
						23.75     &        0            & 0&          2.5&          0.25\\\hline
						1.25&             0&          1&          2.5&         -0.25\\
						3.75&          1&             0&         -1.5&          0.25\\\hline
					\end{tabular}
				\end{center}
				Explain why this Tableau is optimal.\marks{1 \%}
				\item The solution to the LP Relaxation of the IP is $x_1=3.75, x_2=1.25$. 		
				Suppose that we decide to branch on $x_1$. The two branches are $S_0: x_1 \le 3$ and $S_1: x_1 \ge 4$. 
				
				\medskip Consider the branch  $S_0: x_1 \le 3$.
				\begin{enumerate}
					\item First show that the basic variable $x_1$ may be expressed in terms of the non-basic variables $x_3$ \& $x_4$ as: $x_1 =3.75+1.5 x_3 -0.25 x_4$.\marks{2 \%}
					\item Substitute this expression for $x_1$ into the $S_0$ branch constraint and show that it takes the form $1.5 x_3-0.25 x_4 +s=-0.75$. (The variable $s$ is the slack variable for the constraint $x_1 \le 3$.)\marks{1 \%}
					\item Show that the  Simplex Tableau with the addition of this constraint takes the form:\marks{1 \%}
					\vspace{10pt}
					
					\begin{center}
						\begin{tabular}[h]{|l|rrrrr|}\hline
							23.75     &        0            & 0&          2.5&          0.25&0\\\hline
							1.25&             0&          1&          2.5&         -0.25&0\\
							3.75&          1&             0&         -1.5&          0.25&0\\
							-0.75&0&0&1.5 &-0.25&1\\ \hline
						\end{tabular}
					\end{center}
					
					\item Explain why this tableau is optimal but infeasible.\marks{1 \%}
					\item Apply {\bf one} iteration of the Dual Simplex Method to this tableau and show that  the  Simplex Tableau now takes the form:\marks{4 \%}
					\begin{center}
						\begin{tabular}[h]{|l|rrrrr|}\hline
							23            & 0 &            0 &         4         &    0&          1\\ \hline
							2    &         0 &         1&          1       &      0&         -1\\
							3     &     1  &           0 &            0&             0  &        1\\
							3      &       0   &          0     &    -6&          1   &      -4\\\hline
						\end{tabular}
					\end{center}
					\smallskip
					
					\item This tableau is LP optimal and integer feasible. Explain why. What is the solution to the IP?\marks{2 \%}
					\item Finally, {\bf suppose} that we had started with the branch $S_1: x_1 \ge 4$, expressed  $x_1$ in terms of the non-basic variables $x_3$ \& $x_4$ as $x_1 =3.75+1.5 x_3 -0.25 x_4$ as above and applied the Dual Simplex method to the resulting tableau.\\
					\smallskip
					We would have found  ({\bf N.B.do not perform the arithmetic!})
					\begin{center}
						\begin{tabular}[h]{|l|rrrrr|}\hline
							23.33       &      0&             0&             0&          0.67&          1.67\\ \hline
							0.83&             0&          1&             0&          0.17&          1.67\\
							4 &         1 &            0 &            0 &            0 &        -1\\
							0.17  &           0  &           0  &        1  &       -0.17  &       -0.67\\\hline
						\end{tabular}
					\end{center}
					\begin{enumerate}
						\item 
						This tableau is now optimal. Is the solution integer?\marks{1 \%}
						\item What would be the next branch \& bound step? \marks{1 \%} ({\bf N.B.do not perform the arithmetic!})
					\end{enumerate}
				      \item Finally, {\bf suppose} that we had started with the branch $S_1: x_1 \ge 4$, expressed  $x_1$ in terms of the non-basic variables $x_3$ \& $x_4$ as $x_1 =3.75+1.5 x_3 -0.25 x_4$ as above and applied the Dual Simplex method to the resulting tableau.
				      
				      We would have found  ({\bf N.B.do not perform the arithmetic!}) \smallskip
				      \begin{center}
				      \begin{tabular}[h]{|l|rrrrr|}\hline
				      	23.33       &      0&             0&             0&          0.67&          1.67\\ \hline
				      	0.83&             0&          1&             0&          0.17&          1.67\\
				      	4 &         1 &            0 &            0 &            0 &        -1\\
				      	0.17  &           0  &           0  &        1  &       -0.17  &       -0.67\\\hline
				      \end{tabular}
				      \end{center}
				      \smallskip
				      \begin{enumerate}
				      	\item 
				      	Is this tableau optimal?\marks{1\%}
				      	\item Is the solution integer?\marks{1\%}
				      	\item What would be the next branch \& bound step? \marks{1\%} ({\bf N.B.do not perform the arithmetic!})
				      \end{enumerate}
				\end{enumerate}
			\end{enumerate}

			
		\end{itemize}
		
		%================================================================================================%
		\subsection*{Question 5}
		% Algorithms
		\begin{itemize}

			%================================================================================%
			\item[(a)] 
			
			GAMMA Investments is considering investments into 6 projects: A, B, C, D, E and F.
			
			Each project has an initial cost, an expected profit rate (one year from now) expressed as a
			percentage of the initial cost, and an associated risk of failure.
			These numbers are given in the table below:
			
			\begin{center}
			\begin{tabular}{|c||c|c|c|c|c|c|}
				\hline  & A & B & C & D & E & F \\ 
				\hline Initial Cost & 1.8 & 1.5 & 1.1 & 1.8 & 2.1 & 3.2 \\ 
				\hline Profit Rate & 13\% & 12\% & 10\% & 12\% & 11\% & 9\% \\ 
				\hline Failure Risk & 6\% & 4\% & 5\% & 5.5\% & 4.5\%  & 4.5\%\\ 
				\hline 
			\end{tabular} 
		\end{center}
			\begin{enumerate}[(i)] \item  Provide a formulation to choose the projects that maximize total
				expected profit, such that GAMMA Investments does not invest more than
				5M dollars and its average failure risk is not over 5\%. 
				
				You may assuming equal weighting for each project when determining average risk. For example, if GAMMA Investments invests only into A,B and C, it invests
				only 4.2M dollars and its average failure risk is $(6\%+4\% +5\%)/3=5\%$.\marks{4 \%}
				
				\item Suppose that if C is chosen, D must be chosen. Modify your
				formulation.\marks{2 \%}
				\item Suppose that if E is chosen, F must not be chosen. Modify your
				formulation.\marks{2\%}
				\item   Suppose that if A and C are chosen, D must be chosen. Modify your
				formulation.\marks{2 \%}
			%	\item Suppose that only two projects, at most, can be chosed from A, B and C.  Modify your formulation.\marks{2 \%}
			\end{enumerate}

			\item[(b)] 
			\begin{enumerate}[(i)]
				\item Explain briefly why the following strategy for the solution of Integer
				Linear Programs (IP’s) is not useful: ``Solve the LP relaxation
				then round off the components of the solution to the nearest integers". \marks{3 \%  }
				\item Given an LP (the \emph{Primal} problem) we can write a closely related LP, its \emph{Dual}:
				\begin{eqnarray*}
					z &=& \max\{c^T x : Ax \le b,  x \in \mathbb{R}^n, x \ge 0 \}\quad \emph{Primal}\label{eq:lp-primal-1}\\
					w&=&    \min \{b^T  y : A^T y  \ge c,  y \in \mathbb{R}^m, y  \ge 0\}. \quad\emph{Dual}\label{eq:lp-dual-1}
				\end{eqnarray*}
				Prove the Weak Duality Theorem: for \emph{any} primal feasible
				point $x$ and \emph{any} dual feasible point $y$, $ b^T y \ge c^T x$.\marks{4 \%}
			\end{enumerate}	

			%------------------------------------------------------------%
			\item[(c)]
			\begin{enumerate}[(i)]

				
							\item Provide a short description of the Dynamic Programing Paradigm.  \marks{4 \%}
							\item What is a Greedy Algorithm? Support your answer with a simple example, and discuss the advantages and disadvantages of using Greedy Algorithms.   															\marks{2 \%}
								\item In the context of the design of algorithms, describe the Divide and Conquer paradigm. \marks{2 \%  }
			\end{enumerate}
			%------------------------------------------------------------%

		\end{itemize}
		
		%================================================================================================%
\newpage
				\subsection*{Question 6}
				
				%% Solution Grid Tableau
				
				\begin{itemize}
					%------------------------------------------------------------%
					\item[(a)] 
					Maximize P = 3x+2y
			
					\begin{tabular}{cl}
						
						\phantom{space}
						\phantom{space} & 12x - 18y $\leq$ 90\\
						& 20x + 13y $\leq$ 252\\
						& y $\leq$ 10\\
			
						& x,y $\geq$ 0\\
						& x,y integers\\
						
					\end{tabular} 
					
					\smallskip
					This Integer Program is to be solved using the tabular Branch and Bound method.
					
	\medskip
					Use the solution grids below to solve the problem. Each node is referenced by its tree level, ordered from left to right so that the annotation {\bf Node XY} is the node at level {\bf X} at position {\bf Y} where ${\bf Y}=A$ is the left-most position in level {\bf X}, where ${\bf Y}=B$ is the 2nd from the left in level {\bf X}, and so on. 
					
					{\bf You must draw an enumeration tree/diagram to keep track of your progress. Draw the enumeration tree on an otherwise blank page.}
					
					\marks{25\%}
					
					
					
					
					
				\end{itemize}
				
		\normalsize	
		\begin{tabular}{||c|c||c|l||c|l||}
			\hline  & Node 0 &   & Node 1A &  & Node 1B     \\  \hline
			\hline (i) & $x=7.50$, $y=9.00$ & (i) & $x=6.50$, $y = 8.50$ &  (i)  & $x=7.61$, $y=7.00$\\  \hline
			\hline (ii)  & $x=6.10$, $y =10.00 $  &  (ii) & $x=6.00$, $y = 8.91$ & (ii) & $x=7.60$, $y = 8.00 $ \\  \hline
			\hline (iii) & $x=7.00$, $y =9.00$  & (iii) & $x=6.91$, $y = 7.61 $ & (iii) & $x=6.70$, $y = 6.50$ \\  \hline
			\hline (iv)  & $x=6.00$, $y = 10.50$  &  (iv) & $x=7.00$, $y = 10.00$  & (iv)  & $x=6.91$, $y=8.25$\\  \hline
			\hline (v) & $x=7.00$, $y = 8.50$ & (v)  & $x=6.00$, $y=11.00$ & (v) & $x=7.00$, $y = 8.61$\\  \hline & & & & & \\
			\hline 
			
			\hline  & Node 2A &   & Node 2B &  & Node 2C  \\  \hline
			\hline (i) & $x=7.00$, $y=9.75$ & (i)  & $x=6.25 $, $y = 8.30 $  & (i)  & $x=7.00$, $y = 6.75$\\  \hline
			\hline (ii)  & $x=7.25$, $y = 8.50$  &  (ii) &$x=7.50$, $y = 6.50$ & (ii) & $x=6.75$, $y = 8.00 $ \\  \hline
			\hline (iii) & $x=6.75  $, $y = 10.00 $  &  (iii) &$x=6.25 $, $y = 5.61 $ &  (iii) & $x=7.40$, $y = 8.00$ \\  \hline
			\hline (iv)  & $x=8.50$, $y = 6.61 $  & (iv) & $x=8.60 $, $y = 6.61 $ & (iv)  & $x=7.80$, $y =8.00$\\  \hline
			\hline (v) & $x=8.25 $, $y = 8.25 $ & (v) & $x=8.00$, $y=9.00$  &(v) & $x=6.61$, $y = 9.61$\\  \hline & & & & & \\
			\hline 
		\end{tabular} 
		\newpage	
		\begin{tabular}{||c|c||c|c||c|c||}
			
			
			\hline  & Node 2D &   & Node 3A &  & Node 3B   \\  \hline
			\hline (i) & $x=6.90$, $y=8.25$ &  (i) & $x=7.125$, $y = 7.75$  & (i)  & $x=6.30$, $y=6.61$\\  \hline
			\hline (ii)  & $x=7.00$, $y=8.50$  &  (ii) & $x=6.50$, $y = 7.50$ & (ii) & $x=7.25$, $y =5.50$ \\  \hline
			\hline (iii) & $x=7.00$, $y = 8.40$  & (iii) & $x=7.40$, $y = 6.61 $ &(iii) & $x=7.25$, $y=7.61$ \\  \hline
			\hline (iv)  & $x=7.00$, $y=7.90$  & (iv) & $x=6.50$, $y = 7.91$ & (iv)  & $x=7.10$, $y=8.50$\\  \hline
			\hline (v) & $x=7.50$, $y = 7.10$ & (v) & $x=6.90$, $y = 7.125 $  &(v) & $x=6.00$, $y=7.25$\\  \hline & & & & & \\
			\hline 
			%	\end{tabular} 
			%	
			%	\begin{tabular}{||c|c||c|c||c|c||}
			\hline  & Node 3C &   & Node 3D &  & Node 3E  \\  \hline
			\hline (i) & $x=9.20$, $y =6.10$ &  (i) & $x=8.25$, $y = 8.60$  & (i)  & $x=7.00$, $y = 8.00$\\  \hline
			\hline (ii) & $x=5.75$, $y=8.50$  & (ii) & $x= 7.40$, $y = 9.50 $ &(ii) & $x=7.75$, $y =6.80$ \\  \hline
			\hline (iii)  & $x=8.50$, $y= 7.61$  & (iii) & $x=6.00$, $y =8.00$ & (iii)  & $x=7.00$, $y = 8.50$\\  \hline
			\hline (iv)  & $x=8.25$, $y = 7.61$  &  (iv) & $x=7.25$, $y =9.333$ & (iv) & $x=6.50$, $y = 8.00$ \\  \hline
			\hline (v) & $x=8.60$, $y=9.20$ & (v) & $x=9.00 $, $y = 8.60$  &(v) & $x=7.00$, $y =7.00$\\  \hline & & & & & \\
			\hline 
			%		
			%	\end{tabular} 
			%	
			%	\begin{tabular}{||c|c||c|c||c|c||}
			\hline  &  Node 3F &  & Node 3G &   & Node 3H \\  \hline
			\hline (i) & $x=9.00$, $y = 7.00$ & (i) & $x=7.40$, $y = 11.00$  & (i)  & $x=7.60$, $y = 7.80$\\  \hline
			\hline (ii)  & $x=8.00$, $y =7.25 $  & (ii) & $x=7.50 $, $y=10.50$ & (ii) & $x=8.00$, $y = 8.00$ \\  \hline
			\hline (iii) & $x=7.00$, $y = 7.50 $   & (iii) & $x=7.40$, $y=8.50$ &(iii) & $x=8.30$, $y = 7.50$ \\  \hline
			\hline (iv)  & $x=8.00$, $y = 10.00$  & (iv) & $x=7.30$, $y=7.50$ & (iv)  & $x=7.60 $, $y = 6.60$\\  \hline
			\hline (v) & $x=8.00$, $y = 7.00$ & (v)  & $x=7.90$, $y =6.50 $ & (v) & $x=7.70$, $y = 7.50$\\  \hline
			\hline 
			& & & & & \\ \hline

				\end{tabular} 
			\end{document}
			
