
The complexity of software application is not measured and is not written in big-O notation. It is only useful to measure algorithm complexity and to compare algorithms in the same domain. Most likely, when we say O(n), we mean that it's "O(n) comparisons" or "O(n) arithmetic operations". 

O(1) = simple operation
O(n) = adding n items
A typical example of O(N log N) would be sorting an input array with a good algorithm 
O(N^n) = Monkey Sort
O("^n) = combinatorial explosion
O(Log_2(n) = Binary Search


O(1) - most cooking procedures are O(1), that is, it takes a constant amount of time even if there are more people to cook for (to a degree, because you could run out of space in your pot/pans and need to split up the cooking)

O(logn) - finding something in your telephone book. Think binary search.

O(n) - reading a book, where n is the number of pages. It is the minimum amount of time it takes to read a book.


%======================================%
\subsection{Greedy Alogorithm}
A greedy algorithm is an algorithmic paradigm that follows the problem solving heuristic of making the locally optimal choice at each stage with the hope of finding a global optimum.

A greedy algorithm is a mathematical process that looks for simple, easy-to-implement solutions to complex, multi-step problems by deciding which next step will provide the most obvious benefit.

Such algorithms are called greedy because while the optimal solution to each smaller instance will provide an immediate output, the algorithm doesn’t consider the larger problem as a whole. Once a decision has been made, it is never reconsidered.
%======================================%
