\documentclass{beamer}

\usepackage{amsmath}
\usepackage{amssymb}

\begin{document}


\begin{frame}
\Large
\[
\mbox{Continuous Compounding}
\]
\end{frame}

%------------------------------------------------%
%-------------------------------------------------------%

\begin{frame}
\frametitle{Information Ratio - IR}
A ratio of portfolio returns above the returns of a benchmark (usually an index) to the volatility of those returns. 
The information ratio (IR) measures a portfolio manager's ability to generate excess returns relative to a benchmark, but also attempts to identify the consistency of the investor. This ratio will identify if a manager has beaten the benchmark by a lot in a few months or a little every month. The higher the IR the more consistent a manager is and consistency is an ideal trait.


Information Ratio (IR)

\begin{itemize}
$R_p$ = Return of the portfolio
$R_i$ = Return of the index or benchmark
\end{itemize}

Sp-i = Tracking error (standard deviation of the difference between returns of the portfolio and the returns of the index)  
\end{frame}

%-------------------------------------------------------%

\begin{frame}
\frametitle{Information Ratio - IR}
Investopedia explains 'Information Ratio - IR'
A high IR can be achieved by having a high return in the portfolio, a low return of the index and a low tracking error. 
\end{frame}

%-------------------------------------------------------%

\begin{frame}
\frametitle{Information Ratio - IR}
For example: 
Manager A might have returns of 13\% and a tracking error of 8\% 
Manager B has returns of 8\% and tracking error of 4.5\%
The index has returns of -1.5\%
Manager A's IR = [13-(-1.5)]/8 = 1.81
Manager B's IR = [8-(-1.5)]/4.5 = 2.11

Manager B had lower returns but a better IR. A high ratio means a manager can achieve higher returns more efficiently than one with a low ratio by taking on additional risk. Additional risk could be achieved through leveraging.  
\end{frame}




\end{document}

Maths Resource

http://www.csusm.edu/mathlab/documents/M132BusCalcFormulas%20r1-12e.pdf
http://www.math.ubc.ca/~chau/elasticity.pdf
http://www.textbooksonline.tn.nic.in/books/12/std12-bm-em-1.pdf
