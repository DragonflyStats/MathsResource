MODULE CODE: MA4104 SEMESTER: Spring 2014/15

MODULE TITLE: Business Statistics DURATION: 2.5 hours

LECTURER: Dr. Helen Purtill

EXTERNAL Prof. Adele Marshall

EXAMINERS:

INSTRUCTIONS TO CANDIDATES:

* Answer all 4 questions.

%===============================================================%
\newpage
\subsection*{Question 1}

Q1. (a) The manager of a shopping centre wants to profile Sunday shoppers at the centre.

Three questions from the questionnaire given to a random sample of shoppers are as follows:

\begin{framed}
Q1. How often do you visit this shopping centre on Sundays?

Every week ?

Once a month ?

A few times a year ?

Q2. How much do you spend on average per visit?

Q3. Rate your opinion of the facilities offered by this centre on a scale of 0 to 10 where 0=very poor and 10=excellent.
\end{framed}

Classify the data generated for each of the above questions by data type and scale of

measurement. (6 marks)

(b) What is a simple random sample? What are the advantages of taking a simple

random sample? (2 marks)

(c) The sales of a product are normally distributed with a mean of 200 units per week and a standard deviation of 40 units per week.

\begin{itemize}
\item[(i)] What is the probability that more than 250 units will be sold in any given week?

\item[(ii)] What is the probability that between 220 and 260 units will be sold in any given week?

\item[(iii)] To have a 98% probability that the company will have sufficient stock to cover the weekly demand, how many units should be produced?

\item[(iv)] Use a control chart to identify which of the following weekly sales figures (in units) were unusually high or unusually low: 280, 330, 50, 310, 285, 65.

\end{itemize}
%---------------------------------%
(10 marks)

(Question 1 is continued on overleaf)

(c) A mobile phone company wanted to find out how its customers rated the customer service offered by its call centre operators. A list of all customers was obtained and 500 customers were randomly selected from the list. These 500 customers were posted a questionnaire asking them to rate customer service on a scale of 0 to 10. Customers who responded were given €10 credit on their account. 250 customers responded and gave a mean rating of 5 for customer service.

\item[(i)] For this example, identify the population, the sampling frame, the sample and the variable measured.

\item[(ii)] What is the main parameter of interest in this example?

\item[(iii)] What is the best estimate of this parameter?

\item[(iv)] Describe the potential bias in this example.

(7 marks)

%===============================================================%
\newpage
\subsection*{Question 2}

Q2. (a) The mean hourly wage in an EU country is €10. A sample of 55 individuals in the capital city of the country has a mean hourly wage of €10.83 with a standard deviation of €3.25 per hour.

\item[(i)] Calculate a 95% confidence interval for the mean hourly wage in the capital city. Interpret this interval.

\item[(ii)] Test the hypothesis that workers in the capital city earn more than the mean hourly wage for the country using a 5% level of significance. Clearly state your null and alternative hypotheses and your conclusion. Give a p-value for this hypothesis test and interpret this p-value

(10 marks)

(b) A retail business would like to estimate the proportion of gift vouchers sold by the business that expire without being used by the customer. The accounts department selects a random sample of 250 gift vouchers from a sales database and finds that 120 expired without being used.

\item[(i)] Construct and interpret a 95% confidence interval for the proportion of gift vouchers that expire without being used.

\item[(ii)] Using your answer to (b) \item[(i)], is there evidence that at least 40% of the gift vouchers sold by the business expire without being used?


(c) A marketing manager working in a large retail business would like to determine if a text messaging service alerting customers to special offers has been successful in increasing sales. The manager selects a random sample of 20 customers from the business’s loyalty card database and computes the amount spent by each customer for the month before and the month after the introduction of the text messaging service. The computed mean difference (After – Before) for the sample was €29.34 with the standard deviation of the differences was €21.21.

Have sales increased following the introduction of the text messaging service? Test this hypothesis using a 5% significance level.

(8 marks)

%===============================================================%
\newpage
\subsection*{Question 3}

Q3. (a) A business would like to compare customer ratings (on a scale of 0-10, where 0 = poor and 10 = excellent) of two new products (A and B). A sample of 80 customers were randomly assigned to rate either Product A or Product B. The sample data are summarised as:

Product A Product B

Sample size = 40 Sample size = 40

Mean = 7.2 Mean = 6.1

Standard Deviation = 2.3 Standard Deviation = 3.8

\item[(i)] At the 5% significance level, test whether product ratings differ between Product A and Product B. Interpret the results.

\item[(ii)] Compute and interpret a 95% confidence interval for the difference in mean product ratings between product A and product B.

(13 marks)

(b) An accountant is interested in investigating whether the number of loans defaulting within one year in a credit institution has significantly changed from 2013 to 2014. He selects a random sample of loans from the loan book and summarises the data in the following table.

Year Total 2013 2014

Default No 115 85 200 Yes 20 7 27

Total 135 92 227

\item[(i)] What is the percentage of loans taken out in 2014 that have defaulted?

(2 marks)

\item[(ii)] Test whether there is an association between year and default, stating clearly the null and alternative hypotheses. What conclusions would you make following the analysis? (Note: ) (8 marks)

\item[(iii)] What condition must be satisfied for the Chi-square test to be valid? Do the data in this question satisfy this condition? Justify your answer.

(2 marks)

