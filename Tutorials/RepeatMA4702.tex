 \documentclass[a4paper,12pt]{article}
%%%%%%%%%%%%%%%%%%%%%%%%%%%%%%%%%%%%%%%%%%%%%%%%%%%%%%%%%%%%%%%%%%%%%%%%%%%%%%%%%%%%%%%%%%%%%%%%%%%%%%%%%%%%%%%%%%%%%%%%%%%%%%%%%%%%%%%%%%%%%%%%%%%%%%%%%%%%%%%%%%%%%%%%%%%%%%%%%%%%%%%%%%%%%%%%%%%%%%%%%%%%%%%%%%%%%%%%%%%%%%%%%%%%%%%%%%%%%%%%%%%%%%%%%%%%
\usepackage{eurosym}
\usepackage{vmargin}
\usepackage{amsmath}
\usepackage{multicol}
\usepackage{graphics}
\usepackage{enumerate}
\usepackage{epsfig}
\usepackage{framed}
\usepackage{subfigure}
\usepackage{fancyhdr}

\setcounter{MaxMatrixCols}{10}
%TCIDATA{OutputFilter=LATEX.DLL}
%TCIDATA{Version=5.00.0.2570}
%TCIDATA{<META NAME="SaveForMode" CONTENT="1">}
%TCIDATA{LastRevised=Wednesday, February 23, 2011 13:24:34}
%TCIDATA{<META NAME="GraphicsSave" CONTENT="32">}
%TCIDATA{Language=American English}

%\pagestyle{fancy}
\setmarginsrb{20mm}{0mm}{20mm}{25mm}{12mm}{11mm}{0mm}{11mm}
%\lhead{MA4413 2013} \rhead{Mr. Kevin O'Brien}
%\chead{Midterm Assessment 1 }
%\input{tcilatex}

\begin{document}

\begin{center}
       \includegraphics[scale=0.60]{shieldtransparent2}
\end{center}

\begin{center}
\vspace{1cm}
\large \bf {FACULTY OF SCIENCE AND ENGINEERING} \\[0.5cm]
\normalsize DEPARTMENT OF MATHEMATICS AND STATISTICS \\[1.25cm]
\large \bf {REPEAT EXAMINATION} \\[1.5cm]
\end{center}

\begin{tabular}{ll}
MODULE CODE: MA4702 & SEMESTER: Autumn 2015\\[1cm]
MODULE TITLE: Technology Mathematics 2 & DURATION OF EXAM: 2.5 hours \\[1cm]
LECTURER: Kevin O'Brien & GRADING SCHEME: 100 marks\\
 & \phantom{GRADING SCHEME:} \footnotesize {70\% of total module marks}   \\[0.8cm]
EXTERNAL EXAMINER: Prof. John King & \\[1cm]
\\[1cm]
\end{tabular}
\begin{center}
{\bf INSTRUCTIONS TO CANDIDATES}
\end{center}

{\noindent \\ This paper is comprised of six questions. Question 1 is compulsory and is worth 40 Marks.  You must also attempt any four of the other five questions, each of which are worth 15 marks.
\\ Scientific calculators approved by the University of Limerick can be used. 
\\ Formula sheet and statistical tables are provided.
}
\normalsize
\newpage
%\section*{Attempt ALL questions}

\subsection*{Question 1}
%========================================================%
% Short Questions
% 1 - Inverse
% 2 - Domain and Range
% 3 - Vertical Asymptotes
% 4 - Horizontal Asymptotes
% 5 - Limit ( Indeterminate Answer - Factorize)
%======================================================%
\begin{enumerate}[(i)]
	
	\item Determine the vertical asymptotes of the following function
	
	\[  f(x)  = \frac{2x-3}{x+7} \]
	
	
	
	\item Determine the horizontal asymptotes of the following function
	
	\[  f(x)  = \frac{2x-3}{x+7} \]

	
	\item Solve the following limit:
	\[\lim_{x \to 1 } \frac{x^2-4x +3}{x-1}\]
	
	\item Find the domain and the range of the function:
	\[ f(x) = \frac{4x}{2x+2} \]

 \item Find $f^{-1}(x)$ the inverse of the function
 \[f(x) = \log_e{4x} \]
 
% answer = 2

%======================================================%
% Short Questions
% 6 - Indefinite Integral
% 7 - Definite Integral
% 8 - Partial Derivatives
% 9 - Convergence
% 10 - Sum of a Series
%======================================================%


\item Evaluate the following indefinite integral:
\[ \int (\cos(4x)+e^{2x}) dx   \]
\item Evaluate the following definite integral:

\[  \int^{2}_{0}  x^2+3x+1 dx  \]

\item Determine both of the first order partial derivatives with respect to $z$ of the following expression.

\[ z = \frac{y^2x^3}{3} +  2y \sin(x)\]
 

\item Find $\displaystyle{ \frac{ \partial^2 z }{ \partial^2 x  }}$ and $\displaystyle{ \frac{ \partial^2 z }{ \partial^2 y }}$  for the following function
	\[ z = xy^2 + 4x^3y^2 + 5x^2.\]	

	
	

\item Compute the following summation

\[ \sum_{i=21}^{40} i \]

%item What does the following sequence converge to?
% \[\] u_n  = \frac{n+3}{4n+2}\]
\end{enumerate}

\newpage
%========================================================================================= %
%========================================================================================= %
\subsection*{Question 2 - Limits and Functions}
\subsubsection*{Part A - Limits}
\begin{enumerate}
	\item[(i)] (2 Marks)  Compute the limit of the following function.
	
	\[\lim_{x \to 6 } \frac{x^2 + 2x-10}{x-4}.\]
		
		\bigskip
	
	\item[(iii)] (3 Marks) Compute the limit of the following function.
	\[ \lim_{x \to \infty } \frac{3 + x^2 - 6x^3 }{3x^3 - 5x + 7}. \]	
	
\end{enumerate}

\subsubsection*{Part B - Functions}
\begin{enumerate}[(i)]
	\item (3 Marks) Determine if the function $f(x) = x^3\sin(x)$ is an even function, an odd function or neither.
	\bigskip
	\item (2 Marks) Given the functions $g(x) = x^2+1$ and $f(x) = 1-3x$ determine expressions for $f \circ g(x)$ and $g \circ f(x)$.
\end{enumerate}
	\bigskip
\subsubsection*{Part C - Hyperbolic Functions}


\begin{enumerate}[(i)]
	\item (5 Marks) Using their definition in terms of exponentials, prove the following hyperbolic identity  \[ \mbox{Cosh(x+y)} = \mbox{Cosh(x)Cosh(y)} + \mbox{Sinh(x)Sinh(y)}.\]
%	\item Show that \[Cosh^2 x = Cosh2x + Sinh2x\]	
%	\item Show that \[ cosh(x+y) = cosh(x)cosh(y) + sinh(x)sinh(y)\]
	
\end{enumerate}
%========================================================================================= %
%========================================================================================= %
\newpage
\subsection*{Question 3 - Curve Sketching}
The concentration of a drug in a patient's bloodstream 7 hours
after it was injected is given by
\[ A(h) = \frac{0.18h}{h^2 + 3} \]
	
	\bigskip
\begin{enumerate}[(i)]
\item (4 Marks) Find the axis intercepts of $A(h)$.
	
	\bigskip
\item (5 Marks) Find and classify the critical points of $A(h)$ as local maxima or
local minima.
	
	\bigskip
\item (3 Marks) Determine the behaviour of $A(h)$ as $h \rightarrow + \infty$.
	
	\bigskip
	\item (3 Marks) Sketch the graph of $y = A(h)$ for $h \geq 0$ illustrating clearly
the features of the curve obtained in parts (i - iii).
	
	

\end{enumerate}


%========================================================================================= %
%========================================================================================= %
\newpage
\subsection*{Question 4 - Sequences and Series}
% Sequences and Series

% 4 Marks - Geometric Series
% 3 Marks - Arithmetic Series
% 3 Marks - Telescoping Series
% 3 Marks - Ratio Test
% 2 Marks - Repeating Decimal 

\begin{enumerate}[(i)]


	\item (3 Marks)	Three consecutive terms of an arithmetic series are \[7x-22, 3x+2 , 5x-4. \]
	Find the value of $x$.
	%answer is 460
	

\bigskip
	\item (4 Marks)
	The second term $u_3$ of a geometric sequence is 24. The third term $u_4$ is -72. \\ \bigskip Answer the following questions. Both questions are worth 2 Marks each.
	\begin{itemize}
		\item[(a)] Find the common ratio $r$. 
		\item[(b)] Find the first and fourth term $u_1$ and $u_5$.
	\end{itemize}
	
	
	\bigskip	
	\item (3 Marks) Suppose that the following term is the general term for a series. Use the Ratio Test to test this series for convergence
	
	\[u_n=\frac{n!n!}{(2n)!}\]

\bigskip
	
%		%------------------------%
%		\item (3 Marks) Find the sum of the telescoping series  \[ \sum^{\infty}_{n=1} \frac{2}{(2n-1)(2n+1)}\]

\bigskip
		
	\item (2 Marks) Express the following repeating decimal number as a simple fraction. Show your workings.
	
	\[0.5151515\ldots\]
	\item (2 Marks) Find the sum of the following telescoping series
	\[  \sum^{\infty}_{n=1}   \frac{6}{(3n+1)(3n+4)}  \]
%	
%	
	
	
	
\end{enumerate}

%========================================================================================= %
%========================================================================================= %
\newpage
\subsection*{Question 5 - Integration}


%Integration
%Substitution Technique
%Area Between Curves
%Definite Integrals

% - http://media.wix.com/ugd/b064dd_62ab7d7000c34b7eb4a8e2108269cf27.pdf

\begin{enumerate}[(i)]
	\item (2 Marks)  Evaluate the indefinite integral
	%\[ \int^2_1 x^2 + \sqrt{x} + \cos(5x) dx. \]
	 \[ \int cos x(1 + sin x)^3 dx \]
	%---------------------------%
	\item (3 Marks) By finding a good substitution, evaluate the definite integral

  \[ \int^2_1 \frac{8x + 3}{4x^2 + 3x + 3} dx.\]
	
	\item (3 Marks) By finding a good substitution, evaluate the indefinite integral
	% - \[ \int 2x(x^2 + 2)^9 dx.\]
	 \[ \int x(x^2 - 1)^5 dx\]
	% - \[ \int 2x cosh(3x^2+2) dx \]
	% - \[ \int \frac{12x + 14}{3x^2 + 7x + 3} dx\]
	%---------------------------%
	\item (3 Marks) Use integration by parts to evaluate the indefinite integral
	\[\int x cos(x) dx\]
	% Use integration by parts to find \[ \int x cos(x) dx \]
	%---------------------------%
	\item (4 Marks) By first performing a partial fraction expansion (that is, by writing the integrand
	as follows)
	\[  \frac{A}{x + 2} + \frac{B}{x + 3},
	\]evaluate the definite integral

	\[ \int \frac{5x-1}{(x + 2)(x + 3)}  dx. \]

\end{enumerate}

\newpage
%========================================================================================= %
%========================================================================================= %
\subsection*{Question 6  - Applications of Integration and Partial Derivatives}

\subsubsection*{Part A - Applications of Integration}
\begin{enumerate}[(i)]
%	\item (5 Marks) Find the area enclosed by the curves $y=x^2-6x+5$ and $y=x-5$.
\item (5 Marks) Find the area enclosed by the curve $y = 3x^2 − 4x + 1$ and the x axis.
%\item (5 Marks) A current $i(t) = 4 + 6\cos(2t)$ passes through a capacitor at time $t$.
%The capacitor is uncharged at time $ t = 0 $. Find the charge $q(t)$ at all times $t$.
\item (5 Marks) A moving object has acceleration
$a(t) = 3 + 5 sin t$ at time t. It starts from rest at time $t = 0$. Find its velocity at all time t.
Also, find its velocity at time $t = 4$.
\end{enumerate}



%\item Use Simpsons with 5 equal subintervals to find an approximation
%for
\bigskip
\subsubsection*{Part B - Partial Derivatives}
\begin{enumerate}[(i)]

 \item (5 Marks)
 Show that the function $z=e^{-2t}\cos(x)$ satisfied the partial differential equation
 \[ 4\frac{\partial^2 z}{\partial x^2} + \frac{\partial^2 z}{\partial t^2} = 0. \]
\end{enumerate}
%
%\subsubsection*{MacLaurin Series}
%\begin{enumerate}[(i)]
%	\item	Find the Maclaurin series of $e^x$ up to an including the term containing $x^4$
%	
%	%\item Evaluate all the first partial derivatives, i.e. $\frac{\partial z}{\partial x}$ and $\frac{\partial z}{\partial y} $, 
%	%of the following function:
%	%
%	%\[ z = 2xy^3 + cos(2x) - sin(y) \]
%
%
%
%	\item
%\end{enumerate}

\newpage
	
	\section*{Formula Sheet}
	
	\subsection*{Logarithms}
	If $a^b = c$ then $\mbox{log}_a c = b$.
	\subsection*{Sum and Difference of Two Cubes}
	\[ a^3 + b^3 = (a-b)(a^2 - ab + b^2)\]
	\[ a^3 - b^3 = (a-b)(a^2 + ab + b^2)\]
	
	%======================================== %
	
	\subsection*{Sequences and Series}
	
	\[ \sum_{i=1}^{n} i = \frac{n(n+1)}{2}\]
	
	Arithmetic Series Summation:
	\[ S_n = \frac{n}{2} \left(2a + (n-1) d \right)\]
	
	Geometric Series Summation:
	\[ S_n = a\left(\frac{1-r^n}{1-r}\right)\]
	
	\[ S_\infty = \frac{a}{1-r}\]

	\subsection*{Ratio Test}
	
	For a series with general term $u_n$, if
	
	\[ \lim_{n \to \infty } \left| \frac{u_{n+1}}{u_n} \right| = r\]
	then the series converges (absolutely) if $r<1$


	%==========================================================================================%
	
	
\subsection*{Curve Sketching}
\begin{description}
	\item[Horizontal Asymptote:] The horizontal asymptote is computed as
		\[ \lim_{x \to \infty } f(x) \]
\end{description}
	%==========================================================================================%
	
			\subsection*{Maclaurin Series}
			\[f(x) = f(0) + f^{\prime}(0) + \frac{f^{\prime \prime}(0)}{2!} + \frac{f^{\prime \prime}(0)}{2!} + \frac{f^{\prime\prime \prime}(0)}{3!} + \ldots \]
	%==========================================================================================%
	
	\subsection*{Hyperbolic Functions }
	
	\[ \cosh(x)  =  \frac{e^{x} + e^{-x}}{2} \]
	
	\[ \sinh(x)  = \frac{e^{x} - e^{-x}}{2} \]
	
	%==========================================================================================%
	
	\subsection*{Rules of Differentiation}
	%% - http://media.wix.com/ugd/b064dd_62ab7d7000c34b7eb4a8e2108269cf27.pdf
\begin{description}
\item[Product Rule:]  with $y = uv$
	
	
	\[ \frac{dy}{dx} = u \frac{dv}{dx} +  v \frac{du}{dx} \]
	
\item[Quotient Rule:] \[ y = \frac{u}{v}\]
	\[ \frac{dy}{dx}  = \frac{v \frac{du}{dx} - u \frac{dv}{dx} }{v^2} \]
	
	
	
\item[Chain Rule:]
	
	% y = f(u) and u = u(x), that is y = f(u(x)) ⇒ dy
	% dx
	\[ \frac{dy}{dx} = \frac{dy}{du} \times \frac{du}{dx}  \]
\end{description}
	%==========================================================================================%
	
	\subsection*{Integration}
	
	Integration by parts: 
	
	\[ \int u dv = uv - \int v du \]  
	
	Further formulae and special cases on pages 41 \& 42 of the log tables provided.

\subsection*{Dynamics}
Where $s(t)$ denotes displacement at time $t$, $v(t)$ denotes the velocity at time $t$ and $a(t)$
denotes the acceleration at time $t$, 
\[  \frac{ds(t)}{dt}  = v(t),\]
\[  \frac{dv(t)}{dt}  = a(t).\]

\subsection*{Electrical Circuits}
Where $q(t)$ denotes the charge at time $t$ and $i(t)$ denotes the current at time $t$,
\[  \frac{dq(t)}{dt}  = i(t).\]



\end{document}
