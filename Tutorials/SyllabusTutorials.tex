\documentclass[]{article}
\voffset=-1.5cm
\oddsidemargin=0.0cm
\textwidth = 470pt

\usepackage{amsmath}
\usepackage{amssymb}
%=========================================================================================== %

\begin{document}
\begin{itemize}
\item Define the domain and range of a function and define and plot simple inverse trigonometric and hyperbolic functions. 
\item Sketch curves using properties such as symmetry, intercepts, discontinuities, turning points and asymptotic behaviour. 
\item Sum arithmetic, geometric and telescoping series; test series for convergence; find the Maclaurin series of a function; manipulate power series; use lHopitals rule.
\item Integrate standard functions using substitution and parts; Apply to calculation of areas and volumes. 
\item Integrate numerically using Simpsons rule. 
\item Find partial derivatives of functions of two variables as well as higher partial derivatives; apply to analysis of small errors.
\end{itemize}

\newpage
\section{Week 3 Fundamentals and Functions}
\begin{enumerate}
	\item Exponentials and Powers
	\item Logarithms
	\item Factorizations
	\item Number Types (natural numbers, real numbers, integers, rational numbers)
\end{enumerate}
\newpage
%========================================================================= %
\section{Week 4 Functions}
\begin{enumerate}
	\item Domain, Codomain and Range
	\item One to One and Onto Functions (using Arrow Diagrams)
	\item Special Functions
	\item Inverting a Function
	\item hyperbolic Functions \\ show that \[ cosh^2(x) - sinh^2(x) = 1\]
	\[ cosh(x+y) = cosh(x)cosh(y) + sinh(x)sinh(y)\]
	% %- http://www.sosmath.com/trig/hyper/hyper01/hyper01.html
\end{enumerate}

\begin{itemize}
\item[1] Find $f^{-1}(x)$ the inverse of the function 
\[f(x) = \frac{1}{2x-5}\]
\item[2] Check whether the following functions are even, odd or neither.
\begin{itemize}
	\item[(i)] \[f(x) = \frac{4}{x2+1}\]
	\item[(ii)] $f(x) = sin(4x) $
	\item[(iii)] $f(x) = -cos(3x) $
	\item[(iv)] \[f(x) = \frac{3x+2}{4x+3} \]
\end{itemize}
\end{itemize}

\newpage
\section{Week 6 Sequences and Series}
% % Following Questions LC 2009
\begin{enumerate}
\item Three consecutive terms of an arithmetic series are \[4x +11, 2x +11, 3x +17 \].
Find the value of x.
\item Find the sum of the first 10 numbers of this arithmetic series: $1, 11, 21, 31 , \ldots$
%answer is 460

\item Find the sum of the following geometric series: 
\[3 + 6 + 12 + 24 + \ldots + 3072\]

\item  Answer the following Questions
\begin{itemize}
% % Revise Cross Multiplication
\item[(i)] Show that , where $r \neq \pm 1$.
\[ \frac{2}{r^2+1} =  \frac{1}{r+1} + \frac{1}{r-1} \]
\item[(ii)] Hence, find the following summation


\[  \sum^{n}_{r=2} \frac{2}{r^2+1} \]
\item[(iii)] Hence, evaluate the following summation
\[  \sum^{n}_{r=2} \frac{2}{r^2+1} \]
\end{itemize}

\end{enumerate}

Consider the function $ y = f(x) = x^4 - 6x^2 + 10$
%======================================================================================================= %
\newpage
\section{Week 6 Curve Sketching}
\begin{itemize}
	\item[Ex. 1]
\begin{itemize}
	\item[(i)] Find the $y$ intercept of the function $y=f(x)$.
	\item[(ii)] Show that $(\sqrt{3},1)$ is a stationary point of the function. Find the other two stationary points and classify all
	three points as local maxima or minima.
	\item[(iii)] Find the two inflection pints of $f(x)$.
	\item[(iv)] Find the $x$ values for which is $y=f(x)$ is concave up/down.
	\item[(v)] Determine the behaviour of $y$ as $x \rightarrow + \infty$ and as $x \rightarrow -\infty$ 
	\item[(vi)] Sketch the graph of $y = f(x)$ indicating clearly the features of the curve obtained in aprts (i) to (v) of this quesiton.
\end{itemize}


\item[Ex. 2] Find $\alpha$ and $\beta$ so that the function 

\begin{displaymath}f(x) = \alpha x^3 + \beta x^2+1\end{displaymath}

has a point of inflection at  $\displaystyle \left(-1,2\right)$
\end{itemize}
\subsection*{Ratio Test}
% Red LC book Pages 31

Use the ratio test to show that

\[ \lim\limits_{n=1} \frac{x^n}{n!} \]

is convergent for all $x \in \mathbb{R}$

%=========================================================== %

\subsection*{Simple Integration Exercises}
% Red LC book Pages 16 and 52
\[  \int sin 4x dx \]

\[  \int sin 4x + cos 3x dx \]

\[ \int (1+ \sqrt{x})^2 dx \]

\[ \int  \frac{1}{x^2} dx  \]

\[ \int (2x-1)^2 dx  \]

%=========================================================== %
\subsection*{Integration by Parts}
% Red LC book Pages 175 and 222
\[  \int^{1}_{0} x e^{3x} dx \]

\[  \int^{\frac{\pi}{4}}_{0} x sin (2x) dx \]


%========================================================== %
%======================================================================================================= %
\newpage
\section{Week 9 Integration}
\begin{itemize}
\item[(i)] 
Evaluate the following indefinite integral: 
\[\int 3x^2 +2e^x -1 dx\] 
\item[(ii)] Evaluate the following definite integral: 
\[\int^{9}_{4} \frac{1}{\sqrt{x}} dx\] 
\item[(iii)] Find \[\int (4x +3 + \frac{1}{x^2} dx\]

\item[(iv)] The area enclosed between the curve $y=cos(x)$ and the x-axis between $x=0$ and $x= \frac{\pi}{3}$
\end{itemize}
\newpage
\section*{Week 10 Integration and Numerical Integration}

\begin{enumerate}
	\item \textbf{Integrations by Parts} \\ Evaluate the following expression, using the Integration by Parts" technique
	
	%% - http://tutorial.math.lamar.edu/Classes/CalcII/IntegrationByParts.aspx
	
	\begin{itemize}
		\item[(i)] \[ \int x \sqrt{x+1} dx \]
		\item[(ii)] \[ \int x^5 \sqrt{x^3+1} dx \]
		\item[(ii)] \[ \int e^x cos(x) dx \]
	\end{itemize}
	%======================================================= %
	\item \textbf{Trapezoidal Rule}
	% % - http://www.intmath.com/integration/5-trapezoidal-rule.php
	\item \textbf{Simpson's Rule}
\end{enumerate}
%========================================================================================================= %
\newpage
\section{Week 11 Partial Derivatives}

\begin{enumerate}
\item Revision of Differentiation
\item 
\end{enumerate}
\newpage
\section{Week 12 Partial Derivatives}
%================================================================================%
\end{document}
