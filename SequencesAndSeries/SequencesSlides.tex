\documentclass{beamer}

\usepackage{amsmath}
\usepackage{amssymb}

\begin{document}

%------------------------------------------------------------------------------------ %
\section{Aritmetic Progressions}
\begin{frame}
An arithmetic progression (AP) or arithmetic sequence is a sequence of numbers such that the difference between the consecutive terms is constant. For instance, the sequence $5, 7, 9, 11, 13, 15 \ldots$ is an arithmetic progression with common difference of 2.
\end{frame}
%------------------------------------------------------------------------------------ %

\begin{frame}
\frametitle{Aritmetic Progressions}



If the initial term of an arithmetic progression is $a_{1}$ and the common difference of successive members is d, then the $n-$th term of the sequence ($a_{n}$) is given by:
\[ a_{n}=a_{1}+(n-1)d, \] 
and in general
\[ a_{n}=a_{m}+(n-m)d.\]

\end{frame}
%------------------------------------------------------------------------------------ %

\begin{frame}
	\frametitle{Aritmetic Progressions}
A finite portion of an arithmetic progression is called a \textbf{finite arithmetic progression} and sometimes just called an arithmetic progression. The sum of a finite arithmetic progression is called an arithmetic series.
\newline
\end{frame}
%------------------------------------------------------------------------------------ %

\begin{frame}
	\frametitle{Aritmetic Progressions}
The behavior of the arithmetic progression depends on the common difference d. If the common difference is:
\begin{itemize}
\item Positive, the members (terms) will grow towards positive infinity.
\item Negative, the members (terms) will grow towards negative infinity.
\end{itemize}

\end{frame}
%------------------------------------------------------------------------------------ %

\begin{frame}
	\frametitle{Aritmetic Progressions}
%----------------------------------------------------------------------------------- %
\textbf{Summation of an Arithmetic Progression}
The sum of the members of a finite arithmetic progression is called an arithmetic series. For example, consider the sum:
\[2+5+8+11+14\]
This sum can be found quickly by taking the number n of terms being added (here 5), multiplying by the sum of the first and last number in the progression (here 2 + 14 = 16), and dividing by 2:
\[{\frac  {n(a_{1}+a_{n})}{2}}\]
\end{frame}
%------------------------------------------------------------------------------------ %

\begin{frame}
	\frametitle{Aritmetic Progressions}
In the case above, this gives:
\[2+5+8+11+14={\frac  {5(2+14)}{2}}={\frac  {5\times 16}{2}}=40.\]
This formula works for any real numbers $a_{1}$ and $a_{n}$. For example:
\[\left(-{\frac  {3}{2}}\right)+\left(-{\frac  {1}{2}}\right)+{\frac  {1}{2}}={\frac  {3\left(-{\frac  {3}{2}}+{\frac  {1}{2}}\right)}{2}}=-{\frac  {3}{2}}.\]
\end{frame}
%------------------------------------------------------------------------------------ %

\begin{frame}
	\frametitle{Geometric Progression}
A geometric progression, also known as a geometric sequence, is a sequence of numbers where each term after the first is found by multiplying the previous one by a fixed, non-zero number called the \textit{\textbf{common ratio}}. 

\noindent For example, the sequence $2, 6, 18, 54, \ldots$ is a geometric progression with common ratio 3. 

\noindent Similarly $10, 5, 2.5, 1.25, \ldots$ is a geometric sequence with common ratio 1/2.

\end{frame}
%------------------------------------------------------------------------------------ %

\begin{frame}
	\frametitle{Geometric Progression}

\noindent Examples of a geometric sequence are powers $r_k$ of a fixed number r, such as 2k and 3k. The general form of a geometric sequence is
\[a,\ ar,\ ar^{2},\ ar^{3},\ ar^{4},\ \ldots \]
where $r \neq 0$ is the common ratio and $a$ is a \textit{\textbf{scale factor}}, equal to the sequence's start value.
\end{frame}
%------------------------------------------------------------------------------------ %

\begin{frame}
	\frametitle{Geometric Progression}
\textbf{Summations of Geometric Progressions}
A summation of a geometric progression, a \textit{\textbf{geometric series}}, is the sum of the numbers in a geometric progression. For example:
\[2+10+50+250=2+2\times 5+2\times 5^{2}+2\times 5^{3}.\,\]
\end{frame}
%------------------------------------------------------------------------------------ %

\begin{frame}
	\frametitle{Geometric Progression}
Letting $a$ be the first term (here 2), $m$ be the number of terms (here 4), and r be the constant that each term is multiplied by to get the next term (here 5), the sum is given by:
\[{\frac  {a(1-r^{m})}{1-r}}\]
In the example above, this gives:
\[2+10+50+250={\frac  {2(1-5^{4})}{1-5}}={\frac  {-1248}{-4}}=312.\]
\end{frame}
\end{document}