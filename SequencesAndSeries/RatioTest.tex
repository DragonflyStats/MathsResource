\documentclass{beamer}

\usepackage{amsmath}
\usepackage{amssymb}

\begin{document}
%====================================================================%
\begin{frame}
	\frametitle{Ratio Test}
	In mathematics, the ratio test is a test (or "criterion") for the convergence of a series 
	\[ \sum_{n=1}^\infty a_n,\] where each term is a real or complex number and $a_n$ is nonzero when n is large. 
	
	The test was first published by Jean le Rond d'Alembert and is sometimes known as d'Alembert's ratio test or as the Cauchy ratio test.
\end{frame}
%====================================================================%
\begin{frame}
	\frametitle{Motivation}
	Given the following geometric series:
	\[ \sum_{n=1}^\infty \left(\frac{1}{2}\right)^n = \frac{1}{2} + \frac{1}{4} + \frac{1}{8} + \cdots\]
	
	The quotient $ a_{n+1}/a_n = (1/2)^{n+1}/(1/2)^n$ of any two adjacent terms is 1/2. 
	The sum of the first m terms is given by:
	\[1 - \frac{1}{2^m}.\]
\end{frame}
%====================================================================%
\begin{frame}
	As m increases, this converges to 1, so the sum of the series is 1. On the other hand given this geometric series:
	\[\sum_{n=1}^\infty 2^n = 2 + 4 + 8 + \cdots\]
	
\end{frame}
%====================================================================%
\begin{frame}
	\begin{itemize}
		\item The quotient $a_{n+1}/a_n$ of any two adjacent terms is 2. 
		\item The sum of the first m terms is given by
		$2^{m+1} - 2$,
		which increases without bound as m increases, so this series diverges. 
		\item More generally, the sum of the first m terms of the geometric series $\sum_{n=1}^\infty r^n$ is given by:
		\[\sum_{n=1}^{m} r^n = \frac{r}{r-1} (r^m - 1).\]
	\end{itemize}
\end{frame}
%====================================================================%
\begin{frame}
	Whether this converges or diverges as m increases depends on whether r, the quotient of any two adjacent terms, is 
	less than or greater than 1. Now consider the series:
	\[\sum_{n=1}^\infty \frac{n+1}{n} \left(\frac{1}{2}\right)^n = \frac{2}{1} \cdot \frac{1}{2} + \frac{3}{2} \cdot \frac{1}{4} + \frac{4}{3} \cdot \frac{1}{8} + \cdots\]
\end{frame}
%====================================================================%
\begin{frame}
	This is similar to the first convergent sequence above, except that now the ratio of two terms is not fixed at exactly 1/2:
	\[ \left(\frac{n+1}{n} \left(\frac{1}{2}\right)^n\right)/\left(\frac{n}{n-1} \left(\frac{1}{2}\right)^{n-1}\right) = \frac{n^2-1}{2n^2} = \frac{1}{2} - \frac{1}{2n^2}.\]
	
\end{frame}
%====================================================================%
\begin{frame}
\begin{itemize}	
\item 	However, as n increases, the ratio still tends in the limit towards the same constant 1/2. 
\item The ratio test generalizes the simple test for geometric series to more 
	complex series like this one where the quotient of two terms is not fixed, but in the limit tends towards a fixed value.\item The rules are similar: if the quotient approaches a value less than one, the series converges, whereas if it approaches a value greater than one, the series diverges.
	\end{itemize}
\end{frame}
%====================================================================%
\begin{frame}
\frametitle{The Ratio Test}
	The usual form of the test makes use of the limit
	\[L = \lim_{n\rightarrow\infty}\left|\frac{a_{n+1}}{a_n}\right|.\]
	
\end{frame}
%====================================================================%
\begin{frame}
	
	
	The ratio test states that:
	\begin{itemize}
		\item if $L < 1$ then the series converges absolutely;
		\item if $L > 1$ then the series does not converge;
		\item if $L = 1$ or the limit fails to exist, then the test is inconclusive, because there exist both convergent and divergent series that satisfy this case.
	\end{itemize}
\end{frame}
\end{document}
%====================================================================%
\begin{frame}
	It is possible to make the ratio test applicable to certain cases where the limit L fails to exist, 
	if limit superior and limit inferior are used. 
	
	The test criteria can also be refined so that the test is sometimes conclusive even when L = 1. More specifically, let
	R = \lim\sup \left|\frac{a_{n+1}}{a_n}\right|    and    r = \lim\inf \left|\frac{a_{n+1}}{a_n}\right|.
	Then the ratio test states that:
	
	if R < 1, the series converges absolutely;
	if r > 1, the series diverges;
	if \left|\frac{a_{n+1}}{a_n}\right|\ge 1 for all large n (regardless of the value of r ), the series also diverges; this is because |a_n| is nonzero and increasing and hence a_n does not approach zero;
	the test is otherwise inconclusive.
	If the limit L in (1) exists, we must have L=R=r. So the original ratio test is a weaker version of the refined one.
	Examples[edit]
\end{frame}
%====================================================================%
\begin{frame}
	Convergent because L<1[edit]
	Consider the series
	\sum_{n=1}^\infty\frac{n}{e^n}
	Putting this into the ratio test:
	L = \lim_{n\to\infty} \left| \frac{a_{n+1}}{a_n} \right|
	= \lim_{n\to\infty} \left| \frac{\frac{n+1}{e^{n+1}}}{\frac{n}{e^n}}\right|
	= \frac{1}{e} < 1.
	Thus the series converges.
\end{frame}
%====================================================================%
\begin{frame}
	Divergent because L>1[edit]
	Consider the series
	\sum_{n=1}^\infty\frac{e^n}{n}.
	Putting this into the ratio test:
	L
	= \lim_{n\to\infty} \left| \frac{a_{n+1}}{a_n} \right|
	= \lim_{n\to\infty} \left| \frac{\frac{e^{n+1}}{n+1}}{\frac{e^n}{n}} \right|
	= e > 1.
	Thus the series diverges.
\end{frame}
%====================================================================%
\begin{frame}
	Inconclusive because L=1[edit]
	Consider the three series
	\sum_{n=1}^\infty 1,    \sum_{n=1}^\infty \frac{1}{n^2}   and    \sum_{n=1}^\infty (-1)^n\frac{1}{n}.
	The first series diverges, the second one converges absolutely and the third one converges conditionally. However, the term-by-term magnitude ratios \left|\frac{a_{n+1}}{a_n}\right| of the three series are respectively 1,    \frac{n^2}{(n+1)^2}    and \frac{n}{n+1}. So, in all three cases, we have\lim_{n\rightarrow\infty}\left|\frac{a_{n+1}}{a_n}\right|=1. This illustrates that when L=1, the series may converge or diverge and hence the original ratio test is inconclusive. For the first series \sum_{n=1}^\infty 1, however, as the term-by-term magnitude ratio \left|\frac{a_{n+1}}{a_n}\right|=1 for all n, we can apply the third criterion in the refined version of the ratio test to conclude that the series diverges.
\end{frame}


\end{document}
