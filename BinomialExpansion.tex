Statement of the theorem[edit]
According to the theorem, it is possible to expand any power of x + y into a sum of the form

(x+y)^n = {n \choose 0}x^n y^0 + {n \choose 1}x^{n-1}y^1 + {n \choose 2}x^{n-2}y^2 + \cdots + {n \choose n-1}x^1 y^{n-1} + {n \choose n}x^0 y^n,
where each  \tbinom nk  is a specific positive integer known as a binomial coefficient. (When an exponent is zero, the corresponding power expression is taken to be 1 and this multiplicative factor is often omitted from the term. Hence one often sees the right side written as \binom{n}{0} x^n + \ldots.) This formula is also referred to as the binomial formula or the binomial identity. Using summation notation, it can be written as

(x+y)^n = \sum_{k=0}^n {n \choose k}x^{n-k}y^k = \sum_{k=0}^n {n \choose k}x^{k}y^{n-k}.
The final expression follows from the previous one by the symmetry of x and y in the first expression, and by comparison it follows that the sequence of binomial coefficients in the formula is symmetrical. A simple variant of the binomial formula is obtained by substituting 1 for y, so that it involves only a single variable. In this form, the formula reads

(1+x)^n = {n \choose 0}x^0 + {n \choose 1}x^1 + {n \choose 2}x^2 + \cdots + {n \choose {n-1}}x^{n-1} + {n \choose n}x^n,
or equivalently

(1+x)^n = \sum_{k=0}^n {n \choose k}x^k.
