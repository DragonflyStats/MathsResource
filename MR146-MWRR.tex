
\documentclass{beamer}

\usepackage{amsmath}
\usepackage{amssymb}
\usepackage{graphics}

\begin{document}
%-------------------------------------------------- %
\begin{frame}
\bigskip
{
\Huge
\[ \mbox{Financial Mathematics}  \]
\huge
\[ \mbox{Money-Weighted Rate of Return}  \]
}

{
\LARGE
\[ \mbox{MathsResource.com}  \]
%\[ \mbox{Twitter: @StatsLabDublin} \]
}
\end{frame}
%-------------------------------------------------- %
\begin{frame}
\frametitle{Money-Weighted Rate of Return}
\Large

\end{frame}
%-------------------------------------------------- %
\begin{frame}

% DEFINITION of 'Money-Weighted Rate Of Return'

A measure of the rate of return for an asset or portfolio of assets. 

It is calculated by finding the rate of return that will set the present values of all cash flows and terminal values equal to the value of the initial investment. The money-weighted rate of return is equivalent to the internal rate of return (IRR). 

\end{frame}
%-------------------------------------------------- %
\begin{frame}
Money-weighted returns is a measure of the rate of return for a portfolio that sets the present value of all cash flows and terminal values equal to the initial investment. In other words, the money-weighted rate of return is simply the Internal Rate of Return (IRR). This is in contrast to time weighted returns. In order for firms to meet Global Investment Performance Standards, they must use time-weighted returns as opposed to money-weighted returns. 

In other words, you would solve for IRR in this equation: Amount of Investment = Cash flow 1 / (1+IRR) + Cash flow 2 / (1+IRR)^2 + ...Cash flow n / (1+IRR)^n.
 
\end{frame}
%-------------------------------------------------- %
\begin{frame}

Differences between time-weighted and money-weighted returns
 
The main difference between time weighted and money weighted returns is that time weighted returns
ignores the effect of cash inflows/outflows, whereas money weighted returns incorporates the size and timing of cash flows. 

\end{frame}
%-------------------------------------------------- %
\begin{frame}

The money-weighted rate of return internalizes both the timing and size of external cash flows (such as deposits and withdrawals), whereas time-weighted return correct for this by allowing different "subperiods" to have different returns.
 

\end{frame}
%-------------------------------------------------- %
\end{document}


