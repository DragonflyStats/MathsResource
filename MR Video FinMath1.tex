\documentclass{beamer}

\usepackage{amsmath}
\usepackage{amssymb}

\begin{document}

\begin{frame}
\frametitle{Financial Mathematics}
Example 3: Solving for the period needed to double money
Consider a deposit of £100 placed at 10\% (annual). How many years are needed for the value of the deposit to double to £200?
\end{frame}
%-------------------------------------------------- %
\begin{frame}
\frametitle{Financial Mathematics}
Using the algrebraic identity that if:

\[x \ = \ b^y \]  
then
\[y \ = \ {\log (x) \over \log(b)}\]  
The present value formula can be rearranged such that:
\[y \ = \ {\log ({FV \over PV}) \over \log (1+i)} \ = \  {\log ({200 \over 100}) \over \log (1.10)} \ =\ 7.27   {(years)} 
\] 
\end{frame}
%-------------------------------------------------- %
\begin{frame}
\frametitle{Financial Mathematics}
This same method can be used to determine the length of time needed to increase a deposit to any particular sum, as long as the interest rate is known. For the period of time needed to double an investment, the Rule of 72 is a useful short-cut that gives a reasonable approximation of the period needed.
\end{frame}
\end{document}