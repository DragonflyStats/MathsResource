
\documentclass[11pt]{article} % use larger type; default would be 10pt

\usepackage[utf8]{inputenc} % set input encoding (not needed with XeLaTeX)

\usepackage{geometry} % to change the page dimensions
\geometry{a4paper} % or letterpaper (US) or a5paper or....

\usepackage{framed}
\usepackage{graphicx} % support the \includegraphics command and options
\usepackage{subfiles}
\usepackage{booktabs} % for much better looking tables
\usepackage{array} % for better arrays (eg matrices) in maths
\usepackage{paralist} % very flexible & customisable lists (eg. enumerate/itemize, etc.)
\usepackage{verbatim} % adds environment for commenting out blocks of text & for better verbatim
\usepackage{subfig}
\usepackage{fancyhdr} % This should be set AFTER setting up the page geometry
\pagestyle{fancy} % options: empty , plain , fancy
\renewcommand{\headrulewidth}{0pt} % customise the layout...
\lhead{MA4702}\chead{MID TERM EXAM 1 session B}\rhead{29th February 2016}
\lfoot{}\cfoot{\thepage}\rfoot{}
\voffset=-1.5cm
\oddsidemargin=0.0cm
\usepackage{amsmath}
\textwidth = 470pt
\usepackage{sectsty}
\allsectionsfont{\sffamily\mdseries\upshape} % (See the fntguide.pdf for font help)
\usepackage[nottoc,notlof,notlot]{tocbibind} % Put the bibliography in the ToC
\usepackage[titles,subfigure]{tocloft} % Alter the style of the Table of Contents
\renewcommand{\cftsecfont}{\rmfamily\mdseries\upshape}
\renewcommand{\cftsecpagefont}{\rmfamily\mdseries\upshape} % No bold!
\begin{document}
	
	% % Part A (4 Marks)
	% 1 Floor and Ceiling Functions
	% 2 Evaluate Function for two values
	% 2 Cross Multiplication
	
	% % PArt B Limits
	% Simple one
	% Divide by Zero Problem
	% Divide by Infinity Problem
	
	% Part C sequences and Series
	% 2 Marks Sum of integers
	% 3 Geometric Sequence
	
	\Large
\newpage

\begin{framed}
\begin{description}
\item[NAME]  ..........................................................................................
\bigskip
\item[STUDENT ID] ............................................................................
\bigskip
\item[DEGREE] .....................................................................................
\end{description}
\end{framed}
\noindent Are You Approved for Extra Time in Exams (LENS students only)  .......
\section*{Examination Procedures}
\begin{itemize} 
	\item This exam will start at 13:05, and will last 45 minutes.
		\item Each question will be worth 1 Mark each. There are 10 Marks worth of questions.
	\item All questions must be attempted (LENS students please see below)
		\item Write \textbf{all of your answers} in the exam script. Write the script number on any other documents you submit.
		\item It is your responsibility to return the script to the collection box. 
		An audit of scripts will take place immediately after the exam. If your script is not accounted for in that audit,  you are deemed to be absent, and will receive no marks.
	
	\item \textbf{LENS Students: }
	Specifically approved LENS students have to answer any selection of questions that have an aggregate total of 7 Marks.  

	
	
\end{itemize}

%============================================================================ %

\newpage


\LARGE
\section*{Formula Sheet}

\noindent \textbf{Logarithms}\\
	If $a^b = c$ then $\mbox{log}_a c = b$.
	
\medskip

\noindent \textbf{Change of Base Formula}

\[ \log_A(B) = \frac{ \log_e(B) }{ \log_e(A) }  \]
\smallskip

\noindent \textbf{Sum and Difference of Two Cubes}
	\[ a^3 + b^3 = (a-b)(a^2 - ab + b^2)\]
	\[ a^3 - b^3 = (a-b)(a^2 + ab + b^2)\]
	\medskip
	%======================================== %
	
\noindent \textbf{Sequences and Series}\\
	Finite Series
	\[ \sum_{i=1}^{n} i = \frac{n(n+1)}{2}\]
	\medskip
	Arithmetic Series:
	\[ S_n = \frac{n}{2} \left(2a + (n-1) d \right)\]
	\medskip
	Geometric Series:
	\[ S_n = a\left(\frac{1-r^n}{1-r}\right)\]
	
	\[ S_\infty = \frac{a}{1-r}\]


	%==========================================================================================%
	
	

\newpage

%================================================================================== %
\subsection*{Part A Fundamentals of Mathematics (2 Marks) } %1AREADY

\begin{itemize}
	%========================================================%
	
	\item[(i)] (1 Mark) Determine the values of A and B from the following expression
	\[  \frac{6}{x^2-1} = \frac{A}{x-1} + \frac{B}{x+1}\]
	% %	\item[(ii)] (2 Marks) Determine whether or not the function \[f(x) = x cos(x)\] is odd, even or neither.
	\vspace{2.5cm}
	
%	\item[(ii)] (1 Mark) Evaluate the function for the values of  $ x = \{0.25, 0.5 , 0.75 \}$
%	{ 
%		\Huge
%	\[  f(x) = e^{x^2+5}\]
%}
%
%	\vspace{2.5cm}


		
	\item[(ii)] (1 Mark) Find the value of $x$
	{ \huge	\[e^{x^2-7} = 8103.084 \]
}
\end{itemize}
\newpage
.
%======================================================== %
\newpage
\subsection*{Part B Limits of Functions (4 Marks)}


%------------------------------------%
%Limits
\begin{itemize}
\item[(iii)](1 Mark)
Evaluate the following limits
\[  \lim_{x\to 4} \frac{x^3-64}{x-4}      \]

\vspace{2cm}
\item[(iv)](1 Mark)

Evaluate the following limits

\[\lim_{x \to 3 } \frac{x^2-9}{x-3}\]  %ANSWER = 6

\end{itemize}



%------------------------------------%

% Early Questions 
%
%
%\[  \lim_{x\to -1} \frac{x+1}{x^2+x}      \]
%\[  \lim_{x\to 1} \frac{1}{x^2+1}          \]
%\[  \lim_{x\to 1} x^ + 5x - \frac{1}{2-x}   \]
%\[  \lim_{x\to 1} \frac{x^2-1}{x^2+2x-3}    \]

\newpage

\begin{enumerate}
%	\item[(iv)]  Compute the limit of the following function
%	
%	\[\lim_{x \to 3 } \frac{x^2-15}{x-3}\]
%	\vspace{2.9cm}
%	
%	\item[(v)]  Compute the limit of the following function
%	\[\lim_{x \to 4 } \frac{x^2-x-12}{x-4}\]
%	\vspace{2.9cm}

	\item[(v)]  Compute the limit of the following function
	\[ \lim_{x \to \infty } \frac{6 + x^2 - 7x^3 }{4x^3 - 6x + 2} \]
	 \vspace{2cm}	
		\item[(vi)](1 Mark)
		Evaluate the following limits
		\[  \lim_{x\to \infty } \frac{x+1}{x^2+x}      \]
\end{enumerate}
\newpage
.
\newpage
%===================================================%
\subsection*{Part C Sequences and Series (4 Marks)}

\begin{enumerate} %GOOG
	\item[(vii)](1 Mark) Compute the following summation
	
	\[ \sum_{i=5}^{50} i \]

	\vspace{5.5cm}
	\item[(viii)] (1 Mark) Find the sum of the following geometric series: 
		\[2 + 6 + 18 +  \ldots + 39366		\]



\newpage
.
\newpage	
	\item[(ix)] (1 Mark) Express the following repeating decimal number as a simple fraction. Show your workings.
	
	\[0.297297297....\]
\textit{(Simplify your answer such that both the numerator and the denominator are prime numbers.)}
	\vspace{5.5cm}

	\item[(x)] (1 Marks) The three terms below are three successive terms in an aritmetic progression. Compute the value for $x$, and the common difference $d$
		\[ \ldots,\; 3x-1,\; 2x+6,\;	5x-7,\; \ldots		  \]
		
\end{enumerate}

\end{document}
\begin{itemize}



% Fundamentals

\item[(v)](1 Mark)

\[ \log_e (2x) = 5\]

\item[(vi)](1 Mark)

\[ \cosh(2x) \]
%------------------------------------%
% Sequences and Series

\item[(vii)](1 Mark)

\[u_n = (2n)! \]

\item[(viii)](1 Mark)

\item[(ix)](1 Mark)
% Geometric Series

\item[(x)](1 Mark)


%------------------------------------%
\end{itemize}



%=============================================%


% End of Semester Question






Limits of Piecewise Functions[edit]
Evaluate the following limits or state that the limit does not exist.

37. Consider the function

\[ f(x) = \begin{cases} (x-2)^2 & \mbox{if }x<2 \\ x-3 & \mbox{if }x\geq 2. \end{cases} \]


a.  \[\lim_{x\to 2^-}f(x) \]
0
b.  \[\lim_{x\to 2^+}f(x)\] 
-1




% Questions 30 and 31

\[ \lim_{x\to \infty} \frac{2x^2-32}{x^3-64} \]

Answer: 0

\end{document}
Status API Training Shop Blog About Pricing
© 2016 GitHub, Inc. Terms Privacy Security Contact Help
